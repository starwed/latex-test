\chapter{Spin one-half}

%skeleton of calculation

%define spinors and structures
\subsection{Conventions}
Spinors:

\beq
	\sr = \begin{pmatrix} \eta \\ \xi \end{pmatrix}
\eeq

\beqa
	\eta &=& \left( 1  - \frac{\v{p}^2}{8m^2} \right ) w	\\
	\xi &=& 	\frac{ \gv{\sigma} \cdot \v{p}}{2m} \left(1 - \frac{3\v{p}^2}{8m^2} \right ) w	\\
\eeqa

structures (in the useful represenation)
\beq
	\gamma^0 = \Mblock{1}{0}{-1}{0}
\eeq

\beq
	\gamma^i = \Mblock{0}{\sigma_i}{-\sigma_i}{0}
\eeq	
\beq
	\sigma^{\mu \nu} = i \frac{1}{2} [ \gamma^\mu, \gamma^\nu]
\eeq

\beq
	[\gamma^0, \gamma^i] = 2 \gamma^0 \gamma^i = 2 \Mblock{0}{\sigma_i}{\sigma_i}{0} 
\eeq


\beq
	\gamma^i \gamma^j = \Mblock{0}{\sigma_i}{-\sigma_i}{0} \Mblock{0}{\sigma_j}{-\sigma_j}{0}
		=	\Mblock{ - \sigma_i \sigma_j}{0}{0}{ - \sigma_i \sigma_j}
\eeq
\beq
	[\gamma^i, \gamma^j] = \Mblock{ [\sigma_j, \sigma_i]}{0}{0}{ [\sigma_j, \sigma_i]}
		=	i\epsilon_{jik} \Mblock{ \sigma_k }{0}{0}{\sigma_k}
		=	-i\epsilon_{ijk} \Mblock{ \sigma_k }{0}{0}{\sigma_k}
\eeq

\subsection{QED calculations}
%First form
First form:
\beq
	(p + p')^\mu \srb \sr  = (p + p')^\mu \left( \eta^\dagger \eta - \xi^\dagger \xi \right ) 
\eeq

\beq
	= (p + p')^\mu \left \{
		w^\dagger \left( 1 - \frac{ \v{p'}^2}{8m^2} \right )  \left( 1 - \frac{ \v{p}^2}{8m^2} \right ) w
		- w^\dagger \left( \frac{ \gv{\sigma} \cdot \v{p'}}{2m} \frac{ \gv{\sigma} \cdot \v{p}}{2m} \right ) w \right \}
\eeq 

\beq
	= (p + p')^\mu w^\dagger \left( 
		1 -  \frac{ \v{p}^2 + \v{p'}^2}{8m^2} - \frac{ \gv{\sigma} \cdot \v{p'} \gv{\sigma} \cdot \v{p} }{4m^2} 
		\right ) w
\eeq


Second form:
For the term $\srb \gamma^\mu \sr$ it'll be necessary to treat the spatial/time-like indices separately.

time-like
\beq
	\srb \gamma^0 u = u^\dagger u
\eeq

\beq
	= \eta^\dagger \eta + \xi^\dagger \xi 
\eeq

\beq
	= 	w^\dagger \left( 1 - \frac{ \v{p'}^2}{8m^2} \right )  \left( 1 - \frac{ \v{p}^2}{8m^2} \right ) w
		+ w^\dagger \left( \frac{ \gv{\sigma} \cdot \v{p'}}{2m} \frac{ \gv{\sigma} \cdot \v{p}}{2m} \right ) w 
\eeq

\beq
	=	 w^\dagger \left( 
		1 -  \frac{ \v{p}^2 + \v{p'}^2}{8m^2} + \frac{ \gv{\sigma} \cdot \v{p'} \gv{\sigma} \cdot \v{p} }{4m^2} 
		\right ) w
\eeq

spatial
\beq
	\srb \gamma^i \sr = \sr^\dagger \gamma^0 \gamma^i \sr
\eeq

\beq
	= \srb^\dagger \begin{pmatrix}
		0 & \sigma_i \\ \sigma_i & 0 		
	\end{pmatrix} \sr
\eeq

\beq
	= \eta^\dagger \sigma_i \xi + \xi^\dagger \sigma_i \eta
\eeq

\beq
	= w^\dagger \left\{
		\left( 1 - \frac{ \v{p'}^2}{8m^2}   \right ) \sigma_k \left( 1 - \frac{3\v{p}^2}{8m^2} \right )
			- \left( 1 - \frac{ 3\v{p'}^2}{8m^2} \right ) \sigma_k \left( 1 - \frac{\v{p}^2}{8m^2} \right )
			-	\frac{ \gv{\sigma} \cdot \v{p'} \sigma_k \gv{\sigma} \cdot \v{p} }{4m^2}
	\right\}
\eeq


Third type (tensor)
\beq
	\srb  \frac{i}{2m} q_j \sigma^{ij} \sr 
		=  \frac{i \epsilon_{ijk} q_j}{2m} \srb \Mblock{0}{\sigma_k}{\sigma_k}{0} \sr
\eeq
		
\beq
	\frac{i \epsilon_{ijk} q_j}{2m} \left( \eta^\dagger \sigma_k \eta - \xi^\dagger \sigma_k \xi \right )
\eeq

\beq
	\frac{i \epsilon_{ijk} q_j}{2m} w^\dagger \left \{
		\left( 1 - \frac{\v{p'}^2}{8m^2} \right ) \sigma_k \left( 1 - \frac{\v{p'}^2}{8m^2} \right )- \frac{ \gv{\sigma} \cdot \v{p'} \sigma_k \gv{\sigma} \cdot \v{p} }{4m^2} w
	\right \}
\eeq

Need triple sigma identity
\beq
	\sigma_a \sigma_b \sigma_c = \sigma_a (\delta_{bc} + i\epsilon_{bcd}\sigma_d)
		=	\sigma_a \delta_{bc} - \sigma_b \delta_{ca} + \sigma_c \delta_{ab} + i \epsilon_{abc}	
\eeq

Then using above
\beq
	\srb  \frac{i}{2m} q_j \sigma^{ij} \sr 
		= \frac{i \epsilon_{ijk} q_j}{2m} w^\dagger \left \{
			\sigma_k \left( 1 - \frac{\v{p'}^2  + \v{p}^2}{8m^2} \right ) - \frac{ \gv{\sigma} \cdot (\v{p} + \v{p'})p_k - \sigma_k \v{p} \cdot \v{p'} + i \epsilon_{akc} q_a p_c }{4m^2} 
		\right \} w 
\eeq

The 'time-like' part of the tensor term
\beq
	\srb  \frac{i}{2m} q_j \sigma^{0j} \sr
		=	 - \frac{q_j}{2m} \srb \gamma^0 \gamma^j \sr 		
 \eeq
 
 \beq
 	=  - \frac{q_j}{2m} \sr^\dagger \gamma^j \sr
 \eeq
 
 \beq
 	= - \frac{q_j}{2m}  \left( \eta^\dagger \sigma_j \chi - \chi^\dagger \sigma_j \eta \right )
 \eeq
 
\beq
	= - \frac{q_j}{2m}  w^\dagger \left \{
		\left(1 - \frac{\v{p'}^2}{8m^2} \right )  \frac{ \sigma_j \gv{\sigma} \cdot \v{p} }{2m} \left(1 - \frac{3\v{p}^2}{8m^2} \right )
		- \left(1 - \frac{3 \v{p'}^2}{8m^2} \right ) \frac{\gv{\sigma} \cdot \v{p'} \sigma_j  }{2m} \left(1 - \frac{\v{p}^2}{8m^2} \right )
	\right \} w
\eeq
Dropping terms quadratic in q, all $p'$ can be written just as $p$.
\beq
	\approx - \frac{q_j}{2m}  w^\dagger \left \{
		\frac{ \sigma_j \gv{\sigma} \cdot \v{p} - \gv{\sigma} \cdot \v{p} \sigma_j }{2m} \left(1 - \frac{\v{p}^2}{2m^2} \right )
	\right \} w
\eeq		

\beq
	=  \frac{q_j}{2m}  w^\dagger \left \{
		\frac{ i\epsilon_{ijk}\sigma_k p_i  }{2m} \left(1 - \frac{\v{p}^2}{2m^2} \right )
	\right \} w
\eeq

\beq
	=  w^\dagger \left \{
		\frac{ i\epsilon_{ijk} p_i q_j \sigma_k   }{4m^2} \left(1 - \frac{\v{p}^2}{2m^2} \right )
	\right \} w
\eeq

			


 \section{Fouldy-Wouthyusen approach}
 
 

\section{Equations of motion}


The regular Dirac Lagrangian can be modified by including a term responsible for the anomalous magnetic moment.  The added term is $\mu' \bar{\Psi} \sigma^{\mu\nu}F_{\mu\nu} \Psi$, where $\mu' = \frac{g-2}{2}\mu_0$

The Lagrangian then is
\begin{eqnarray*}
\mathcal{L} &=&	
	\bar{\Psi}(\cancel{p} - m)\Psi + \frac{1}{2} \mu' \bar{\Psi} \sigma^{\mu\nu}F_{\mu\nu} \Psi	
\end{eqnarray*}

The Euler-Lagrange equations give us
\[
	\left( (p_\mu- eA_\mu)\gamma^\mu -m + i\mu' \frac{1}{4}[ \gamma^\mu, \gamma^\nu]F_{\mu\nu} \right) \Psi
		=0
\]
From this equation of motion we'll derive exact relations between the upper and lower components of $\Psi$.

First we'll replace $F_{\mu\nu} \equiv \partial_\mu A_\nu - \partial_\nu A_\mu$ with terms involving the electric and magnetic fields.  We use the antisymmetry of $\sigma^{\mu\nu} \equiv \frac{1}{2}[\gamma^\mu, \gamma^\nu]$, and that we deal with time-independent fields.

\begin{eqnarray*}
	  {[\gamma^0, \gamma^i]}
		&=&  \begin{pmatrix}	0 & 2\sigma_i \\ 2\sigma_i & 0\end{pmatrix}	\\
	  {[\gamma^i, \gamma^j ]}
		&=&	 \begin{pmatrix}	-2i\epsilon_{ijk}\sigma_k & 0 \\ 0 & -2i\epsilon_{ijk}\sigma_k\end{pmatrix}	\\
	F_{\mu\nu} \sigma^{\mu\nu} &=& F_{i}\sigma^{ij} - F_{0i}\sigma^{0i} 	- F_{i0}\sigma^{i0} +F_{00}\sigma^{00}	\\
		&=&	F_{ij} \sigma^{ij} -2F_{0i} \sigma^{0i}	\\
		&=&	2 \partial_i A_j \sigma^{ij} - 2\partial_i \Phi \sigma^{0i}	\\
		&=&	-2i \begin{pmatrix} \sigdot{B} & 0 \\ 0 & \sigdot{B}\end{pmatrix}	
			-2 \begin{pmatrix} 0 & \sigdot{E} \\ \sigdot{E} & 0 \end{pmatrix}	
\end{eqnarray*}





Considering $\Psi = \begin{pmatrix} \phi \\ \chi \end{pmatrix}$ as a bispinor
\[
	\left\{
		\begin{pmatrix}
			p_0 - e\Phi	- m &	0	\\
			0	&	-p_0 + e\Phi - m	\\
		\end{pmatrix}
		+
		\begin{pmatrix}	0 & -\sigdotg{\pi} \\  \sigdotg{\pi} & 0 \end{pmatrix} 
		+\mu'\left [
			\begin{pmatrix}
				\sigdot{B} & 0 \\ 0 & \sigdot{B}
			\end{pmatrix}
			-i \begin{pmatrix}
				0 & \sigdot{E} \\ \sigdot{E} & 0
			\end{pmatrix}
		\right ] 
	\right\} \begin{pmatrix} \phi \\ \chi \end{pmatrix}
		= 0
\]
This gives rise to exact coupled equations for $\phi$ and $\chi$.

\subsection{NR limit}
We're interested in the nonrelativistic limit, in which we can treat the amplitude of $\chi$ as much smaller than $\phi$.  So we'll solve the coupled equations for $\phi$ to the desired order in $\frac{p}{m}$.  Here we're interested in finding the nonrelativistic Hamiltonian to $\mathcal{O}(mv^4)$.  We can also throw away terms proportional to $\mu'^2$, as well as terms nonlinear in the magnetic field.  In taking the relativistic limit we'll use the nonrelativistic energy $\epsilon = p_0 - m$.



Considering the order of various types of terms, we can see that $\epsilon \sim mv^2$, $\Phi\sim mv^2$, $\v{\pi} \sim mv$, and $\v{E} \sim m^2v^3$.

Under these approximations we first find $\chi$.
\begin{eqnarray*}
	\left( \epsilon + 2m - \mu' \sigdot{B} \right ) \chi &=& \left( \sigdotg{\pi} - i\sigdot{E} \right) \phi	\\
	\chi &\approx&	\frac{1}{2m} \left ( 1- \frac{\epsilon - e\Phi - \mu' \sigdot{B}}{2m} \right ) (\sigdotg{\pi} - i\mu' \sigdot{E} )\phi
\end{eqnarray*}
Then, solving for $\epsilon\phi$,
\begin{eqnarray*}
	\epsilon \phi 	&=& (e\Phi - \mu' \sigdot{B} )\phi + (\sigdot{E} + \sigdotg{\pi}) \chi	\\
					&\approx& \left \{
		e\Phi - \mu' \sigdot{B} + \frac{ \exminus \explus}{2m}	\right. \\
		&& \left. +\frac{1}{4m^2} \exminus (\mu' \sigdot{B} - [\epsilon - e\Phi]) \explus 	
	\right \} \phi		\\
		&=& \left \{
				e\Phi - \mu' \sigdot{B} + \frac{(\sigdotg{\pi})^2 - i\mu' [\sigdotg{\pi}, \sigdot{E}]}{2m}
				+\frac{1}{4m^2} \sigdotg{\pi} (\mu' \sigdot{B} - [\epsilon - e\Phi]) \sigdotg{\pi} 
			\right \} \phi
\end{eqnarray*}

We have an expression for the kinetic energy $\epsilon$ of the particle, in terms of operators.  This will give us the nonrelativistic Hamiltonian. Since the leading order term in H is $\mathcal{O}(mv^2)$, this suggests we split it into two parts: $H = H_0 + H_1 +\mathcal{O}(mv^6)$, where $H_1$ consists of only $\mathcal{O}(mv^4)$ terms.  
As written our expression for $H_1$ contains $\epsilon$, so we'll have to solve for it perturbatively.

Our two terms are
\begin{eqnarray*}
	\hat{H_0} 
			&=&  e\Phi - \mu' \sigdot{B} + \frac{ (\sigdotg{\pi})^2}{2m}	\\
	\hat{H_1} 
			&=& -\frac{i\mu'}{2m}  [\sigdotg{\pi}, \sigdot{E}]
				+\frac{1}{4m^2} \sigdotg{\pi} (\mu' \sigdot{B} - [\epsilon - e\Phi]) \sigdotg{\pi} 
\end{eqnarray*}	
To eliminate $\epsilon$ from $H_1$
\begin{eqnarray*}
	\epsilon \phi
			&=& \left(\hat{H_0} + \mathcal{O}(mv^4) \right)\phi 										\\
			&=& \left(e\Phi - \mu' \sigdot{B} + \frac{ (\sigdotg{\pi})^2}{2m}	\right) \phi			\\
	(\sigdotg{\pi})^2 (\epsilon - e\Phi) \phi		
			&=&	(\sigdotg{\pi})^2 \left ( \frac{(\sigdotg{\pi})^2}{2m}- \mu' \sigdot{B} \right )		\\
	\sigdotg{\pi} (\epsilon - e\Phi) \sigdotg{\pi}\phi
			&=&	\left( \frac{(\sigdotg{\pi})^4}{2m} 
				- \mu' (\sigdotg{\pi})^2 \sigdot{B} 
				- \sigdotg{\pi}[e\Phi, \sigdotg{\pi}]\right ) \phi
\end{eqnarray*}
So we can now write down $H_1$
\[
	\hat{H}_1	=
		-\frac{i\mu'}{2m}  [\sigdotg{\pi}, \sigdot{E}]
		+\frac{1}{4m^2}\left(
			\mu' \sigdotg{\pi} \sigdot{B} \sigdotg{\pi}
			- \frac{(\sigdotg{\pi})^4}{2m} 
			+ \mu' (\sigdotg{\pi})^2 \sigdot{B} 
			+ \sigdotg{\pi}[e\Phi, \sigdotg{\pi}]
		\right)
\]
To simplify, we'll eliminate terms of second order in the magnetic field.  We'll also use the following identities:
\begin{eqnarray*}
\{ \sigdot{B}, \sigdot{p} \}	&=&		2\v{B}\cdot \v{p}	\\
(\sigdotg{\pi})^2			&=&		\pi^2 - e\sigdot{B}		\\
\end{eqnarray*} 
\[	[\Phi, \sigdotg{\pi}]	= -i\sigdot{E} \]

Then
\[
	\hat{H_0} 
			=  e\Phi - \mu' \sigdot{B} + \frac{ \pi^2}{2m} - \frac{e}{2m} \sigdot{B}	\\
\]
\[
	\hat{H}_1	=
		- \frac{\pi^4}{8m^3}  
		+ e\frac{p^2}{4m^3} \sigdot{B}
		-i \frac{\sigdotg{\pi} \sigdot{E}}{4m^2} 
		+ \mu' \left(	
			\frac{ \sigdot{p} \v{B}\cdot \v{p} }{2m^2}
			-\frac{i[\sigdotg{\pi}, \sigdot{E}]}{2m} 
		\right )
\]


\section{FW Transform}

To find a complete description of a single nonrelativistic particle in normal quantum mechanics, we must work in a basis where the lower component $\chi$ is truly negligible, at least at the desired order.  While there exists a formal technique for finding this Fouldy-Wouthyusen transformation, for our purposes we can simply demand that the wave function after transformation $\phi_s = (1 + \Delta)\phi$ obeys the normalization $<\phi_s, \phi_s> = 1$.

To find the necessary transformation, we can use our expression for $\chi$ in terms of $\phi$ to find the normalization
\begin{eqnarray*}
	\int d^3x (\phi^\dagger \phi + \chi^\dagger \chi) 	
		&=& \int d^3x 	\left[ \phi^\dagger \phi + \left (\frac{\sigdotg{\pi}}{2m} \phi \right)^\dagger
										   \left (\frac{\sigdotg{\pi}}{2m} \phi \right)
						\right ]	\\
		&=& \int d^3x 	\phi^\dagger \left[ 1 + \frac{ (\sigdotg{\pi})^2 }{4m^2} \right ] \phi
\end{eqnarray*}

Since we know that $<\Psi, \Psi>=1$ this shows that if $\phi = \left( 1 - \frac{ (\sigdotg{\pi})^2 }{8m^2} \right )\phi$, the Schrodinger wave functions are properly normalized.


We now need to find the form of $\hat{H}$ after this transformation:
\begin{eqnarray*}
\epsilon \phi = (\hat{H}_0 + \hat{H}_1) \phi \\ 
\epsilon (1 - \frac{(\sigdotg{\pi})^2}{8m^2})\phi_S =  (\hat{H}_0 + \hat{H}_1) (1 - \frac{(\sigdotg{\pi})^2}{8m^2})\phi_S \\
\epsilon \phi_S =  (1 + \frac{(\sigdotg{\pi})^2}{8m^2})(\hat{H}_0 + \hat{H}_1) (1 - \frac{(\sigdotg{\pi})^2}{8m^2})\phi_S \\
\epsilon \phi_S = \left (  \hat{H}_0 + \frac{1}{8m^2}[(\sigdotg{\pi})^2, \hat{H}_0] + \hat{H}_1 \right )\phi_S \\
\end{eqnarray*}

So under the FW transformation, the leading order term is unchanged, and the second order term is:
$$ \hat{H}_1 \to \hat{H}_1' = \hat{H}_1 + \frac{1}{8m^2}[(\sigdotg{\pi})^2, \hat{H}_0] $$

$ [(\sigdotg{\pi})^2, \hat{H}_0] $ can be simplified.  
$$[(\sigdotg{\pi})^2, \hat{H}_0] = [(\sigdotg{\pi})^2, e\Phi - \mu' \sigdot{B} + \frac {(\sigdotg{\pi})^2}{2m}]$$

Obviously $(\sigdotg{\pi})^2$ commutes with itself, and since $\sigdot{B}$ is constant, that commutator also vanishes:
$$[(\sigdotg{\pi})^2, \mu' \sigdot{B}] =[\pi^2 - e \sigdot{B}, \mu'\sigdot{B}] = 0 $$

Leaving:
\begin{eqnarray*}
[(\sigdotg{\pi})^2, \hat{H}_0] &=& [(\sigdotg{\pi})^2, e\Phi] \\
		&=&	e( \sigdotg{\pi}[\sigdotg{\pi},  \Phi] + [\sigdotg{\pi},  \Phi]\sigdotg{\pi}) \\
		&=&	i e( \sigdotg{\pi} \sigdot{E} + \sigdot{E} \sigdotg{\pi} ) \\
\end{eqnarray*}
So,
\begin{eqnarray*}
\hat{H}_1' 
	&=&  \hat{H}_1 + \frac{i e}{8m^2}( \sigdotg{\pi} \sigdot{E} + \sigdot{E} \sigdotg{\pi} )	\\
	&=&	 -\frac{\pi^4}{8m^3} + e \frac{p^2}{4m^3}\sigdot{B} - \frac{ie}{8m^2}[\sigdot{E},\sigdotg{\pi}]
		 + \mu' \left( \frac{ (\sigdot{p}) (B \cdot p) }{2m^2} - \frac{i}{2m}[\sigdot{E},\sigdotg{\pi}] \right)
\end{eqnarray*}



\subsection*{NR Hamiltonian}
Now we'll write down the entire Hamiltonian, using $\mu' = \frac{g-2}{2}\mu_0 = \frac{g-2}{2}\frac{e}{2m} $.
\begin{eqnarray*}
H	&=&
		e\Phi  + \frac{ \pi^2}{2m} -\frac{\pi^4}{8m^3} - (1 + \frac{g-2}{2})\frac{e}{2m} \sigdot{B}	\\
	&&
		 + e \frac{p^2}{4m^3}\sigdot{B} + \frac{ie}{8m^2}(1 + (g-2))[\sigdot{E},\sigdotg{\pi}]
		 + (g-2) \frac{e}{2m}  \frac{ (\sigdot{p}) (\v{B} \cdot \v{p}) }{4m^2}  	\\
	&=&
		e\Phi  + \frac{ \pi^2}{2m} -\frac{\pi^4}{8m^3} 
	\\&&
		- \frac{e}{2m} \left\{
			\frac{g}{2}\sigdot{B} - \frac{p^2}{2m^2}\sigdot{B}
			- (g-2) \frac{ (\sigdot{p}) (\v{B} \cdot \v{p}) }{4m^2} 
			+(g-1) \gv{\sigma} \cdot (\v{E} \times \gv{\pi})
		\right \}	\\
	&=&
		e\Phi  + \frac{ \pi^2}{2m} -\frac{\pi^4}{8m^3} 
	\\&&
		- \frac{e}{2m} \left\{
			\frac{g}{2} \left( 1 - \frac{p^2}{2m^2} \right) \sigdot{B} + \frac{g-2}{2} \frac{p^2}{2m^2}\sigdot{B}
			- \frac{g-2}{2} \frac{ (\sigdot{p}) (\v	{B} \cdot \v{p}) }{2m^2} 
			+\left(\frac{g}{2} + \frac{g-2}{2}\right) \gv{\sigma} \cdot (\v{E} \times \gv{\pi})
		\right \}	\\	
\end{eqnarray*}
 
 
 