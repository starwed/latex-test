\chapter{Introduction}
We wish to calculate the gyromagnetic ratio of a particle of arbitrary spin in a loosely bound state system.  We can write down an effective Lagrangian which will capture all the necessary effects.  Our job is then to calculate the coefficients of this effective particle.  By considering the constraints which exist on the electromagnetic current in a general relativistic theory, we can obtain a simple form which holds for arbitrary spin, and then use this to fix the coefficients of the relativistic theory.

While for a particle of general spin we do not have an exact Lagrangian, we do have one for a spin-$1$ theory.  We can use this Lagrangian to perform the same calculation as above, but in an exact theory.  We derive the nonrelativistic potential first from the equations of motion, and then from diagrammatic calculations, and obtain a result that agrees with the more general calculation.  
\subsection{Description of problem}
We consider a loosely bound system two particles of arbitrary spin, and wish to calculate the correction to the gyromagnetic ratio of the particles.  So we consider the case of the two particles interacting with both each other, via a Coulomb potential, and a very weak external magnetic field.
\subsubsection{Experimental Motivation}

Talk about 


\subsubsection{Approach}
This system allows for several simplifications.  First, because the system is loosely bound all energy scales are nonrelativistic.  
%TODO details of nonrelativistic nature
Second, the magnetic field we consider is both very weak and constant.  So we can ignore all corrections which involve derivatives of the magnetic field or which are quadratic in its strength.

The approach we'll take is to first consider the electric potential as an external field acting on a single particle.  Next we can consider the bound system as a whole sitting in an external magnetic field, and take into account recoil effects.

%Discuss to what order we need to do the calculations

\subsection{Theoretical Background}

(Here we talk about existing theoretical work in this area)
\subsubsection{Spin-$1/2$}

(Here talk about the approach that works for spin 1/2, and why it breaks down in the general case.)

\subsubsection{BMT equation}

(Here discuss the BMT equation, which holds for general spin and can be used to derive the g-factor corrections)

\subsubsection{Khriplovich general spin work}
(Discuss the work Khriplovich did with general spinors.)
