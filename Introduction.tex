%Abstract like thing?
The binding corrections to the gyromagnetic ratio $g$ are investigated for a particle of arbitrary spin.  The corrections are worked out for specific spin-half and spin-one theories, before considering the more general case.  By considering the constraints lying upon the relativistic theory, an effective nonrelativistic Lagrangian is developed.

\chapter{Introduction}


A way to calculate the binding corrections to the $g$-factor of a charged particle of arbitrary spin is desired.

The $g$ factor in this sense can be defined by the energy separation of two particles which differ only by spin orientation.  A free electron, with its spin oriented along the same axis as a weak magnetic field $B_z\hat{z}$, has energy
\beq
 E = - \gv{\mu} \cdot \v{B} = - g \frac{e}{2m}s_z B_z
\eeq
The difference between such an electron and one with its spin flipped is then
\beq
	\delta E = - g \frac{e}{m}s_z B_z = -g \mu_B B
\eeq


The general definition of the $g$-factor then follows -- the energy difference between two particles with spin projection parallel and anti-parallel to a small constant magnetic field will be proportional to $ - \mu_B B$, and the coefficient of this difference defines the $g$-factor.

For the electron the ``natural'' value is $g=2$.  This is the value obtained from the Dirac equation.  There will be radiative corrections to this that come from quantum field theory, the value of such being known very precisely.  This can be treated as an expansion in $\alpha$.

Of interest here is the $g$ factor of a bound state system, which will differ from that of a free particle.  For a hydrogen-like system, there will be several types of corrections to the free $g$-factors of the constituents, corresponding to different scales of the atom.  There are binding corrections, coming from the nonrelativistic expansion over $v \sim Z\alpha$.  There are recoil corrections which related to the mass ratio of the two particles.  And there will be effects related to the finite size of the nucleus.

In this work the binding and recoil corrections to $g$ will be investigated.  These corrections will be proportional to the free $g$ factor, including the radiative corrections.  Because the binding corrections have a relativistic origin, it is necessary to work starting with a relativistic theory.

However, relativistic field theories are a poor choice for dealing with a nonrelativistic system like a low $Z$ hydrogen-like atom.  Because of the proliferation of energy scales, it is difficult to determine what diagrams contribute at a particular order.  The straightest path is to use the relativistic theory to develop an effective nonrelativistic theory of the electromagnetic interaction: nonrelativistic quantum electrodynamics (NRQED).  This may be done entirely by considering scattering processes where the bound state complications do not occur.  Then, the nonrelativistic theory may be used to analyse the bound system.


%FIXME refer properly to exp.
This approach can in principle be used to study arbitrarily high order corrections to the free $g$ factor, if the relativistic theory is known.  However, there exist specific experimental cases where the constituents are not fundamental particles.  Atoms such as hydrogenic $^{12}C$ or $^{16}O$ have a composite nucleus.  While there are corrections from this to the bound $g$ factor, and there are binding corrections of relativistic origin, at the order of interest corrections containing \emph{both} scales will be too small.  So for calculating the binding corrections, the nucleus may be treated as a point particle.  The internal structure is reflected in the form factors and total spin of the particle.


