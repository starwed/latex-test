
\chapter{Exact equations of motion for spin-1}

\beq
\mathcal{L} 
	=	-\frac{1}{2} (D^\mu W^\nu - D^\nu W^\mu)^\dagger (D_\mu W_\nu - D_\nu W_\mu)
		+ m^2 W^{\mu \dagger} W_\mu - i \lambda e  W^{\mu \dagger} W^\nu F_{\mu\nu}
\eeq
where as usual $D$ is the long derivative $D^\mu = \partial^\mu + ieA^\mu$.

Obtain the equations of motion from the Euler-Lagrange equations:
\[
	\pd{\mathcal{L}}{W^{\dagger \alpha}} - \partial_\mu \pd{ \mathcal{L} }{ [\partial_\mu W^{\dagger \alpha}] } = 0
\]
Or equivalently, %TODO show this explicitly?
\[
	\pd{\mathcal{L}}{W^{\dagger \alpha}} - D_\mu \pd{ \mathcal{L} }{ [D_\mu W^{\dagger \alpha}] } = 0
\]

\beqa
\pd{\mathcal{L}}{W^{\dagger \alpha}} 
	&=&	\pd{}{W^{\dagger \alpha} } \left( m^2 W^{\mu \dagger} W_\mu - i \lambda e  W^{\mu \dagger} W^\nu F_{\mu\nu} \right )	\\
	&=&	m^2 W_\alpha - i e \lambda W^\nu F_{\alpha \nu} 	\\
\eeqa


\beqa
\pd{ \mathcal{L} }{ [D_\gamma W^{\dagger \alpha}] }
	&=& -\frac{1}{2} \pd{}{ [D_\gamma W^{\dagger \alpha}] }	 (D^\mu W^\nu - D^\nu W^\mu)^\dagger (D_\mu W_\nu - D_\nu W_\mu) 	\\
	&=& -\frac{1}{2} ( g^{\mu \gamma} g^\nu_\alpha - g^{\nu\gamma} g^\mu_\alpha)(D_\mu W_\nu - D_\nu W_\mu)	\\
	&=&  D_\alpha W^\gamma - D^\gamma W_\alpha 
\eeqa


So the complete equation from Euler-Lagrange is
\beq \label{eq:ELeq}
		m^2 W_\alpha - i e \lambda W^\nu F_{\alpha \nu} + D_\mu (D^\mu W_\alpha - D_\alpha W^\mu) = 0
\eeq




This is a set of four coupled second order equations for the field $W$.  We rewrite as a set of first order equations by introducing a field $W_{\mu\nu} = D_\mu W_\nu - D_\nu W_\mu$.  So \eqref{eq:ELeq} becomes 
\beq
	m^2 W_\alpha - i e \lambda W^\mu F_{\alpha \mu} + D^\mu W_{\mu\alpha} = 0
\eeq

%TODO add note about two three-vectors to form a bispinor with?
$W^{\mu\nu}$ is antisymmetric and so has six degrees of freedom, corresponding to six independent fields.  Together with $W^\mu$ this represents a total of ten fields.  However, upon examination only some of these fields are dynamic.  The  fields $W^{0i}$ and $W^{i}$ appear in the equations with time derivatives, while the fields $W^{ij}$ and $W^0$ never do.  So it is only necessary to consider the former six fields.  So that these six fields all have the same dimension, we will define $\frac{W^{i0}}{m} = i \eta^i$.

We will now eliminate the extraneous fields and solve for $iD_0 W^i$, $iD_0 \eta^i$.

%%%%%%%%%%%%%%%%%%%%%%%%%%%%%%%%%%%%%%%%%%%%%%

\beq 
	m^2 W_\alpha - ie \lambda W^\nu F_{\alpha \nu} + D_\mu (D^\mu W_\alpha - D_\alpha W^\mu) = 0 
\eeq

Define $W_{\mu\nu} = D_\mu W_\nu - D_\nu W_\mu$.  
Then
\beq \label{eq:s1:EL1st}
	m^2 W_\alpha - ie \lambda W^\nu F_{\alpha \nu} + D^\mu W_{\mu \alpha} = 0
\eeq
To get the exact Hamiltonian of a bispinor, we can eliminate nondynamic fields.  

First consider \eqref{eq:s1:EL1st} with $\alpha=0$.
\beq
	m^2 W_0 - i e \lambda W^\nu F_{0\nu} + D^\mu W_{\mu0}
\eeq
\beq
	m^2 W_0 - i e \lambda W^j F_{0j} + D^j W_{j0}
\eeq

Solve this for $W_0$
\beq \label{eq:s1:w0}
	W_0 = \frac{1}{m^2} \left( i e \lambda W^j F_{0j} - D^j W_{j0} \right )
\eeq

Now, consider \eqref{eq:s1:EL1st} with $\alpha=i$
\beq
	m^2 W_i - ie \lambda W^\mu F_{i \mu} + D^\mu W_{\mu i} = 0
\eeq

\beq
	m^2 W_i - ie \lambda W^0 F_{i 0} + D^0 W_{0 i} - ie \lambda W^j F_{i j} + D^j W_{j i}= 0
\eeq

Using \eqref{eq:s1:w0} we can replace $W_0$

\beq
	m^2 W_i - \frac{i e \lambda}{m^2} \left( i e \lambda W^\nu F_{0\nu} - D^j W_{j0} \right ) F_{i 0} + D^0 W_{0 i} - ie \lambda W^j F_{i j} + D^j W_{j i}= 0
\eeq
Using $W_{ji} = D_j W_i - D_i W_j$
\beq
	m^2 W_i - \frac{i e \lambda}{m^2} \left( i e \lambda W^\nu F_{0\nu} - D^j W_{j0} \right ) F_{i 0} + D^0 W_{0 i} 
	- ie \lambda W^j F_{i j} + D^j (  D_j W_i - D_i W_j ) = 0
\eeq



Solve this for $D^0 W_{0i}$:

\beq \label{eq:s1:w0i}
	D^0 W_{0i} = - m^2 W_i  + \frac{i e \lambda}{m^2} \left( i e \lambda W^\nu F_{0\nu} - D^j W_{j0} \right ) F_{i 0} 
	+ ie \lambda W^j F_{i j} - D^j (  D_j W_i - D_i W_j )
\eeq


To get a similar equation for $W_i$, consider
\beq
	W_{i0} = D_i W_0 - D_0 W_i
\eeq
Then
\beq
	D^0 W_i = D_i W_0 - W_{i0}
\eeq
\beq \label{eq:s1:wi}
	D^0 W_i = D_i \frac{1}{m^2}\left( i e \lambda W^j F_{0j} - D^j W_{j0} \right )  - W_{i0}
\eeq

We now have equations that tell us the time evolution of a total of six fields: $W_i$ and $W_{i0}$.  We want to treat these as the components of some sort of bispinor.  To that end, first define $\eta_i = -i/m W_{i0}$ so that we have a pair of fields with the same mass dimension and hermiticity.  (Since $W_{i0} = D_i W_0 - D_0 W_i$ would pick up another minus sign under complex conjugation compared to $W_i$.)  Going the other way $W_{i0} = im \eta_i$.

Since eventually we want a nonrelativistic expression, we should write the spatial components of four vectors as three vectors.  To that end also write the components of the tensor $F_{\mu\nu}$ in terms of three vectors.
$$ F_{0i} = E^i , \; F_{ij} = -\epsilon_{ijk} B^k $$



Regular spatial vectors are ``naturally raised" while $D_i$ is ``naturally lowered".  Then we can rewrite \eqref{eq:s1:w0i}.
%TODO define D_i and \v{D}
\beq
	-im D_0 \eta_i =  - m^2 W_i  - \frac{i e \lambda}{m^2} \left( i e \lambda W^j E^j - D^j W_{j0} \right ) E^i
	+ ie \lambda W^j \epsilon_{ijk} B^k - D^j (  D_j W_i - D_i W_j )
\eeq

\beq
	i D_0 \eta^i =  m W^i  + \frac{i e \lambda}{m^3} \left( i e \lambda W^j E^j - D^j W_{j0} \right ) E^i
	- \frac{ie \lambda}{m} W^j \epsilon_{ijk} B^k + \frac{1}{m}D_j (  D_j W^i - D_i W^j )
\eeq

And likewise \eqref{eq:s1:wi} becomes
\beq
		D^0 W_i = D_i \frac{1}{m^2}\left( -i e \lambda W^j E^j - im  D^j \eta_j \right )  - im \eta_i
\eeq
\beq
		i D^0 W^i = - i D_i \frac{1}{m^2}\left( -i e \lambda W^j E^j - i m  D_j \eta^j \right )  + m \eta^i
\eeq

\beq
	i D^0 W^i= -\frac{1}{m^2} D_i  e \lambda \v{W} \cdot \v{E} - \frac{1}{m} D_i \v{D} \cdot \v{\eta} + m \eta^i
\eeq

\subsubsection{Spin identities}
To obtain some Schrodinger like equation for a bispinor, we need to express the time derivative of the bispinor in terms of other operators which can be interpreted as the Hamiltonian: $i \partial_0 \Psi = \hat{H} \Psi$.  We have expressions for the time derivatives of $W_i$ and $\eta_i$, but some mixing of the indices is involved.  To write the Hamiltonian in a block form we can introduce spin matrices whose action will mix the components of the fields.
	

The spin matrix for a spin one particle can represented as:
\beq
	(S^k)_{ij} = - i\epsilon_{ijk}
\eeq
which leads to the following identities:
\beq
	(\v{a} \times \v{v})_i = -i (\v{S} \cdot \v{a})_{ij} v_j
\eeq
\beq
	\{ \v{a} \times (\v{a} \times \v{v}) \}_i = - ( \{\v{S} \cdot \v{a}\}^2)_{ij} v_j
\eeq

\beq
	a_i (\v{b} \cdot \v{v}) = \{ \v{a} \cdot \v{b} \; \delta_{ij} - (S^k S^\ell)_{ij} a^\ell b^k \} v_j
\eeq

Using these identities

\beq
	i D^0 \v{W}= -\frac{e \lambda}{m^2} \v{D}   (\v{W} \cdot \v{E}) + \frac{1}{m} \v{D} (\v{D} \cdot \gv{\eta}) + m \gv{\eta}
\eeq

\beq
	i D^0 \v{W} = 
		-\frac{1}{m^2} \{ \v{D} \cdot \v{E} - (\v{S} \cdot \v{E}) (\v{S} \cdot \v{D})	\} \v{W}
		+ \frac{1}{m}\{ \v{D}^2 -  (\v{S} \cdot \v{D})^2 \} \gv{\eta} + m \gv{\eta}
\eeq
and for the other equation
\beq
	i D_0 \eta^i =  m W^i  - \frac{i e \lambda}{m^3} \left( i e \lambda W^j E^j - D^j W_{j0} \right ) E^i
	+ \frac{ie \lambda}{m} W^j \epsilon_{ijk} B^k - \frac{1}{m}D_j (  D_j W^i - D_i W^j )
\eeq

\beq
	i D_0 \eta^i =  m W^i  - \frac{i e \lambda}{m^3} \left( i e \lambda W^j E^j - D^j W_{j0} \right ) E^i
	+ \frac{ie \lambda}{m} W^j \epsilon_{ijk} B^k - \frac{1}{m}D_j (  D_j W^i - D_i W^j )
\eeq

\beq
	i D_0 \gv{\eta} = m \v{W} 
			+ \frac{e^2 \lambda^2}{m^3} \v{E} (\v{W} \cdot \v{E} )
			- \frac{1}{m^2} \v{E} (\v{D} \cdot \gv{\eta})
			+ \frac{i e \lambda}{m} (\v{W} \times \v{B} )
			+\frac{1}{m} \v{D} \times (\v{D} \times \v{W} ) 
\eeq

%TODO check order of D and E -- it matters!
\beq
	i D_0 \gv{\eta} = m \v{W} 
			+ \frac{e^2 \lambda^2}{m^3} \{ \v{E}^2 - (\v{S} \cdot \v{E})^2 \} \v{W}
			-\frac{1}{m^2} \{ \v{E} \cdot \v{D} - (\v{S} \cdot \v{E}) (\v{S} \cdot \v{D}) \} \gv{\eta}
			+ \frac{ e \lambda}{m} (\v{S} \cdot \v{B} ) \v{W}
			 - \frac{1}{m} (\v{S} \cdot \v{D})^2  \v{W}  
\eeq

\beq
	i D_0 \gv{\eta} = \left(
				m 
				+ \frac{e^2 \lambda^2}{m^3} \{ \v{E}^2 - (\v{S} \cdot \v{E})^2 \} 
				- \frac{ e \lambda}{m} (\v{S} \cdot \v{B} )
				- \frac{1}{m} (\v{S} \cdot \v{D})^2 
			\right ) \v{W} 
			+\frac{1}{m^2} \{ \v{E} \cdot \v{D} - (\v{S} \cdot \v{E}) (\v{S} \cdot \v{D}) \} \gv{\eta}  
\eeq

Now that the equations are in the correct form, we can write them as follows:



%%TODO This is the correct form, but go back and make sure equations leading up to this don't have sign errors
%% Also, add the spin ids that produce the cross products.
\beq
i D_0 \spinor{W}{\eta} = 
	\Mblock
	{ \lambda \frac{e}{m^2} \left [ \v{E} \cdot \v{D} - (\v{S} \cdot \v{E})( \v{S} \cdot \v{D}) + i \v{S} \cdot \v{E} \times \v{D} \right ] }
		{m
	-\frac{1}{m} (\v{S} \cdot \v{D})^2 
	- \lambda \frac{e}{m} \v{S} \cdot \v{B}
	+ \lambda^2 \frac{e^2}{m^3} \left [ \v{E}^2 - (\v{S} \cdot \v{E})^2 \right ]  }
	{ m
	- \frac{1}{m}\left [ \v{D}^2 - (\v{S} \cdot \v{D})^2 + e \v{S} \cdot \v{B} \right ]  }
	{ - \lambda \frac{e}{m^2} \left [ \v{D} \cdot \v{E} - (\v{S} \cdot \v{D}) (\v{S} \cdot \v{E}) + i \v{S} \cdot \v{D} \times \v{E} \right ]  }\eeq




\subsubsection{Current}

We can also derive the conserved current from the Lagrangian:
\begin{eqnarray*}
\mathcal{L} 
	&=&	-\frac{1}{2} (D \times W)^\dagger \cdot (D \times W) 
				+ m_w^2 W^\dagger W 
				-\lambda i e {W^\dagger}^\mu W^\nu F_{\mu \nu}	\\
\end{eqnarray*}

Where
\begin{eqnarray*}
		D^\mu	=	\partial^\mu - i e A^\mu ,
	&&
		D \times W = D^\mu W^\nu - D^\nu W^\mu
\end{eqnarray*}


We want the conserved current corresponding to the transformation $ W_i \to e^{i \alpha}W_i $, which in infinitesimal form is:
\begin{equation*}
	W_\mu \to W_\mu + i \alpha W_\mu, \;
	W_\mu^\dagger \to W_\mu^\dagger - i \alpha W_\mu^{\dagger}
\end{equation*}

The 4-current density will be:
\begin{equation*}
j^{\sigma} = -i \pd {\mathcal{L} }{W_{\mu, \sigma}} W_\mu  +  i\pd {\mathcal{L} }{W^\dagger_{\mu, \sigma}} W^\dagger_\mu
\end{equation*}

Only one term contains derivatives of the field:

\begin{eqnarray*}
 \pd {\mathcal{L} }{W_{\alpha, \sigma}} 
		&=& \pd{}{W_{\alpha, \sigma}} \left \{ - \frac{1}{2} (D_\mu W_\nu - D_\nu W_\mu)^\dagger(D^\mu W^\nu - D^\nu W^\mu) \right \}\\
		&=& - \frac{1}{2} (D_\mu W_\nu - D_\nu W_\mu)^\dagger (g_{\sigma \mu} g_{\alpha \nu} - g_{\sigma \nu}g_{\alpha \mu})\\
		&=& - (D_\alpha W_\sigma - D_\sigma W_\alpha)^\dagger
\end{eqnarray*}
Likewise:
\begin{eqnarray*}
	\pd {\mathcal{L} }{W_{\alpha, \sigma}^\dagger} 
		&=& - (D_\alpha W_\sigma - D_\sigma W_\alpha)
\end{eqnarray*}

If  we define $W_{\mu \nu} =  D^\mu W^\nu - D^\nu W^\mu$ then the 4-current and charge density are:

\begin{eqnarray*}
	j_\sigma &=& i W_{\sigma \mu}^\dagger W^{\mu} - i W_{\sigma \mu} {W^{\dagger}}^\mu \\
	j_0 	&=& i W_{0 \mu}^\dagger W^{\mu} - i W_{0 \mu} {W^{\dagger}}^\mu \\
		&=& i W_{0 i}^\dagger W^i - i W_{0 i} {W^{\dagger}}^i \\
\end{eqnarray*}
Where the last equality follows from the antisymmetry of $W_{\mu \nu}$.

Now, we defined the fields $\eta_i = -i \frac{W_{i0}}{m}$.  In terms of these fields,
$j_0 =  m (\eta_i^\dagger  W^i + \eta_i {W^\dagger}^i )$. 

We can do the same to find the vector part of the current.
\begin{eqnarray*}
	j_i &=& i W_{i \mu}^\dagger W^{\mu} - i W_{i \mu} {W^{\dagger}}^\mu 	\\
	&=&	i W_j^\dagger W_{ij}  + i W_{i0}^\dagger W_0 + c.c.
\end{eqnarray*}

We have $W_{ij} = D_i W_j - D_j W_i$.  Using the identities developed in the appendix, we can obtain
\[ D_j W_i = D_i W_j - D_k(S_i S_k)_{ja} W_a 	\]
Then
\[ W_{ij} = D_k (S_i S_k)_{ja} W_a 	\]

In the absence of an electric field E, $W_{0} = \frac{i}{m} D_j \eta_j$, with $W_{i0} = i m \eta_i$.
\[	W_{i0}^\dagger W_0 = - \eta_i^\dagger D_j \eta_j 	\]
Again we introduce spin matrices to get the equation in the desired form, and obtain
\[	W_{i0}^\dagger W_0 = - \eta_j^\dagger D_k (\delta_{ik} - S_k S_i) \eta_j 	\]

This leads to
\[ j_i = i W_j^\dagger D_k (S_i S_k W)_j - i \eta_j^\dagger D_k ([\delta_{ik} - S_k S_i]\eta)_j + c.c. \]

Writing this in terms of the bispinor $\begin{pmatrix}\eta \\ W\end{pmatrix}$, the expression for the current is

\begin{equation}	j_i	=
		\frac{i}{2} \begin{pmatrix}\eta^\dagger && W^\dagger \end{pmatrix} \left [
		(\{S_i, S_j\} - \delta_{ij})  
		\begin{pmatrix} 
			1 & 0 \\
			0 & 1 \\ 
		\end{pmatrix}
		- ([S_i, S_j] +\delta_{ij})	\begin{pmatrix} 1 & 0 \\ 0 & -1 \\ \end{pmatrix}
		\right ]
		D_j \begin{pmatrix}\eta \\ W\end{pmatrix} + c.c.
\end{equation}



\subsubsection{Hermiticity of Hamiltonian}


It's clear from inspection that the above Hamiltonian is not Hermitian in the standard sense.  (Of the operators in use, the only one which is not self adjoint is $D_i^\dagger = - D_i$.)  Noticing that 
\[
	\left [ \v{E} \cdot \v{D} - (\v{S} \cdot \v{E})( \v{S} \cdot \v{D}) + i \v{S} \cdot \v{E} \times \v{D} \right]^\dagger 
		= - \left [ \v{D} \cdot \v{E} - (\v{S} \cdot \v{D}) (\v{S} \cdot \v{E}) + i \v{S} \cdot \v{D} \times \v{E} \right ] 
\]
we can see it has the general form 
\[
	H = 
\begin{pmatrix}
	A	&	B	\\
	C	&	A^\dagger
\end{pmatrix}
\]
where the off diagonal blocks are Hermitian in the normal sense: $B^\dagger = B$ and $C^\dagger=C$.

At this point we should consider that an operator is defined as Hermitian with respect to a particular inner product.  In quantum mechanics this inner product is normally defined as:
\[	\langle \Psi, \phi \rangle = \int d^3x \Psi^\dagger \phi	\]
This definition, however, does not produce sensible results for the states in question.  We need the inner product of a state with itself to be conserved; in other words, it should play the role of a conserved charge.  From the considerations above we already have one such quantity: the conserved charge $\int d^3x j_0$, where 

\[	
	j_0 = m [\eta^\dagger W + W^\dagger \eta] 
		= 	m \begin{pmatrix} \eta^\dagger & W^\dagger \end{pmatrix}
			\begin{pmatrix} 0 & 1 \\ 1 & 0 \end{pmatrix}
			\begin{pmatrix} \eta \\ W \end{pmatrix}
\]

A more general definition of the inner product includes some weight M: 
\[	\langle \Psi, \phi \rangle = \int d^3x \Psi^\dagger M \phi	\]
In normal quantum mechanics M would be the identity matrix, but here, as implied by the charge density, we want $M=\begin{pmatrix} 0 & 1 \\ 1 & 0 \end{pmatrix}$.  Such a definition will lead to the inner product $<\Psi, \Psi>$ being conserved.

An operator H is hermitian with respect to this inner product if
\[ \langle H\Psi, \phi \rangle = \langle \Psi, H\phi \rangle \to
		\int d^3x \Psi^\dagger H^\dagger M \phi	=	\int d^3x \Psi^\dagger M H \phi
\]
For this equality to hold, it is sufficient for $H^\dagger M = M H$.  With $M=\begin{pmatrix} 0 & 1 \\ 1 & 0 \end{pmatrix}$ and $H=\begin{pmatrix} A & B \\ C & D \end{pmatrix}$ this condition reduces to 
\[
	\begin{pmatrix} A^\dagger & C^\dagger \\ B^\dagger & D^\dagger \end{pmatrix}
	=\begin{pmatrix} D & C \\ B & A \end{pmatrix}
\]
Our Hamiltonian fulfills exactly this requirement, and so is Hermitian with respect to this particular inner product.



\subsection{Non-relativistic Hamiltonian}
Now we will consider the non-relativistic limit of the above Hamiltonian.  To work in this regime constrains the order of both the momentum and the electromagnetic field strength.
\begin{eqnarray*}
	D	&\sim&	mv	\\
	\Phi	&\sim&	mv^2	\\
	E	&\sim&	m^2v^3	\\
	B	&\sim&	m^2v^2
\end{eqnarray*}

We can write the Hamiltonian matrix in terms of the basis of 2x2 Hermitian matrices: $({\bf I}, \rho_i)$:

\begin{eqnarray*}
H &=&	a_0 {\bf I} + a_i \rho_i \\
a_0  	&=& 
 \frac{1}{2}(H_{11}+H_{22}) =
			e\Phi + 
				\lambda \frac{e}{2m^2} 
				\left [ 
					\v{E} \cdot \v{D} - (\v{S} \cdot \v{E})( \v{S} \cdot \v{D}) + i \v{S} \cdot \v{E} \times \v{D}
					- \v{D} \cdot \v{E} + (\v{S} \cdot \v{D}) (\v{S} \cdot \v{E}) - i \v{S} \cdot \v{D} \times \v{E} 
				\right ]\\
a_3 	&=&
 \frac{1}{2}(H_{11}-H_{22}) =
				\lambda s\frac{e}{2m^2} 	\left [ 
					\v{E} \cdot \v{D} - (\v{S} \cdot \v{E})( \v{S} \cdot \v{D}) + i \v{S} \cdot \v{E} \times \v{D}
					+ \v{D} \cdot \v{E} - (\v{S} \cdot \v{D}) (\v{S} \cdot \v{E}) + i \v{S} \cdot \v{D} \times \v{E} 
				\right ]\\
ia_2	&=& 
 \frac{1}{2}(H_{21}-H_{12}) =
				-\left [	
					\frac{\v{D}^2}{2m} - \frac{1}{m} (\v{S} \cdot \v{D})^2 
					-\frac{\lambda-1}{2} \frac{e}{m} \v{S} \cdot \v{B}
					+ \frac{e^2}{2m^3}(\v{E}^2 -(\v{S}\cdot\v{E})^2)
				\right ]	\\
a_1		&=&	
 \frac{1}{2}(H_{12}-H_{21}) =
				\left [
					m - \frac{\v{D}^2}{2m} - \frac{1+\lambda}{2}\frac{e}{m}\v{S} \cdot \v{B}
					+ \frac{e^2}{2m^3}(\v{E}^2 -(\v{S}\cdot\v{E})^2)
				\right ]
\end{eqnarray*}

We can see that to leading order, the Hamiltonian is
\begin{equation*}
H = m \rho_1 = 
\begin{pmatrix}
0	&	m \\
m	&	0	\\
\end{pmatrix}
\end{equation*}
Since we wish to separate positive and negative energy states, this poses a problem.  So we first switch to a basis where at least the rest energies are separate.  Then, remaining off-diagonal elements can be treated as perturbations.

An appropriate transformation which meets our requirements is a "rotation" in the space spanned by $\rho_i$ matrices, about the $\rho_1 + \rho_3$ axis.  This has the explicit form:
\begin{equation}
U =\frac{1}{\sqrt {2}}  (\rho_1 + \rho_3)	
	= \frac{1}{\sqrt {2}}
\begin{pmatrix}
1	&	1	\\
1	&	-1	\\
\end{pmatrix}
\end{equation}
Any transformation U can be described by it's action on the basis of 4x4 matrices.   This takes:
\begin{eqnarray*}
{\bf I} & \to &	{\bf I}	\\
\rho_1	& \to &	\rho_3	\\
\rho_3	& \to &	\rho_1	\\
\rho_2	& \to &	-\rho_2	\\
\begin{pmatrix}
	\eta	\\	W
\end{pmatrix}
&	\to	&
\begin{pmatrix}
\Psi_u \\	\Psi_\ell
\end{pmatrix}
=	\frac{1}{\sqrt{2}}
\begin{pmatrix}
\eta	+ W \\	\eta - W
\end{pmatrix}
\end{eqnarray*}

(This transformation will transform the current to
\[	j_0 =  m (\Psi_u^\dagger \Psi_u  - \Psi_\ell^\dagger \Psi_\ell)		\]
and the weight M, used in the inner product, to
\[	M \to M' = U^\dagger M U = \begin{pmatrix} 1 & 0 \\ 0 & -1 \end{pmatrix}	\]
The transformed Hamiltonian will of course be Hermitian with respect to the transformed inner product.)

Our equation is now of the following form:
\begin{eqnarray*}
i\partial_0 \begin{pmatrix} \Psi_u \\	\Psi_\ell \end{pmatrix}  
	& =&
	H' \begin{pmatrix} \Psi_u \\	\Psi_\ell  \end{pmatrix}
\end{eqnarray*}

We can see that the Hamiltonian still contains off-diagonal elements, so this represents a pair of coupled equations for the upper and lower components of $\gv{\Psi}$.  But the off-diagonal terms are small, so we can consider the case where $\Psi_\ell$ is small compared to $\Psi_u$. Solving for $\Psi_\ell$ in terms of $\Psi_u$: 

\begin{eqnarray*}
	E \Psi_\ell &=& 	H'_{21} \Psi_u + H'_{22} \Psi_\ell \\
	\Psi_\ell 	&=& 	(E - H'_{22})^{-1} H'_{21} \Psi_u
\end{eqnarray*}

This gives the exact formula:
\begin{eqnarray*}
	E \Psi_u 	&=&		\left( H'_{11} + H'_{12}[E-H'_{22}]^{-1} H'_{21} \right) \Psi_u
\end{eqnarray*}

However, we only need corrections to the magnetic moment of order $v^2$.  With $\frac{e}{m}\v{S} \cdot \v{B} \sim mv^2$, this means we only need the Hamiltonian to at most order $mv^4$.  Examining the leading order terms of the matrix H', the diagonal elements are order m while the off-diagonal elements are order $mv^2$.  To leading order the term $[E-H'_{22}]^{-1}=\frac{1}{2m}$. So we'll need $H'_{11}$ to $\mathcal{O}(v^4)$, and $H'_{12}$, $H'_{21}$, and $[E-H'_{22}]^{-1}$ each to only leading order.


\[	E \Psi_u 	=		\left( H'_{11} + \frac{1}{2m}H'_{12} H'_{21} + \mathcal{O}(mv^6)\right) \Psi_u \]

The needed terms of H are, using $\lambda=g-1$
\begin{eqnarray*}
	H'_{11} 	= a_0+ a_3
			&=& m + e\Phi - \frac{D^2}{2m} - \frac{g}{2}\frac{e}{m} \v{S} \cdot \v{B}	\\
			&& +(g-1)\frac{e}{2m^2} 
				\left [ 
					\v{E} \cdot \v{D} - (\v{S} \cdot \v{E})( \v{S} \cdot \v{D}) + i \v{S} \cdot \v{E} \times \v{D}
					- \v{D} \cdot \v{E} + (\v{S} \cdot \v{D}) (\v{S} \cdot \v{E}) - i \v{S} \cdot \v{D} \times \v{E} 
				\right ]
			+\mathcal{O}(mv^6)	\\
	H'_{12} 	= a_1 - ia_2
			&=& \frac{D^2}{2m} - \frac{1}{m}(\v{S} \cdot \v{D})^2 - \frac{g-2}{2}\frac{e}{m} \v{S} \cdot \v{B}
			+\mathcal{O}(mv^4)	\\
	H'_{21}  = a_1 + ia_2
			&=&  -\frac{D^2}{2m} + \frac{1}{m}(\v{S} \cdot \v{D})^2 + \frac{g-2}{2}\frac{e}{m} \v{S} \cdot \v{B}
			+\mathcal{O}(mv^4)
\end{eqnarray*}

The product $H'_{12}H'_{21}$ is calculated in the appendix.  To first order in the magnetic field strength it is:
\begin{eqnarray*}
\frac{1}{2m}H'_{12}H'_{21}
	&=&	-\frac{1}{2m^3}\left(  \frac {\v{D}^2} {2} -  (\v{S} \cdot \v{D})^2  -  \frac{g-2}{2}  e\v{S} \cdot \v{B} \right )^2	\\	
	&=& 	-\frac{1}{2m^3}\left( 
				\frac{ \gv{\pi}^4 } {4}  -  e \v{p}^2  \v{S} \cdot \v{B}   
				-\frac{g-2}{2}e (\v{S} \cdot \v{p}) (\v{B} \cdot \v{p})
			\right)
\end{eqnarray*}

So finally,replacing all $\v{D}$ with $\gv{\pi} \equiv  \v{p} - e \v{A}$, we have a direct expression for $\Psi_u$:
\begin{eqnarray*}
	E \Psi_u 
		&=&\left \{ m + e\Phi + \frac{\pi^2}{2m} - \frac{g}{2}\frac{e}{m} \v{S} \cdot \v{B}
			- \frac{\pi^4 } {8m^3}  
			+ \frac{ e \v{p}^2  (\v{S} \cdot \v{B}) }{2m^3}
			+ (g-2)\frac{e}{4m^3} (\v{S} \cdot \v{p}) (\v{B} \cdot \v{p})
				 \right .	\\
		&&	\left . 
			+ (g-1)\frac{ie}{2m^2}  \left [ 
					\v{E} \cdot \gv{\pi} - (\v{S} \cdot \v{E}) (\v{S} \cdot \gv{\pi}) + i \v{S} \cdot \v{E} \times \gv{\pi}
					- \gv{\pi} \cdot \v{E} + (\v{S} \cdot \gv{\pi}) (\v{S} \cdot \v{E}) - i \v{S} \cdot \gv{\pi} \times \v{E} 
			\right ]			
			\right \}\Psi_u
\end{eqnarray*}
The complicated expression in square brackets can be cleaned up a bit:

\begin{eqnarray*}
\v{E} \cdot \gv{\pi} - \gv{\pi} \cdot \v{E}
	&=&	[E_i, \pi_i]			\\
	&=&	[E_i, -i\partial_i]	\\
	&=&	i(\partial_i E_i)		\\
(\v{S} \cdot \v{E}) (\v{S} \cdot \gv{\pi}) - (\v{S} \cdot \gv{\pi}) (\v{S} \cdot \v{E})
	&=&	S_i S_j E_i \pi_j - S_i S_j \pi_i E_j						\\
	&=&	(S_i S_j) (E_i \pi_j - E_j \pi_i - [\pi_i, E_j])					\\
	&=&	[S_i, S_j](E_i \pi_j) - (S_i S_j)(-i \nabla_i E_j)				\\
	&=&	(i\epsilon_{ijk}S_k)E_j \pi_i -  (S_i S_j)(-i \nabla_i E_j)		\\
	&=&	i \v{S} \cdot \v{E} \times \gv{\pi} + i S_i S_j \nabla_i E_j)	\\
(\v{E} \times \gv{\pi} - \gv{\pi} \times \v{E})_k
	&=&	\epsilon_{ijk}(E_i \pi_j - \pi_i E_j)		\\
	&=&	\epsilon_{ijk}(E_i \pi_j + \pi_j E_i)		\\
	&=&	\epsilon_{ijk}(2 E_i \pi_j + [\pi_i, E_j])	\\
	&=&	2 \epsilon_{ijk} E_i \pi_j				\\
	&=&	2 (\v{E} \times \gv{\pi})_k
\end{eqnarray*}

Using these identities and collecting terms, and then writing everything in terms of $g$, $g-2$
\begin{eqnarray*}
	E \Psi_u 
		&=&\left\{ m + e\Phi + \frac{\pi^2}{2m} - \frac{\pi^4 } {8m^3}
			- \frac{e}{m} \v{S} \cdot \v{B} \left ( \frac{g}{2} - \frac{p^2}{2m^2} \right )
			+ (g-2)\frac{e}{4m^3} (\v{S} \cdot \v{p}) (\v{B} \cdot \v{p})	\right. \\
		&&	\left.
			- (g-1)\frac{e}{2m^2} 
				\left [ 
					\v{\nabla} \cdot \v{E} 
					- S_i S_j \nabla_i E_j +\v{S} \cdot \v{E} \times \gv{\pi}
				\right ]
			\right\}\Psi_u	\\
		&=& \left\{ m + e\Phi + \frac{\pi^2}{2m} - \frac{\pi^4 } {8m^3}
			- \frac{g}{2}\frac{e}{m} \v{S} \cdot \v{B} \left ( 1 - \frac{p^2}{2m^2} \right )
			- \frac{g-2}{2} \frac{e}{m} \frac{p^2}{2m^2} \v{S} \cdot \v{B} 
			+ (g-2)\frac{e}{4m^3} (\v{S} \cdot \v{p}) (\v{B} \cdot \v{p})	\right.	\\
		&&	\left.
			- \left ( \frac{g}{2} + \frac{g-2}{2} \right) \frac{e}{2m^2} 
				\left [ 
					\v{\nabla} \cdot \v{E} 
					- S_i S_j \nabla_i E_j +\v{S} \cdot \v{E} \times \gv{\pi}
				\right ]
			\right\}\Psi_u
\end{eqnarray*}
			
We have a Hamiltonian for the upper component of the bispinor, as desired.  But is it truly Schrodinger-like?  In the general case we would need to perform the Foldy-Wouthyusen transformation, to a representation where all the physics up to the desired order is contained in the single spinor equation.  But we can show that, at the order we are working at, the Hamiltonian above is correct.



