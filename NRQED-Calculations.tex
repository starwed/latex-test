\section{Scattering off external field in NRQED}
We now have established the general form of the Lagrangian for NRQED of arbitrary spin.  In order to fix the coefficients, it will be necessary to compare the predictions of NRQED with some other source, such as an exact relativistic theory.


\subsection{One-photon Lagrangian}
The simplest (and most important for our purposes) set of coefficients to fix are those involved in the process of the charged particle scattering off an external field.  We can write down those terms in the NRQED Lagrangian which have one power of the external field.  This set of terms will not, by themselves, be gauge invariant.  However, we had previously taken care to group terms in a gauge invariant way.  So first we must untangle from these gauge invariant terms (involving one or more powers of $\v{D}$), those which involve a single field.


First take the kinetic terms.  $\v{D} = \grad - ie\v{A}$, so
\beq
	\v{D}^2 = (\nabla_i - ieA_i)(\nabla_i - ieA_i)
\eeq
It is a mixture of terms with two, one, or no powers of the external field $A$.  If just the terms with one power of $\v{A}$ are included, what remains is
\beq
	-ie( A_i \nabla_i + A_i \nabla_i = -ie \{A_i, \nabla_i\}
\eeq
So from $\v{D}^2/ 2m$ we can write $-ie \{\nabla_i, A_i\}/2m$ in  $\mathcal{L}_A$.

The second kinetic term is $\v{D}^4$:
\beq
	\v{D}^4 = (\nabla_i - ieA_i)(\nabla_i - ieA_i) (\nabla_j - ieA_j)(\nabla_j - ieA_j)
\eeq
or
\beq
	\v{D}^4 = ( \grad^2 - ie \{A_i, \nabla_i\} - e^2 \v{A}^2)( \grad^2 - ie \{A_j, \nabla_j\} - e^2 \v{A}^2)
\eeq
Keeping again only the terms with a single power of $A$, what remains may be expressed as the double anti-commutator
\beq
	-ie \{ \grad^2, \{\nabla_i, A_i \} \} 
\eeq

There are then several terms involving one or more powers of $D$ combined with either $E$ or $B$.  Since only terms with a single power of the field are of interest, from all such $\v{D}$ only the part involving $\grad$ is kept.



So finally, writing all such terms from the original Lagrangian involved in scattering off an external field leaves:
\small
\beqa
\mathcal{L}_A &=& \fnrb (  -eA_0 - ie  \frac{ \{\nabla_i, A_i \} }{2m} -ie \frac{ \{\grad^2, \{\nabla_i, A_i \}  \} }{8m^3} 
		+ c_F e \frac{\v{S} \smalldot \v{B}} {2m}   	
		+ c_D \frac{ e(\v{\grad} \smalldot \v{E} - \v{E} \smalldot \v{\grad} ) }{8m^2}	
		+ c_Q \frac{e Q_{ij} (\nabla_i E_j - E_i \nabla_j) }{8m^2}	
	\\&&	+ c^{1}_S \frac{ ie \v{S} \smalldot ( \v{\grad} \times \v{E} - \v{E} \times \v{\grad} )}{8m^2}
		+ c_{W_1} \frac{ e [ \v{\grad}^2 (\v{S} \smalldot \v{B} ) + (\v{S} \smalldot \v{B} ) \v{\grad}^2] }{8m^3}	
		- c_{W_2} \frac{ e \nabla^i (\v{S} \smalldot \v{B} ) \nabla^i }{4m^3}
		+ c_{p'p} \frac{ e [ (\v{S} \smalldot \v{\grad}) (\v{B} \smalldot \v{\grad}) + (\v{B} \smalldot \v{\grad})(\v{S} \smalldot \v{\grad}) }{8m^3} \big )\fnr
\eeqa
\normalsize


\subsection{Calculation}



Now we have the Lagrangian that will account for all interactions with a single photon.  We want to calculate from this a particular process: scattering off an external field, with incoming momentum $\v{p}$, outgoing $\v{p'}$, and $\v{q} = \v{p'} - \v{p}$.  There is one diagram associated with each term above, but the total amplitude is just going to be the sum of all these one-photon vertices.  These of course can just be read off directly from the Lagrangian.

It is necessary to switch to the language of momentum space.  The recipe is this: replace the fields $\Psi$ with the spinors $\phi$, and any operator $\grad$ acting will become $i\v{p}$ if it acts on the right, $i\v{p'}$ if it is to the left.  Go through term by term.

The terms originating from $D^2$ are:
\beq
	- ie  \frac{ \{\nabla_i, A_i \} }{2m} 
\eeq
which in position space become
\beq
	e \frac{ \v{A} \cdot (\v{p} + \v{p'}) }{2m} 
\eeq

While the terms arising from $D^4$
\beq
-ie \frac{ \{\grad^2, \{\nabla_i, A_i \}  \} }{8m^3} 
\eeq
become
\beq
- e \frac{ \v{A} \cdot (\v{p} + \v{p'}) (\v{p'}^2 + \v{p}^2 ) }{8m^3} 
\eeq


We can simplify some expressions involving $\grad$ and $\v{E}$:
Because $Q_{ij}$ is symmetric:
\[
	 Q_{ij} (\nabla_i E_j - E_i \nabla_j ) = Q_{ij} [\nabla_i, E_j] = Q_{ij} (\partial_i E_j)
\]

And because $E_i = -\partial_i \Phi$
\[
	\v{\grad} \times \v{E} - \v{E} \times \v{\grad} =  - 2 \v{E} \times \v{\grad}
\]
And also use that
\[
\v{\grad} \smalldot \v{E} - \v{E} \smalldot \v{\grad} = (\partial_i E_i)
\]


Now we can write down the scattering amplitude for scattering off the external field, before we apply any assumptions about the particular process.
\beqa
	iM &=&
		ie\wnrb \Bigg( - A_0 +    \frac{ \v{A} \cdot (\v{p} + \v{p'}) }{2m} 
		- \frac{  \v{A} \cdot (\v{p} + \v{p'}) \v{p}^2 + \v{p'}^2 \v{A} \cdot (\v{p} + \v{p'}) }{8m^3} 
	\\&&	+ c_F  \frac{\v{S} \smalldot \v{B}} {2m}   	
		+ c_D \frac{ ( \partial_i E_i ) }{8m^2}	
		+ c_Q \frac{Q_{ij}  ( \partial_i E_j ) }{8m^2}	
		+ c^{1}_S \frac{  \v{E} \times \v{p} }{4m^2}
	\\&&	- c_{W_1}  \frac{  (\v{S} \smalldot \v{B} ) (\v{p}^2 + \v{p'}^2)  }{8m^3}
		+ c_{W_2} \frac{  (\v{S} \smalldot \v{B} ) (\v{p} \cdot \v{p'}) }{4m^3}
		-  c_{p'p} \frac{  (\v{S} \smalldot \v{p'}) (\v{B} \smalldot \v{p}) + (\v{B} \smalldot \v{p'}) (\v{S} \smalldot \v{p}) }{8m^3} \Bigg )\wnr
\eeqa

The above can be simplified somewhat.  We choose our gauge such that $\nabla_i A_i = 0$.  If we specify elastic scattering then kinematics dictate that $\v{p'}^2 = \v{p}^2$.   Finally, if we consider $\v{B}$ constant, the $c_W$ terms become indistinguishable, since $[ \nabla_i, B_j] = 0$.    (It is only this last assumption that costs us any information.)  Then the scattering amplitude, as calculated from $\mathcal{L}_{NRQED}$, is:

\beq 
\begin{split} \label{eq:nrqedScatter}
	iM =&
		ie\wnrb \Bigg(  -A_0 +  \frac{ \v{A} \cdot \v{p} }{m} - \frac{  (\v{A} \cdot \v{p}) \v{p}^2   }{2m^3} 
		+ c_F  \frac{\v{S} \smalldot \v{B}} {2m}   	
		+ c_D \frac{ ( \partial_i E_i ) }{8m^2}	
		+ c_Q \frac{ Q_{ij} ( \partial_i E_j ) }{8m^2}	
	\\&	+ c^{1}_S \frac{  \v{E} \times \v{p} }{4m^2}
		- (c_{W_1} -c_{W_2}) \frac{   (\v{S} \smalldot \v{B} ) \v{p}^2  }{4m^3}	
		-  c_{p'p} \frac{   (\v{S} \smalldot \v{p}) (\v{B} \smalldot \v{p})  }{4m^3} \Bigg )\wnr 
\end{split}
\eeq





%%%%%%%%%%%%%%%%%%%%%%%%%%%%%
% TWO PHOTON
%%%%%%%%%%%%%%%%%%%%%%%%%%%%%%

%Two photon lagrangian
\subsection{Two photon scattering in NRQED}



We want to calculate the Compton scattering in the nonrelativistic theory.  We choose the gauge such that the photon polarisations obey $\A_0 = 0$.  In general we should consider both terms arising from two-photon vertices, and those from tree level diagrams of two one-photon vertices.  However, because of the approach we take in calculating the process from the relativistic Lagrangian, we only need the former terms.

Further, ultimately we only care about a few of these terms, ignoring terms which have both $\v{B}$ and $\v{A}$.

Note that both $(\v{A} \smalldot \v{E} - \v{E} \smalldot \v{A} )=0$ and $Q_{ij} (A_i E_j - E_i A_j) = 0$ by symmetry.

The remaining terms we're interested in are:
\scriptsize
\beqa
	\mathcal{L}_{A^2} &=& \Psi^\dagger ( - \frac{e^2 \v{A}^2}{2m}  - e^2 \frac{ \{ \grad^2, \v{A}^2 \} 
}{8m^3} - e^2\frac{ \{\nabla_i, A_i \} \{\nabla_j, A_j\} }{8m^3}
		+ c^2_S \frac{ e^2 \v{S} \smalldot ( \v{A} \times \v{E} - \v{E} \times \v{A} )}{8m^2} ) \Psi
\eeqa
\normalsize

If we want the scattering amplitude, we replace $\Psi$ with $\phi$, and contract $\v{A}$ with photon polarisations $\A$ and $\Adag$.  In the gauge chosen, $\v{E}(k) = -\partial_0 \v{\A} = i k_0 \v{\A}(k)$.

Contracted with the photon of momentum $k$ we get $\v{E} \to i k_0 \v{\A}$, while with the photon of momentum $k'$ we have $\v{E} \to i k'_0 \v{\Adag}$.  So:
\[
	\v{A} \times \v{E} = - \v{A} \times (\partial_0 \v{A})
		\to 
	-i(k_0' \v{\A} \times \v{\Adag} + k_0 \v{\Adag} \times \v{\A} ) = i ( k_0' - k_0) \v{\A} \times \v{\Adag}
\]
And
\[
	\v{E} \times \v{A} = - (\partial_0 \v{A}) \times  \v{A}
		\to 
	-i( k_0 \v{\A} \times \v{\Adag} + k_0' \v{\Adag} \times \v{\A} ) = -i ( k_0' - k_0) \v{\A} \times \v{\Adag}
\]
So from the term in the Lagrangian 
\[
 c^2_S \Psi^\dagger \frac{ e^2 \v{S} \smalldot ( \v{A} \times \v{E} - \v{E} \times \v{A} )}{8m^2} ) \Psi
\]
we get in the scattering amplitude

\[
  -i c^2_S \phi^\dagger  \Big ( \frac{e^2}{4m^2}    i(k_0 - k_0')    \v{\A} \times \v{\Adag} \Big ) \phi
	=
     c^2_S \frac{e^2}{4m^2} \phi^\dagger  \Big ( (k_0 - k_0')    \v{\A} \times \v{\Adag} \Big ) \phi
\]

