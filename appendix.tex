

\section{Comparison with Silenko}

We can compare to a result in Silenko (arXiv:hep-th/0401183v1).  Silenko determines a Hamiltonian which is equivalent to that used above.  Dropping terms manifestly beyond $\mathcal{O}(mv^4)$:

\begin{eqnarray*}
H &=& \rho_3 \epsilon' + e\Phi +\frac{e}{4m} 
	\left [ \left \{ 	\left (\frac{g-2}{2} + \frac{m}{\epsilon' + m} \right )\frac{1}{\epsilon'}, 
				(\v{S} \cdot \gv{\pi} \times \gv{E} - \v{S} \cdot \v{E} \times \gv{\pi})
		\right \}_+
		- \rho_3 \left \{ \left ( g-2 + \frac{2m}{\epsilon'} \right ), \v{S} \cdot \v{B} \right \}_+
	\right.	\\
	&& \left.
		+\rho_3 \left \{ \frac{g-2}{2\epsilon' (\epsilon'+m)}, 
			\{\v{S} \cdot \gv{\pi}, \gv{\pi} \cdot \v{B} \}_+ \right \}_+
	\right ]
	+ \frac{e(g-1)}{4m^2}\left \{ \v{S} \cdot \gv{\nabla}, \v{S} \cdot \v{E} \right \}_+
	- \frac{e(g-1)}{2m^2} \v{\nabla} \cdot \v{E}
\end{eqnarray*}


We'll take each term and expand to $\mathcal{O}(mv^4)$. First the operator $\epsilon'$: 
\begin{eqnarray*}
\epsilon' 	
	&=& 	\sqrt{m^2 + \pi^2}						\\
	&=&	m + \frac{\pi^2}{2m} - \frac{\pi^4}{8m^2} +\mathcal{O}(mv^6)
\end{eqnarray*}

Using this, we get
\begin{eqnarray*}
H &=& \rho_3 \epsilon' + e\Phi +\frac{e}{4m} 
	\left [ \left \{ 	\frac{g-1}{2m}, 
				(\v{S} \cdot \gv{\pi} \times \gv{E} - \v{S} \cdot \v{E} \times \gv{\pi})
		\right \}_+
		- \rho_3 \left \{ g -\frac{\pi^2}{m^2} , \v{S} \cdot \v{B} \right \}_+
	\right.	\\
	&& \left.
		+\rho_3 \left \{ \frac{g-2}{4m^2}, 
			\{\v{S} \cdot \gv{\pi}, \gv{\pi} \cdot \v{B} \}_+ \right \}_+
	\right ]
	+ \frac{e(g-1)}{4m^2}\left \{ \v{S} \cdot \gv{\nabla}, \v{S} \cdot \v{E} \right \}_+
	- \frac{e(g-1)}{2m^2} \v{\nabla} \cdot \v{E}
\end{eqnarray*}

Now we'll expand the anticommutators.



We have already shown that $\gv{\pi} \times \v{E} = - \v{E} \times \gv{\pi}$, and that such a term is $\mathcal{O}(mv^4)$
\begin{eqnarray*}
\left \{ 	\frac{g-1}{2m}, 
				(\v{S} \cdot \gv{\pi} \times \gv{E} - \v{S} \cdot \v{E} \times \gv{\pi})
		\right \}_+
	&=&			-\frac{2(g-1)}{m} 	\v{S} \cdot \v{E} \times \gv{\pi}
\end{eqnarray*}

Because we specialize to constant magnetic fields, $\gv{\pi}$ commutes with $\v{B}$, so 
\begin{eqnarray*}
\left \{ g -\frac{\pi^2}{m^2} , \v{S} \cdot \v{B} \right \}_+
	&=&	(2g - \frac{2\pi^2}{m^2}) \v{S} \cdot \v{B}
\end{eqnarray*}

and

\begin{eqnarray*}
\left \{ \frac{g-2}{4m^2}, 
			\{\v{S} \cdot \gv{\pi}, \gv{\pi} \cdot \v{B} \}_+ \right \}_+
	&=&		\frac{g-2}{m^2} (\v{S} \cdot \gv{\pi}) (\gv{\pi} \cdot \v{B})
\end{eqnarray*} 

Using the identity that $[\nabla_i, E_j]=0$, $\v{\nabla} \times \v{E} = 0$:
 \begin{eqnarray*}
 \left \{ \v{S} \cdot \gv{\nabla}, \v{S} \cdot \v{E} \right \}_+
	&=& S_i S_j \nabla_i E_j + S_j S_i E_j \nabla_i	\\
	&=& (S_i S_j + S_j S_i) \nabla_i E_j			\\
	&=& (2S_i S_j + [S_j, S_i])\nabla_i E_j			\\
	&=& (2S_i S_j +iS_k\epsilon_{ijk})\nabla_i E_j	\\
	&=& 2S_i S_j\nabla_i E_j
 \end{eqnarray*}
 
So
\begin{eqnarray*}
H 	&=& \rho_3 (m + \frac{\pi^2}{2m} - \frac{\pi^4}{8m^2})  + e\Phi - (g-1)\frac{e}{2m^2} 
		\left [ 
			 \v{S} \cdot \v{E} \times \gv{\pi}
			-S_i S_j \nabla_i E_j 
			+\v{\nabla} \cdot \v{E}	
		\right ]
	\\&&
		+ \rho_3 (g-2)\frac{2}{4m^2} (\v{S} \cdot \gv{\pi}) (\gv{\pi} \cdot \v{B})
		- \rho_3 \frac{e}{m}(\frac{g}{2} - \frac{\pi^2}{2m^2}) \v{S} \cdot \v{B}	\\
	&=& \rho_3 \left(m + \frac{\pi^2}{2m} - \frac{\pi^4}{8m^2} \right)  + e\Phi - \left(\frac{g-2}{2} + \frac{g}{2}\right)\frac{e}{2m^2} 
		\left [ 
			 \v{S} \cdot \v{E} \times \gv{\pi}
			-S_i S_j \nabla_i E_j 
			+\v{\nabla} \cdot \v{E}	
		\right ]	
	\\&&
		+ \rho_3 (g-2)\frac{2}{4m^2} (\v{S} \cdot \gv{\pi}) (\gv{\pi} \cdot \v{B})
		- \rho_3 \frac{e}{m}\left [\frac{g}{2}\left(1 - \frac{\pi^2}{2m^2}\right) + \frac{g-2}{2}\frac{\pi^2}{2m^2}) \right] \v{S} \cdot \v{B}							
\end{eqnarray*}

\subsection*{Magnetic Moment}
Now we'll keep only those terms which contribute to the magnetic moment, using $\gv{\pi} = \v{p} - e\v{A}$
\begin{eqnarray*}
H_{S\cdot B}	&=&
		 \left(\frac{g-2}{2} + 	\frac{g}{2}\right)\frac{e^2}{2m^2} \v{S} \cdot \v{E} \times \v{A}
		+  (g-2)\frac{e}{4m^2} (\v{S} \cdot \v{p}) (\v{p} \cdot \v{B})
		-  \frac{e}{m}\left [\frac{g}{2}\left(1 - \frac{p^2}{2m^2}\right) + \frac{g-2}{2}\frac{p^2}{2m^2} \right] \v{S} 		\cdot \v{B}		\\
	&=&		-\frac{e}{2m} \left\{
			g \left(1 - \frac{p^2}{2m^2}\right)  \v{S} \cdot \v{B}	
			+(g-2) \frac{p^2}{2m^2} \v{S} \cdot \v{B}	
			-(g-2) \frac{ (\v{S} \cdot \v{p}) (\v{p} \cdot \v{B})}{2m^2}
			-\frac{e}{m} \left(\frac{g-2}{2} + 	\frac{g}{2}\right) \v{S} \cdot \v{E} \times \v{A}
		\right\}
\end{eqnarray*}
This exactly matches with what was found before.



\section{Identities}

Simplify $ \v{W} \times \v{B}$:
\begin{eqnarray*}
(\v{W} \times \v{B})_i
	&=&	\epsilon_{ijk} W_j B_k	\\
	&=&	i(S_k)_{ij} W_j B_k\\
	&=&	i(\v{S} \cdot \v{B})_{ij} W_j	\\
	&=&	i([\v{S} \cdot \v{B}] \v{W})_i
\end{eqnarray*}

Simplify $\v{D} \times ( \v{D} \times \v{W} ) $
\begin{eqnarray*}
(\v{D} \times [ \v{D} \times \v{W} ] )_i
	&=&	\epsilon_{ijk} D_j (\v{D} \times \v{W})_k	\\
	&=&	\epsilon_{ijk} \epsilon_{k\ell m} D_j D_\ell W_m	\\
	&=&	-(S_j)_{ki} (S_\ell)_{mk} D_j D_\ell W_m	\\
	&=&	-(\v{S} \cdot \v{D})_{ki} (\v{S} \cdot \v{D})_{mk} W_m	\\
	&=&	-\left( [\v{S} \cdot \v{D}]^2 \right)_{im} W_m	\\
	&=&	-\left( [\v{S} \cdot \v{D}]^2 \v{W} \right)_i	\\
\end{eqnarray*}

Our representation defines the spin matrices as follows:

$${(S_k)}_{ij}=-i \epsilon_{ijk}$$

They have the commutator
$$	[S_i, S_j] = i \epsilon_{ijk} S_k $$

The product of two such spin matrices is given by:
\begin{eqnarray*}
{(S_k S_\ell)}_{ij} 
	& = & {(S_k)}_{ia} {(S_\ell)}_{aj} \\
	& = & -\epsilon_{iak} \epsilon_{aj\ell} \\
	& = & (\delta_{k\ell} \delta_{ij} - \delta_{kj} \delta_{\ell i} )
\end{eqnarray*}

This implies that:

\begin{eqnarray*}
(S_i S_j A_j B_i \v{v} )_l 
	&=&  	{(S_i)}_{lm} {(S_j)}_{mn} A_j B_i v_n \\
	&=&	{(S_iS_j)}_{ln} A_j B_i v_n \\
	&=&	 (\delta_{ij} \delta_{ln} - \delta_{in} \delta_{lj} ) A_j B_i v_n \\
	&=&	(\v{A} \cdot \v{B}) v_l  - A_l ( \v{B} \cdot \v{v}) \\
\end{eqnarray*}

Or 

$$ \v{A} (\v{B} \cdot \v{v}) =  (\v{A} \cdot \v{B}   -  S_i S_j A_j B_i) \v{v} $$

Now we can use this to establish some identities:

\begin{eqnarray*}
 E^i \v{D} \cdot \v{\eta}
	&=&	\left( \left[\v{E} \cdot \v{D}  - S_j S_k E_k D_j\right] \v{\eta} \right)^i		\\
	&=&	\left( \left[\v{E} \cdot \v{D}  - (S_k S_j  - [S_k, S_j] ) E_k D_j \right] \v{\eta} \right)^i \\
	&=&	\left( \left[\v{E} \cdot \v{D}  - (S_k S_j - i \epsilon_{kjl} S_l ) E_k D_j \right] \v{\eta} \right)^i	  \\
	&=&	\left( \left[\v{E} \cdot \v{D}  - (\v{S} \cdot \v{E})( \v{S} \cdot \v{D}) +  i  S_l (\v{E} \times \v{D})_l \right] \v{\eta} \right)^i	  \\
	&=&	\left( \left[\v{E} \cdot \v{D}  - (\v{S} \cdot \v{E})( \v{S} \cdot \v{D}) +  i  \v{S} \cdot (\v{E} \times \v{D})\right] \v{\eta} \right)^i
\end{eqnarray*}

Since E and W commute:
\begin{eqnarray*}
 E^i \v{W} \cdot \v{E}
	&=& \left( \left[\v{E}^2  - S_j S_k E_k E_j\right] \v{W} \right)^i		\\
	&=& \left( \left[\v{E}^2  - (\v{S} \cdot \v{E} )^2 \right] \v{W} \right)^i		\\
\end{eqnarray*}

And
\begin{eqnarray*}
 D^i \v{D} \cdot \v{\eta}
	&=& \left( \left[\v{D}^2  - S_j S_k D_k D_j\right] \v{\eta} \right)^i		\\
	&=& \left( \left[\v{D}^2  - (S_k S_j + [S_j, S_k]) D_k D_j\right] \v{\eta} \right)^i		\\
	&=& \left( \left[\v{D}^2  - (\v{S} \cdot \v{D})^2 + i\epsilon_{jkl} S_l D_k D_j )\right] \v{\eta} \right)^i		\\
	&=& \left( \left[\v{D}^2  - (\v{S} \cdot \v{D})^2 + i\v{S} \cdot (\v{D} \times \v{D})  \right] \v{\eta} \right)^i		\\
	&=& \left( \left[\v{D}^2  - (\v{S} \cdot \v{D})^2 + e \v{S} \cdot \v{B})  \right] \v{\eta} \right)^i		\\
\end{eqnarray*}

Similarly:
\begin{eqnarray*}
 D^i \v{E} \cdot \v{W}
	&=& \left( \left[\v{D} \cdot \v{E}  - S_j S_k D_k E_j\right] \v{\eta} \right)^i		\\
	&=& \left( \left[\v{D} \cdot \v{E}  -  (S_k S_j + [S_j, S_k]) D_k E_j \right] \v{\eta} \right)^i		\\
	&=& \left( \left[\v{D} \cdot \v{E}  -  (S_k S_j +i\epsilon_{jkl} S_l) D_k E_j \right] \v{\eta} \right)^i		\\
	&=& \left( \left[\v{D} \cdot \v{E}  - (\v{S} \cdot \v{D})(\v{S} \cdot \v{E}) + i \v{S} \cdot (\v{D} \times \v{E})\right] \v{\eta} \right)^i		\\
\end{eqnarray*}	

\subsection*{Product of $H_{12}H_{21}$}
We need to calculate $\left(  \frac {\gv{\pi}^2} {2} -  (\v{S} \cdot \gv{\pi})^2 + (g-2)\frac{2}{m} \v{S} \cdot \v{B} \right )^2 $ to first order in magnetic field strength.  As a first step of simplification
\[
\left(  \frac {\gv{\pi}^2} {2} -  (\v{S} \cdot \gv{\pi})^2 + \frac{g-2}{2}\frac{e}{m} \v{S} \cdot \v{B} \right )^2 
	=	\left(  \frac {\gv{\pi}^2} {2} -  (\v{S} \cdot \gv{\pi})^2  \right )^2 
		 +\frac{g-2}{2}\frac{e}{m} \left \{ \frac{p^2}{2} - (\v{S} \cdot \v{p})^2, \v{S} \cdot \v{B} \right \}
\]	

\subsubsection*{First term}
To simplify the first term, consider one element of this matrix operator:
\begin{eqnarray*}
\left\{ \left(  \frac {\gv{\pi}^2} {2} -  (\v{S} \cdot \gv{\pi})^2   \right )^2  \right \} _{ac}
	&=& 	\left (  \frac {\gv{\pi}^2} {2} - S_i S_j \pi_i \pi_j \right)_{ab} 
				\left (  \frac {\gv{\pi}^2} {2} - S_l S_m \pi_l \pi_m \right)_{bc}\\
	&=& 	\left (  \frac {\gv{\pi}^2} {2}\delta_{ab} - [S_i S_j]_{ab} \pi_i \pi_j \right)
				\left (  \frac {\gv{\pi}^2} {2}\delta_{bc} - [S_l S_m]_{bc} \pi_l \pi_m \right)\\
	&=& 	\left (  \frac {\gv{\pi}^2} {2}\delta_{ab} - [\delta_{ab}\delta_{ij} - \delta_{aj}\delta_{bi}] \pi_i \pi_j \right)
				\left (  \frac {\gv{\pi}^2} {2}\delta_{bc} -  [\delta_{bc}\delta_{lm} - \delta_{bm}\delta_{cl}] \pi_l \pi_m \right)\\
	&=& 	\left (  -\frac {\gv{\pi}^2} {2}\delta_{ab} + \pi_b \pi_a \right)
				\left (  -\frac {\gv{\pi}^2} {2}\delta_{bc} + \pi_c \pi_b \right) \\
	&=&  	\frac{ \gv{\pi}^4 } {4} \delta_{ac}
				- \pi_c \pi_a \frac{\gv{\pi}^2}{2}
				- \frac{\gv{\pi}^2}{2} \pi_c \pi_a
				+ \pi_b \pi_a \pi_c \pi_b \\
\end{eqnarray*}

It's very useful to have the following identity:
\begin{eqnarray*}
	e(\v{S} \cdot \v{B})_{ab}
	&=&	e(S_i)_{ab} B_i	\\
	&=&	-i e \epsilon_{iab} B_i 	\\
	&=&	-i e \epsilon_{iab} (\epsilon_{ijk}\partial_j A_k)	\\
	&=&	-i e \epsilon_{iab} \epsilon_{ijk} \frac{1}{2} (\partial_j A_k \partial_k A_j) \\
	&=&	- \epsilon_{iab} \epsilon_{ijk} \frac{1}{2} [\pi_j, \pi_k]	\\
	&=&	-\epsilon_{iab} \epsilon_{ijk} \pi_j \pi_k	\\
	&=&	-(\delta_{aj} \delta_{bk}	- \delta_{ak} \delta_{bj} ) \pi_j \pi_k \\
	&=&	\pi_b \pi_a - \pi_a \pi_b	\\
\end{eqnarray*}

Therefore,
	$$ 	e(\v{S} \cdot \v{B})_{ab} = [\pi_b, \pi_a] $$



Using this, and the fact that $\pi$ commutes with S and B:
\begin{eqnarray*}
\pi_b \pi_a \pi_c \pi_b 
	&=&	\pi_b \pi_c \pi_a \pi_b 
				- \pi_b (e \v{S} \cdot \v{B})_{ac} \pi_b \\
	&=&	\pi_b \pi_c \pi_a \pi_b 
				-\gv{\pi}^2 (e \v{S} \cdot \v{B})_{ac}	\\
\end{eqnarray*}

Also,

\begin{eqnarray*}
\pi_b \pi_c \pi_a \pi_b 
	&=&	\pi_b \pi_c \pi_b \pi_a 
				+ \pi_b \pi_c (e \v{S} \cdot \v{B})_{ba}	\\
	&=&	\pi_b \pi_b \pi_c \pi_a 
				+ \pi_b \pi_a (e \v{S} \cdot \v{B})_{bc}
				+ \pi_b \pi_c (e \v{S} \cdot \v{B})_{ba}	\\
\end{eqnarray*}

So now:
\begin{eqnarray*}
\pi_b \pi_a \pi_c \pi_b  
	- \pi_c \pi_a \frac{\gv{\pi}^2}{2} 
	- \frac{\gv{\pi}^2}{4} \pi_c \pi_a
	&=&	\gv{\pi}^2 \pi_c \pi_a 
				+ \pi_b \pi_a (e \v{S} \cdot \v{B})_{bc}
				+ \pi_b \pi_c (e \v{S} \cdot \v{B})_{ba} \\
	&&			-\gv{\pi}^2 (e \v{S} \cdot \v{B})_{ac}
				- \pi_c \pi_a \frac{\gv{\pi}^2}{2} 
				- \frac{\gv{\pi}^2}{4} \pi_c \pi_a	\\
	&=&	\frac{1}{2} [\gv{\pi}^2, \pi_c \pi_a]
				+ \pi_b \pi_a (e \v{S} \cdot \v{B})_{bc}
				+ \pi_b \pi_c (e \v{S} \cdot \v{B})_{ba}
				-\gv{\pi}^2 (e \v{S} \cdot \v{B})_{ac} \\
\end{eqnarray*}

Now evaluate the commutator:
\begin{eqnarray*}
[\pi_b, \pi_c \pi_a]
	&=&	[\pi_b, \pi_c]\pi_a - \pi_c [\pi_a, \pi_b]	\\
	&=&	(e \v{S} \cdot \v{B})_{cb} \pi_a
				- \pi_c (e \v{S} \cdot \v{B})_{ba}	\\
\end{eqnarray*}

\begin{eqnarray*}
[ \gv{\pi}^2, \pi_c \pi_a]
	&=&	[\pi_b \pi_b, \pi_c \pi_a]	\\
	&=&	\pi_b [\pi_b, \pi_c \pi_a] + [\pi_b, \pi_c \pi_a] \pi_b	\\
	&=&	(e \v{S} \cdot \v{B})_{cb} ( \pi_b \pi_a + \pi_a \pi_b)
				-  (e \v{S} \cdot \v{B})_{ba} (\pi_b \pi_c + \pi_c \pi_b)	\\
\end{eqnarray*}


This gives the result:
\begin{eqnarray*}
\pi_b \pi_a \pi_c \pi_b  
	- \pi_c \pi_a \frac{\gv{\pi}^2}{2} 
	- \frac{\gv{\pi}^2}{4} \pi_c \pi_a
	&=&	\frac{1}{2} \left [ (e \v{S} \cdot \v{B})_{cb} ( \pi_b \pi_a + \pi_a \pi_b)
				-  (e \v{S} \cdot \v{B})_{ba} (\pi_b \pi_c + \pi_c \pi_b)  \right  ] \\
	&&		+ \pi_b \pi_a (e \v{S} \cdot \v{B})_{bc} 
				+ \pi_b \pi_c (e \v{S} \cdot \v{B})_{ba}
				-\gv{\pi}^2 (e \v{S} \cdot \v{B})_{ac}	\\
	&=&	\frac{1}{2} [(e \v{S} \cdot \v{B})_{cb} ( \pi_a \pi_b - \pi_b \pi_a)
				+ (e \v{S} \cdot \v{B})_{ba} (\pi_b \pi_c - \pi_c \pi_b)]
				-\gv{\pi}^2 (e \v{S} \cdot \v{B})_{ac}	\\
	&=&	\frac{1}{2} [(e \v{S} \cdot \v{B})_{cb} (e \v{S} \cdot \v{B})_{ba}
				+ (e \v{S} \cdot \v{B})_{ba} (e \v{S} \cdot \v{B})_{cb}]
				-\gv{\pi}^2 (e \v{S} \cdot \v{B})_{ac}	\\
	&=&	(e \v{S} \cdot \v{B})_{ab}(e \v{S} \cdot \v{B})_{bc})
				-\gv{\pi}^2 (e \v{S} \cdot \v{B})_{ac}	\\
	&=&	\left[ (e \v{S} \cdot \v{B})^2 \right ]_{ac}
				-\gv{\pi}^2 (e \v{S} \cdot \v{B})_{ac}	\\
\end{eqnarray*}

Since we can throw away terms of order $B^2$, the final result tells us that, to first order in B:
\begin{eqnarray*}
 \left(  \frac {\gv{\pi}^2} {2} -  (\v{S} \cdot \gv{\pi})^2   \right )^2
	&=&	\frac{ \gv{\pi}^4 } {4}  -  \gv{\pi}^2 (e \v{S} \cdot \v{B}) \\
	&=& 	\frac{ \gv{\pi}^4 } {4}  -  e \v{p}^2  \v{S} \cdot \v{B} \\
\end{eqnarray*}

\subsubsection*{Second term}
To simplify the second term, we need
\begin{eqnarray*}
\left \{ \frac{\v{p}^2}{2} - (\v{S} \cdot \v{p})^2, \v{S} \cdot \v{B} \right \}
	&=&	 \v{p}^2 \v{S} \cdot \v{B} - [(\v{S} \cdot \v{p})^2 \v{S} \cdot \v{B} + \v{S} \cdot \v{B} (\v{S} \cdot \v{p})^2  ]	\\
	&=&	 \v{p}^2 \v{S} \cdot \v{B} - (S_i S_j S_k + S_k S_j S_i) p_i p_j B_k
\end{eqnarray*}

To simplify that triple product of spin matrices, we can use their explicit form:
\begin{eqnarray*}
	(S_i S_j S_k )_{ab}
		&=&	i\epsilon_{aci}\epsilon_{cdj}\epsilon_{dbk}	\\
		&=&	i(\delta_{id} \delta_{aj} - \delta_{ij} \delta_{ad})\epsilon_{dbk}	\\
		&=&	i(\delta_{aj} \epsilon_{ibk} - \delta_{ij} \epsilon_{abk})		\\
	(S_i S_j S_k + S_k S_j S_i)_{ab}
		&=& i(\delta_aj \epsilon_{ibk} + \delta_{aj} \epsilon_{kbi} -\delta_{ij} \epsilon_{abk} -\delta{kj}\epsilon_{abi})	\\
		&=& -i(\delta_{ij} \epsilon_{abk} + \delta_{kj} \epsilon_{abi}	\\
		&=&	\delta_{ij} {(S_k)}_ab + \delta_{kj} {(S_i)}_{ab}	
\end{eqnarray*}
Now 
\begin{eqnarray*}
 \v{p}^2 \v{S} \cdot \v{B} - (S_i S_j S_k + S_k S_j S_i) p_i p_j B_k
 	&=& \v{p}^2 \v{S} \cdot \v{B} - (\delta_{ij} S_k + \delta_{kj} S_i	) p_i p_j B_k	\\
 	&=& - (\v{S} \cdot \v{p}) (\v{B} \cdot \v{p})
\end{eqnarray*}
\subsubsection*{Result}
At last, the final result is that, to the order we care about 
\beq \label{eq:A:crossterm}
	\left(  \frac {\gv{\pi}^2} {2} -  (\v{S} \cdot \gv{\pi})^2 + (g-2)\frac{e}{m} \v{S} \cdot \v{B} \right )^2 
	=	\frac{ \gv{\pi}^4 } {4}  -  e \v{p}^2  \v{S} \cdot \v{B} \\
		 -\frac{g-2}{2}\frac{e}{m} (\v{S} \cdot \v{p}) (\v{B} \cdot \v{p})
\eeq

