
\section{Genral Spin Formalism}

\subsection{Formalism}
First we need to work out a formalism that will apply to the general spin case.  
We want to represent the spin state of the particles by an object that looks like a generalization of the Dirac bispinor.

%TODO check what to call helical basis
It is easiest to start with the Dirac basis, where the upper and lower components of the bispinor are objects of opposite helicity, each transforming as an object of spin $1/2$.

To that end define an object

\[
\Psi  = \frac{1}{\sqrt{2}} \begin{pmatrix} \xi \\ \eta \end{pmatrix}
\]

that we wish to have the appropriate properties.  Each component should transform as a particle of spin s, but with opposite helicity.  Under reflection the upper and lower components transform into each other.

The irreducible representations of the proper Lorentz group are spinors which are seperately symmetric in dotted and undotted indices.  The spin of the particle will be half the total number of indices.  So if $\xi$ is an object with $p$ undotted and $q$ dotted indices
\[
	\xi = \{ \xi^{\alpha_1 \ldots \alpha_p}_{\dot\beta_1 \ldots \dot\beta_q} \}
\]
Then this is a represenation of a particle of spin $s = (p+q)/2$.

We have some free choice in how to partition the dotted/undotted indices, and we cannot choose exactly the same scheme for all spin as long as both types of indices are present.  However, we can make separately consistent choices for integral and half-integral spin.  For integral spin we can say $p=q=s$, while for the half-integral case we'll choose $p=s+\frac{1}{2}$, $q=s-\frac{1}{2}$.




We want the $\xi$ and $\eta$ to transform as objects of opposite helicity.  Under reflection they will transform into each other.  So 
 \[
	\eta = \{ \eta_{\dot \alpha_1 \ldots \dot \alpha_p}^{\beta_1 \ldots \beta_q} \}
\] 




In the rest frame of the particle, they will have clearly defined and identical properties under rotation.    The rest frame spinors are equivalent to rank $2s$ nonrelativistic spinors.  So the bispinor in the rest frame looks like
\[
\Psi = \frac{1}{\sqrt{2}} \begin{pmatrix} \xi_0 \\ \xi_0 \end{pmatrix}
\]

where
\[
	\xi_0 = \{ (\xi_0)_{\alpha_1 \ldots \alpha_p \beta_1 \ldots \beta_q}  \}
\]
and all indices are symmetric.

We can obtain the spinors in an arbitrary frame by boosting from the rest frame.  The upper and lower components we have defined to have opposite helicity, and so will act in opposite ways under boost:
\[
	\xi = \exp{(\frac{\v{\Sigma} \cdot \v{\rapidity}}{2}) } \xi_0,  
	\hspace{3em} 
	\eta = \exp{(-\frac{\v{\Sigma} \cdot \v{\rapidity}}{2}) } \xi_0
\]

%TODO check dotted/undotted transformations are correct
What form should the operator $\v{\Sigma}$ have?  Under an infitesimal boost by a rapidity $\phi$, a spinor with a single undotted index is transformed as
\[
	\xi_\alpha \to \xi'_\alpha = \left(\delta_{\alpha \beta} + \frac{\gv{\rapidity}\cdot \gv{\sigma}_{\alpha \beta} }{2} \right) \xi_\beta 
\]
while one with a dotted index will transform as
\[
\xi_{\dot\alpha} \to \xi'_{\dot\alpha} = \left(\delta_{\dot \alpha \dot \beta} - \frac{\gv{\rapidity}\cdot \gv{\sigma}_{\dot \alpha \dot\beta} }{2} \right) \xi_{\dot \beta}
\]


The infinitesmal transformation of a higher spin object with the first $p$ indices undotted and the last $q$ dotted would then be
\[
	\xi \to \xi' = \left(1 
		+  \sum\limits_{a=0}^p \frac{\gv{\sigma}_a \cdot \gv{\rapidity} }{2}
		- \sum\limits_{a=p+1}^{p+q} \frac{\gv{\sigma}_a \cdot \gv{\rapidity} }{2}
	\right ) \xi 
\]
where $a$ denotes which spinor index of $\xi$ is operated on.


If we define 
\[
	\v{\Sigma} = \sum\limits_{a=0}^p \gv{\sigma}_a - \sum\limits_{a=p+1}^{p+q} \gv{\sigma}_a 
\]

Then the infinitesmal transformations would be
\[
	\xi \to \xi' = \left( 1 + \frac{\gv{\Sigma} \cdot \gv{\rapidity} }{2} \right) \xi
\]
\[
	\eta \to \eta' = \left( 1 - \frac{\gv{\Sigma} \cdot \gv{\rapidity} }{2} \right) \eta
\]
So the exact transformation should be
\[
		\xi \to \xi' = \exp\left( \frac{\gv{\Sigma} \cdot \gv{\rapidity} }{2} \right) \xi
\]
\[
	\eta \to \eta' = \exp \left( -\frac{\gv{\Sigma} \cdot \gv{\rapidity} }{2} \right) \eta
\]  
  
Therefore, the bispinor of some particle boosted by $\gv{phi}$ will be
% TODO check passive v. active boost
\[
\Psi = \frac{1}{\sqrt{2}} \begin{pmatrix} 
		\exp\left( \frac{\gv{\Sigma} \cdot \gv{\rapidity} }{2} \right)\xi_0 \\ 
		\exp \left( \frac{-\gv{\Sigma} \cdot \gv{\rapidity} }{2} \right) \xi_0 
	\end{pmatrix}
\]


In dealing with the relativistic theory, we'll want a basis that seperates the particle and antiparticle parts of the wave function.  If we want the upper component to be the particle, then in the rest frame the lower component will vanish, and for low momentum will be small compared to the upper component.  The unitary transformation which accomplishes this is

\[
	\Psi' = \begin{pmatrix} \phi \\ \chi \end{pmatrix}
\]

\[
	\phi = \frac{1}{\sqrt{2}}(\xi + \eta)
\]
\[
	\chi = \frac{1}{\sqrt{2}}( \eta - \xi)
\]

Which is equivalent to
\[
	\Psi' = \frac{1}{\sqrt{2}} \begin{pmatrix}1 & 1 \\ -1 & 1 \end{pmatrix} \Psi
\]

Then,
\[
	\phi =  \cosh \left( \frac{\gv{\Sigma} \cdot \gv{\rapidity} }{2} \right ) \xi_0
\]

%Sign confusion compared to original equation again
\[
	\chi =  \sinh \left( \frac{\gv{\Sigma} \cdot \gv{\rapidity} }{2} \right ) \xi_0
\]


\subsection{The NRQED Lagrangian}

We want to construct an effective Lagrangian in the nonrelativistic limit.  Our goal is to calculate the leading order corrections to the $g$-factor, which are corrections of order $\alpha^2$.  To this end, we need terms in the effective nonrelativistic Lagrangian which are equivalent corrections.


%TODO Discussion of energy scales --  possibly move this into introduction, since it applies to each calculation
We consider constant, infinitesimal external magnetic fields, so we need only consider terms linear in $\v{B}$.

The velocity of the particles in our bound state system will be $v \sim \alpha$.

The electric field we consider is the Coulomb field, so $e\Phi \sim m Z\alpha^2 \sim mv^2$, and $eE \sim m^2v^3$.

Each derivative of the electric field will add an additional factor of $mv$, so the operator $\v{D}$ can be taken to be of this order.

We need to keep terms up to order $mv^4$ and $\frac{B}{m} v^2$ in order to calculate the $g$-factor to the necessary precision.  We include $mv^4$ terms so we can be sure that there are no effects entering from second-order perturbation theory.



%TODO insert qualifications on external field

The Lagrangian is constrained to obey several symmetries.  It must be invariant under the symmetries of parity and time reversal.  It must also be invariant under Galilean transformations.  The Lagrangian must also be Hermitian, and gauge invariant.

What are the gauge invariant building blocks we can use to construct this Lagrangian?  We have the external fields $\v{E}$ and $\v{B}$, the spin operators $\v{S}$, and the long derivative $\v{D} = \v{\partial} - i e\v{A}$.  The fields should always be accompanied by the charge $e$ of the particle.

When considering the case of higher spin particles, we might consider terms quadratic and above in spin operators.  For a particle of spin $s$, there must be $(2s+1)^2$ independant hermitian operators.  We can span this set of operators by considering products of up to $2s$ spin matrices which are symmetric and traceless in every vector index.  For example, for spin-$1$ we have quadratic, in addition to $I$ and $S_i$, five independant structures of the form $ S_i S_j + S_j S_i + \delta_{ij} \v{S}$.

We also have the scalar $D_0$, however, we need only include a single such term because we insist on having only one power of the time derivative.

To consider possible terms, we need to know how each of the above behave under the discrete transformations and Hermitian conjugate.  The signs under these transformations are listed in the table below.  (Also included is the imaginary number $i$.)

%TODO cleanup table, don't use pmatrix
\[
\begin{pmatrix}
		& Order	&	P	&	T	&	\dagger	\\
eE_i	&m^2v^3	&	-	& 	+	&	+		\\
eB_i	&m^2v^2	&	+	&   -	&	+		\\
D_i		& mv	&	-	&	+	&	-		\\
D_0		& mv	&	+	&	-	&	-		\\
S_i		& 1		&	+	&	-	&	+		\\
i		& 1		&	+	&	-	&	-		\\
\end{pmatrix}
\]

Our strategy in cataloguing terms will be to first list all the combinations of $E$, $B$ and $D$ which might be allowed at a particular order, to consider the various ways of contracting these vectors, and finally to eliminate terms which do not obey the proper symmetries.  We can always make a particular combination Hermitian, and get the proper behavior under time reversal by adding a factor of $i$, but parity will kill several terms.  Note that of the structures we can contract with, all are even under parity.

%List objects with which we can contract the vector fields

We can also insist that the Lagrangian have the expected form in the absence of external fields, which eliminates terms like $\bar{S}_{ij}D_i D_j$.
The leading order terms should be of order $mv^2$ or $\frac{eB}{m}$.  Combinations of the correct order are:
\begin{itemize}
  \item The single $D_0$ term.  To have the correct transformation properties this should be $iD_0$.
  \item The kinetic $\v{D}^2$ term, which must be simply $\frac{\v{D}^2}{2m}$
  \item A term with a single power of $B_i$.  The only way to contract this is with the spin matrix, so the term will have the form $\frac{e}{m} \v{S} \cdot \v{B}$
\end{itemize}
All these terms are Hermitian in themselves.

So, the allowed terms at this order are:
\[
	iD_0, \frac{\v{D}^2}{m}, \frac{e}{m} \v{S} \cdot \v{B}
\]

The first two terms have their coefficients fixed, while we wish to honestly calculate the factor before the last.

%TODO check signs and other factors for consistency
\[
	\mathcal{L}_{NRQED} = iD_0 -  \frac{\v{D}^2}{2m}  +  c_B \frac{e}{m} \v{S} \cdot \v{B}
\]

Are there any terms of order $mv^3$ or $\frac{B}{m}v$ allowed?  Possible combinations are:
\begin{itemize}
  \item Three powers of D: $D_i D_j D_k$.  However, this is odd under parity, and so not allowed.  
  \item A term with both the derivative and magnetic field: $D_i B_j$.  Again, this is odd under parity and so forbidden.
  \item A single power of $E$.  Again, odd under parity.
\end{itemize}
So, all such terms are foribdden by consideration of parity.

Next we consider terms of order $mv^4$.
\begin{itemize}
  \item Four powers of D: fixed by the kinetic term to be $\frac{\v{D}^2}{8m^3}$
  \item One power of $E_i$ and one of $D_j$.  This combination is even under parity and odd under Hermitian conjugate.  There are three ways of contracting these two fields.
  \begin{itemize}
  		\item	With the delta function.  The allowed Hermitian term is then $\delta_{ij}(D_i E_j - E_j D_i)$.
  		\item	With the combination $i\epsilon_{ijk} S_k$.  The allowed term is $i\epsilon_{ijk}(D_i E_j S_k + S_k E_j D_i)$.
  		\item 	With the quadratic spin structure $Q_{ij}$: $ Q_{ij} ( D_i E_j - E_j D_i)$.
  \end{itemize}
\end{itemize}

In the Lagrangian we'll write these as:
\[
	\mathcal{L}_{mv^4} = 
		\frac{\v{D}^4}{8m^2}
		+ c_D \frac{e (\v{D} \cdot \v{E} - \v{E} \cdot \v{D})}{8m^2} 
		+ c_Q \frac{eQ_{ij}(D_i E_j - E_i D_j)}{8m^2}
		+ c_S \frac{ i e \v{S} \cdot(\v{D} \times \v{E} - \v{E} \times \v{D}}{8m^2}
\]


Terms of order $\frac{B}{m} v^2$.  The only allowed combination is $D_i D_j B_k$.  We can contract two indices with each other and the third with a spin matrix in three different ways:
\begin{itemize}
	\item $(\v{S} \cdot \v{B}) \v{D}^2 +  \v{D}^2 (\v{S} \cdot \v{B})$
	\item $S_i D_j B_i D_j$
	\item $S_i (D_i B_j D_j + D_j B_j D_i)$
\end{itemize}
We can also contract all indices with a cubic spin structure:
\begin{itemize}
  \item $\bar{S}_{ijk} (D_i D_j B_k + B_k D_j D_i)$
  \item $\bar{S}_{ijk} D_i B_j D_k$
\end{itemize}

In the Lagrangian we'll write these as:
\[
	\mathcal{L}_{Bv^2} =
		c_{W1} \frac{ e \v{D}^2 \v{S} \cdot \v{B} + \v{S} \cdot \v{B} \v{D}^2 }{8m^3}
		- c_{W2} \frac{e D_i (\v{S} \cdot \v{B}) D_i}{4m^3}
		+c_{p'p} \frac{ e [ (\v{S}\cdot \v{D})(\v{B} \cdot \v{D}) + (\v{B} \cdot \v{D})(\v{S}\cdot \v{D})]}{8m^3}
\]\[		+ c_{S^3_1} \frac{ e \bar{S}_{ijk} (D_i D_j B_k + B_k D_j D_i)}{8m^3}
		+ c_{S^3_2} \frac{ e \bar{S}_{ijk} D_i B_j D_k }{8m^3}
\]







%Section on bilinear transforms

%FIXME check for possible typos in calculations (I seem to remember there were some here?)


\section{Transformations of bilinears in the case of general spin}
\label{chap:bilinear}
We have the transformation of the spinor under small boosts:
\beqa
	\Psi &\to& \Psi' = \Psi + \frac{\eta_i }{2} \begin{pmatrix} 0 & \Sigma_i \\ \Sigma_i & 0 \end{pmatrix}\Psi
\eeqa
\beqa
	\bar{\Psi} &\to& \bar{\Psi'} = \bar{\Psi} - \frac{\eta_i }{2} \bar{\Psi} \begin{pmatrix} 0 & \Sigma_i \\ \Sigma_i & 0 \end{pmatrix}
\eeqa

We can also see the transformation of the spinor under parity: simply put, because the upper component is even in $\gv{\Sigma} \cdot \v{p}$, whereas the lower component is odd, we obtain
\[
	\Psi \to \begin{pmatrix} 1 & 0 \\ 0 & -1 \end{pmatrix}\Psi
\]
\[	\bar{\Psi} \to \bar{\Psi} \begin{pmatrix} 1 & 0 \\ 0 & -1 \end{pmatrix}
\]
So
\[
	\bar{\Psi} \begin{pmatrix} A & B \\ C & D \end{pmatrix} \Psi
		\to
	\bar{\Psi} \begin{pmatrix} A & -B \\ -C & D \end{pmatrix} \Psi
\]

From these facts we can examine the general behavior of bilinears under Lorentz transformations.

Now we'll examine the behavior of bilinears under boosts.  We can write the general structure of the bilinear as
\[
	\bar{\Psi} T \Psi = 	\bar{\Psi} \begin{pmatrix} A + D & B+C \\ B-C & A - D \end{pmatrix} \Psi
\]
or using a different notation
\[
\bar{\Psi} T \Psi = 	\bar{\Psi} 
	\left [
			A \otimes \begin{pmatrix} 1 & 0 \\ 0 & 1 \end{pmatrix}
			+ D \otimes \begin{pmatrix} 1 & 0 \\ 0 & -1\end{pmatrix}			
			+ B \otimes \begin{pmatrix} 0 & 1 \\ 1 & 0 \end{pmatrix}
			+ C \otimes \begin{pmatrix} 0 & 1 \\ -1 & 0 \end{pmatrix}
	\right)]  \Psi
\]

Under an infitesimal Lorentz boost $\gv{\eta}$ this will transform into
\[
	\bar{\Psi} T \Psi \to \bar{\Psi} T \Psi
		+ \frac{\eta_i}{2} \bar{\Psi} \left (  \begin{pmatrix} A + D & B+C \\ B-C & A - D \end{pmatrix} \begin{pmatrix} 0 & \Sigma_i \\ \Sigma_i & 0 \end{pmatrix} - \begin{pmatrix} 0 & \Sigma_i \\ \Sigma_i & 0 \end{pmatrix} \begin{pmatrix} A + D & B+C \\ B-C & A - D \end{pmatrix} \right ) \Psi					
\]
We can express this in terms of commutators and anti-commutators
\[
	\bar{\Psi} T \Psi \to \bar{\Psi} T \Psi
		+ \frac{\eta_i}{2} \bar{\Psi} \left [
			[B, \Sigma_i] \otimes \begin{pmatrix} 1 & 0 \\ 0 & 1 \end{pmatrix}
			+ [A, \Sigma_i] \otimes \begin{pmatrix} 0 & 1 \\ 1 & 0 \end{pmatrix}
 			+ \{C, \Sigma_i\} \otimes \begin{pmatrix} 1 & 0 \\ 0 & -1\end{pmatrix}
			+ \{D, \Sigma_i\} \otimes \begin{pmatrix} 0 & 1 \\ -1 & 0 \end{pmatrix}
	\right)] \Psi
\]

We can note here that, using only the matrices $\gv{\Sigma}$ and $\gv{S}$ we can build three structures invariant under rotations: and $S^2$, $\Sigma^2$, and $\Sigma \cdot S$.  All three of these structures commute with both $S_i$ and $\Sigma_i$, and their value depends only on the particular representation we're working with.  So for our purposes here, they can just be treated as pure numbers.


\subsubsection{Scalar bilinears}
Since the scalar must be invariant to rotation, then by the logic above it's block elements are proportional to the identity.

It must also be unchanged under boosts.  We can see that this necessitates that $C=D=0$, while providing no constraint on A and B.  So the general form of a bilinear invariant under boosts is
\[
	\bar{\Psi} T \Psi = \bar{\Psi} \begin{pmatrix} A & B \\B & A \end{pmatrix} \Psi
\]
where A and B are proportional to the identity.

Under the discrete partiy transformation this will transform into
\[
	\bar{\Psi} T' \Psi = \bar{\Psi} \begin{pmatrix} A & -B \\-B & A \end{pmatrix} \Psi
\]
This shows that for a true scalar, $B=0$.


\subsubsection{Vector bilinears}
To attempt to construct a vector bilinear, we can start by considering the time-like part of it.  Since this must be invariant under spatial rotations, then by the same logic as above it must essentially be composed of four blocks proportional to the identity.  We also know that under boosts the time-like part is transformed into the spatial and vice versa, so we can use these linked transformations to obtain constraints on the bilinear.

If $T^\mu$ is a vector we know that, under an infinitesimal boost, it's transformation will be
\beqa
	T^0 &\to& T^0 + \eta_i T^i	\\
	T^i &\to& T^i + \eta_i T^0
\eeqa

We again write 
\[T^\mu = \begin{pmatrix}A^\mu + D^\mu & B^\mu+C^\mu \\ B^\mu-C^\mu & A^\mu - D^\mu  \end{pmatrix} \]

Then we see that under boost, $T^0$ transforms as

\[
	\bar{\Psi} T^0 \Psi \to 	\bar{\Psi} T^0 \Psi
	+  \eta_i \bar{\Psi} \left [
			C^0 \Sigma_i \otimes \begin{pmatrix} 1 & 0 \\ 0 & -1\end{pmatrix}
			+ D^0 \Sigma_i \otimes \begin{pmatrix} 0 & 1 \\ -1 & 0 \end{pmatrix}
	\right] \Psi
\]
where we've used that fact that all the components of $T^0$ commute with $\Sigma_i$.
This tells us that for $T^\mu$ to be a 4-vector, the following must be true.
\beqa
	A^i &=& 0	\\
	B^i &=& 0 	\\
	C^i &=& D^0 \Sigma^i	\\
	D^i &=& C^0 \Sigma^i	\\
\eeqa

We can now consider how $T^i$ changes under a boost, and discover

\beqa
\bar{\Psi} T^i \Psi 
	&\to& \bar{\Psi} T^i \Psi
		+ \frac{\eta_j}{2} \bar{\Psi} \left [
			\{C^i, \Sigma^j\} \otimes \begin{pmatrix} 1 & 0 \\ 0 & -1\end{pmatrix}
			+ \{D^i, \Sigma^j\} \otimes \begin{pmatrix} 0 & 1 \\ -1 & 0 \end{pmatrix}
		\right)] \Psi	\\
	&=& \bar{\Psi} T^i \Psi
		+ \frac{\eta_j}{2} \bar{\Psi} \left [
			D^0\{\Sigma^i, \Sigma^j\} \otimes \begin{pmatrix} 1 & 0 \\ 0 & -1\end{pmatrix}
			+ C^0\{\Sigma^i, \Sigma^j\} \otimes \begin{pmatrix} 0 & 1 \\ -1 & 0 \end{pmatrix}
		\right)] \Psi	\\
\eeqa
Again considering our demand that $T^\mu$ transform like a 4-vector, we get
\beqa
	A^0 &=& 0	\\
	B^0 &=& 0	\\
	C^0 \delta^{ij} &=&  C^0 \frac{1}{2}\{\Sigma^i, \Sigma^j\}	\\
	D^0 \delta^{ij} &=&  D^0 \frac{1}{2}\{\Sigma^i, \Sigma^j\}	\\
\eeqa
The last two constraints are met in the spin-1/2 case, but not for higher spins.  This tells us that there's no way to, in the higher spin case, construct a vector bilinear using only I, $\gv{\Sigma}$, and $\v{S}$.

For spin-1/2, where $\Sigma_i = \sigma_i$, we see that a true vector bilinear (with correct transformation properties under parity) will be proportional to 
\[
	(T^0, \v{T} ) = \left( \begin{pmatrix} 1 & 0 \\ 0 & -1 \end{pmatrix} , \begin{pmatrix} 0 & \gv{\sigma} \\ -\gv{\sigma} & 0 \end{pmatrix} \right )
\]
which, of course, is exactly what we knew already.

\subsubsection{Tensor bilinears}
Here we'll be a little less ambitious.  We can tell from the above considerations that, under boosts, we effectively mix A and B components seperately from the C and D blocks.  What's more, we need anti-commutation relationships to deal with the latter transformations.  So we'll just consider tensors that look like
\[
	T^{\mu\nu} = \begin{pmatrix} A^{\mu\nu} & B^{\mu\nu} \\ B^{\mu\nu} & A^{\mu\nu} \end{pmatrix}	
\]
Furthermore, we'll consider only anti-symmetric tensors for now.  Now we can basically procede as in the vector case, knowing how an anti-symmetric tensor should transform:
\beqa
	T^{0i} &\to& T^{0i} +  \eta_j T^{ji}	\\
	T^{ij}	&\to& T^{ij} + \eta_i T^{0j} + \eta_j T^{i0}	\\
\eeqa

Start by considering the components of $T^{0i} = -T{i0}$.  They must transform as vectors under rotations.  We have two vectors available to us, so we can write
\beqa
	A^{0i} &=& \alpha \Sigma^i + \beta S^i	\\
	B^{0i} &=& \gamma \Sigma^i + \delta S^i	\\
\eeqa

Under a boost, we find the relation that
\beqa
	A^{ji} &=& [B^{0i}, \Sigma^j]	\\
	B^{ji} &=& [A^{0i}, \Sigma^j]	\\	
\eeqa
And then looking at how $T^{ij}$ transforms, we get the constraint
\beqa
	\eta_k \left[ [A^{0i}, \Sigma^j], \Sigma^k \right] &=& \eta_j A^{0i} - \eta_i A^{0j}	\\
	\eta_k \left[ [B^{0i}, \Sigma^j], \Sigma^k \right] &=& \eta_j B^{0i} - \eta_i B^{0j}	\\
\eeqa
We're assuming that both $A^{0i}$ and $B^{0i}$ are linear combinations of $\Sigma_i$ and $S_i$.  So what we need are the relationships
\beqa
	[\Sigma^i, \Sigma^j] &=& 4 i \epsilon_{ijk} S^k	\\
	{}[S^i, \Sigma^j] &=& i\epsilon_{ijk} \Sigma^k 	\\
	{}[[\Sigma^i, \Sigma^j], \Sigma^k] 
		&=& 4 i\epsilon_{ij\ell} [S^\ell, \Sigma^k]	\\
		&=& -4 \epsilon_{ij\ell} \epsilon_{\ell k m} \Sigma^m \\		
		&=& -4 (\delta_{ik} \Sigma^j - \delta_{jk} \Sigma^i)	\\ 
	{}[[S^i, \Sigma^j], \Sigma^k] 
		&=&  i\epsilon_{ij\ell} [\Sigma^\ell, \Sigma^k]	\\
		&=& -4 \epsilon_{ij\ell} \epsilon_{\ell k m} S^m \\		
		&=& -4 (\delta_{ik} S^j - \delta_{jk} S^i)	\\ 
\eeqa
So we can see that, no matter what $\alpha$ and $\beta$ are, we get the relation
\[
	[ [A^{0i}, \Sigma^j], \Sigma^k]
		=
	-4 (\delta{ik} A^{0i} -\delta_{jk} A^{0j } 
\]
And so necessarily, 
\[
	\eta_k[ [A^{0i}, \Sigma^j], \Sigma^k]
		=
	4 (\eta_j A^{0i} -  \eta_i A^{0j} )
\]
So any arbitrary combination of $\v{S}$ and $\gv{\Sigma}$ will allow us to construct a bilinear that transforms as a tensor.

(In fact, it's not hard to generalise this to any operator expressable as a linear combination of $\sigma^A_i$, where the index A represents which spinor index $\sigma$ operates on.)




\subsection{Electromagnetic Interaction}
%TODO insert diagram, showing the type of interaction we're talking about, defining momenta of particles in question.
Knowing how the wave functions themselves behave, we want to see what that tells us about possible electromagnetic interaction.  Interaction with a single electromagnetic photon should take the form

\[
	M = A_\mu j^\mu 
\]
where $j^\mu$ is the electromagnetic current.


The electromagnetic current must be built out of the particle's momenta and bilinears of the charged particle fields in such a way that they have the correct Lorentz properties.  We must also demand current conservation: the equation $q_\mu j^\mu = 0$ must hold.  Above we already have shown that, in the case of general spin, there exist only two such bilinears, a scalar and a tensor.

There will be two permissible terms in the current.  We could consider a scalar bilinear coupled with a single power of external momenta.  In order to fufill the current conservation requirement, it should be
\[
	\frac{p^\mu + p'^\mu}{2m} \Psibar^\dagger \Psi
\]
This will obey current conservation because $q = p' -p$, and $ (p+p')\cdot(p'-p) = p^2-p'^2=0$

We can also consider a tensor term contracted with a power of momenta.  To fufill current conservation, we can demand that the tensor bilinear be antisymmetric, and contract it with $q$:
\[
	\frac{q_\nu}{2m} \Psibar^\dagger T^{\mu\nu} \Psi
\]

We don't need to worry about higher order tensor bilinears: they will necessitate too many powers of the external momenta.

So the most general current would look like
\[
	j^\mu = F_e \frac{p^\mu + p'^\mu}{2m} \Psibar^\dagger \Psi + F_m 	\frac{q_\nu}{2m} \Psibar^\dagger T^{\mu\nu} \Psi	
\]
In general the form factors might have quite complicated dependence on $q$, but these corrections will be too small compared to the type of result we're interested in.  At leading order $F_e$ will just be the electric charge of the particle in question, and $F_m$ will, as we'll see after connecting this result to the nonrelativistic limit, be related to the particle's $g$-factor.  So to the order we need, we can write the current as

%TODO check definition of form factors F_e and F_m
\[
	j^\mu =  e \frac{p^\mu + p'^\mu}{2m} \Psibar^\dagger \Psi +   e g \frac{q_\nu}{2m} \Psibar^\dagger T^{\mu\nu} \Psi
\]

%TODO Actually show this formally: that the two types of antisymmetric tensors aren't truly different
The tensor bilinear $\Psibar^\dagger T^{\mu\nu} \Psi$ itself has some free parameters.  However, it'll turn out that, when computing actual nonrelativistic amplitudes, the two types of anti-symmetric tensors collapse into the same general form.

This captures the essence of the interaction between a charged particle of general-spin and a single photon.


\subsection{Connection to nonrelativistic spinors}


In the rest frame, there are two independant bispinors which represent particle and antiparticle states: 
\[
	\Psi = \begin{pmatrix} \xi_0 \\ 0 \end{pmatrix}
\]

or
\[
	\Psi = \begin{pmatrix} 0 \\ \xi_0 \end{pmatrix}
\]

However, when we consider a particle with zero momentum it is not the case that the upper component of the bispinor can be directly associated with the Shrodinger like wave-function of the particle --- for instance, it would not be correctly normalized, for there is some mixing with the lower component.

We can obtain a relation between $\xi_0$ and the Shrodinger amplitude $\phi_s$ by considering the current density at zero momentum transfer.  For $\phi_s$ it will be $j_0 = e \phi_s^\dagger \phi$.  For the relativistic theory we have, as calculated above:
\[
	j^0 = e \frac{p^0 + p'^0}{2m} \Psibar^\dagger \Psi + F_m 	\frac{q_\nu}{2m} \Psibar^\dagger T^{0\nu} \Psi	
\]
At $q=0$ the expression simplifies

\[
	j^0(q=0) = e \frac{p_0}{m} \Psibar^\dagger \Psi
\]

\[
	= e  \frac{p_0}{m}( \phi^\dagger \phi - \chi^\dagger \chi )
\]

$\phi$ and $\chi$ are both related to the rest frame spinor $\xi_0$.  So we can write instead
\[
	j^0 = e \frac{p_0}{m} \xi_0^\dagger \left \{ 
		\cosh^2( \frac{\gv{\Sigma} \cdot \gv{\rapidity} }{2})
		- \sinh^2( \frac{\gv{\Sigma} \cdot \gv{\rapidity} }{2})
	\right \} \xi_0  
		=	e \frac{p_0}{m} \xi_0^\dagger \xi_0
\]
where the last equality follows from the hyperbolic trig identity.

If we demand that the two current densities be equal to each other, we find
\[
	\frac{p_0}{m} \xi_0^\dagger \xi_0 = \phi_s^\dagger \phi_s
\]
Approximating
\[
	\left( 1 + \frac{\v{p}^2}{2m} \right) \xi_0^\dagger \xi_0 = \phi_s^\dagger \phi_s
\]

This will hold to the necessary order if we identify
\[
	\xi_0 = \left( 1 - \frac{\v{p}^2}{4m} \right) \phi_s
\]

%TODO fix cosh expansion

To write the relativistic bispinors in terms of $\phi_s$ we will also need approximations to $\cosh( \frac{\gv{\Sigma} \cdot \gv{\rapidity} }{2})$ and $\sinh( \frac{\gv{\Sigma} \cdot \gv{\rapidity} }{2})$.  We only need the rapidity to the leading order: $\gv{\rapidity} \approx \v{v} \approx \frac{\v{p} }{m}$. 

\[
	\cosh( \frac{\gv{\Sigma} \cdot \gv{\rapidity} }{2}) 
		\approx 1 + \left( \frac{\gv{\Sigma} \cdot \v{p} }{2m} \right)^2
\]
\[
	\sinh( \frac{\gv{\Sigma} \cdot \gv{\rapidity} }{2}) 
		\approx   \frac{\gv{\Sigma} \cdot \v{p} }{2m}
\]

The the two bispinor components are
\beqa
\phi 
	&\approx&  \left(  1 + \left[ \frac{\gv{\Sigma} \cdot \v{p} }{2m} \right]^2 \right) \xi_0 \\
	&\approx&  \left(  1 + \left[ \frac{\gv{\Sigma} \cdot \v{p} }{2m} \right]^2 - \frac{\v{p}^2}{4m} \right ) \phi_s	\\
 \chi
 	&\approx&	\frac{\gv{\Sigma} \cdot \v{p} }{2m} \xi_0 \\
 	&\approx&	\frac{\gv{\Sigma} \cdot \v{p} }{2m} \phi_s
\eeqa



%TODO remember to hunt out all the \phi where I mean \rapidity
\subsection{Bilinears in terms of nonrelativistic theory}
The next step is to express the relativistic bilinears, built out of the bispinors $\Psi$, in terms of the Shrodinger like wave functions.

We have above written the bispinors in terms of $\phi_s$, so we can use those identities to express the bilinears in the same manner.

\beqa
\Psibar^\dagger(p') \Psi(p)
	&=&	\phi^\dagger \phi - \chi^\dagger \chi	\\
	&=&	\phi_s^\dagger \left[1 + \frac{(\gv{\Sigma} \cdot \v{p'})^2 }{8m^2}  - \frac{\v{p'}^2}{4m^2} \right ]
			 \left[1 + \frac{(\gv{\Sigma} \cdot \v{p})^2 }{8m^2}  - \frac{\v{p}^2}{4m^2} \right ] \phi_s
		- \phi_s^\dagger \left[
			\frac{ ( \gv{\Sigma} \cdot \v{p'}) (\gv{\Sigma} \cdot \v{p}) }{4m^2}
		\right ] \phi_s	\\
	&=&	\phi_s^\dagger \left (
			1 - \frac{ \v{p}^2 + \v{p'}^2 }{4m^2}
			+ \frac{1}{8m^2} \left \{
				( \gv{\Sigma} \cdot \v{p'})^2 +  (\gv{\Sigma} \cdot \v{p})^2 
				 - 2 ( \gv{\Sigma} \cdot \v{p'}) (\gv{\Sigma} \cdot \v{p})
			\right \}
	\right ) \phi_s	\\
	&=& \phi_s^\dagger \left (
			1 - \frac{ \v{p}^2 + \v{p'}^2 }{4m^2}
			+ \frac{1}{8m^2} \left \{
				[ \gv{\Sigma} \cdot \v{p},  \gv{\Sigma} \cdot \v{q}]  + ( \gv{\Sigma} \cdot \v{q})^2 
			\right \}
	\right ) \phi_s	\\
	&=& \phi_s^\dagger \left (
			1 - \frac{ \v{p}^2 + \v{p'}^2 }{4m^2}
			+ \frac{1}{8m^2} \left \{
				[ 4 i \epsilon_{ijk} p_i q_j S_k  + ( \gv{\Sigma} \cdot \v{q})^2 
			\right \}
	\right ) \phi_s
\eeqa
%TODO at some point need to translate \Sigma functions into spin functions in NR theory, but where?




In calculating the nonrelativistic limit of the antisymmetric tensor bilinear, we will treat the $0i$ and the $ij$ components seperately.



