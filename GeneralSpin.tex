
\chapter{General Spin Formalism}

%name? 

Our ultimate goal is to calculate corrections to the $g$-factor of a loosely bound charged particle of arbitrary spin.  Our strategy is to obtain an effective Lagrangian in the nonrelativistic limit.

We first consider features of a general-spin formalism in both the relativistic and nonrelativistic cases, and the connection between the wave functions of the free particles.  Then we consider how constraints of the relativistic theory let us calculate scattering off an external field.  Comparing this result to that done with an effective NRQED Lagrangian, we can obtain the coefficients of that Lagrangian for particles of general spin.


\subsection{Spinors for general-spin charged particles}


\subsubsection{Relativistic bispinors}
First we need to work out a formalism that will apply to the general spin case.  
We want to represent the spin state of the particles by an object that looks like a generalization of the Dirac bispinor.

%TODO check what to call helical basis
It is easiest to start with the Dirac basis, where the upper and lower components of the bispinor are objects of opposite helicity, each transforming as an object of spin $1/2$.

To that end define an object

\beq \label{eq:PsiDef}
\Psi  = \frac{1}{\sqrt{2}} \begin{pmatrix} \xi \\ \eta \end{pmatrix}
\eeq

that we wish to have the appropriate properties.  Each component should transform as a particle of spin $s$, but with opposite helicity.  Under reflection the upper and lower components transform into each other.

Representations of the proper Lorentz group are spinors which are separately symmetric in dotted and undotted indices.  If $\xi$ is an object with $p$ undotted and $q$ dotted indices
\beq
	\xi = \{ \xi^{\alpha_1 \ldots \alpha_p}_{\dot\beta_1 \ldots \dot\beta_q} \}
\eeq
Then this can be a representation of a particle of spin $s = (p+q)/2$.

We have some free choice in how to partition the dotted/undotted indices, and we cannot choose exactly the same scheme for all spin as long as both types of indices are present.  However, we can make separately consistent choices for integral and half-integral spin.  For integral spin we can say $p=q=s$, while for the half-integral case we'll choose $p=s+\frac{1}{2}$, $q=s-\frac{1}{2}$.

We want the $\xi$ and $\eta$ to transform as objects of opposite helicity.  Under reflection they will transform into each other.  So 
\beq
	\eta = \{ \eta_{\dot \alpha_1 \ldots \dot \alpha_p}^{\beta_1 \ldots \beta_q} \}
\eeq




In the rest frame of the particle, they will have clearly defined and identical properties under rotation.    The rest frame spinors are equivalent to rank $2s$ nonrelativistic spinors.  So the bispinor in the rest frame looks like
\beq \label{eq:PsiRest}
\Psi = \frac{1}{\sqrt{2}} \begin{pmatrix} \xi_0 \\ \xi_0 \end{pmatrix}
\eeq

where
\beq \label{eq:xi0def}
	\xi_0 = \{ (\xi_0)_{\alpha_1 \ldots \alpha_p \beta_1 \ldots \beta_q}  \}
\eeq
and all indices are symmetric.

We can obtain the spinors in an arbitrary frame by boosting from the rest frame.  The upper and lower components we have defined to have opposite helicity, and so will act in opposite ways under boost:
\beq \label{eq:xi0boosted}
	\xi = \exp{(\frac{\v{\Sigma} \cdot \v{\rapidity}}{2}) } \xi_0,  
	\hspace{3em} 
	\eta = \exp{(-\frac{\v{\Sigma} \cdot \v{\rapidity}}{2}) } \xi_0
\eeq

%TODO check dotted/undotted transformations are correct
What form should the operator $\v{\Sigma}$ have?  Under an infinitesimal boost by a rapidity $\phi$, a spinor with a single undotted index is transformed as
\beqB
	\xi_\alpha \to \xi'_\alpha = \left(\delta_{\alpha \beta} + \frac{\gv{\rapidity}\cdot \gv{\sigma}_{\alpha \beta} }{2} \right) \xi_\beta 
\eeqB
while one with a dotted index will transform as
\beqB
\xi_{\dot\alpha} \to \xi'_{\dot\alpha} = \left(\delta_{\dot \alpha \dot \beta} - \frac{\gv{\rapidity}\cdot \gv{\sigma}_{\dot \alpha \dot\beta} }{2} \right) \xi_{\dot \beta}
\eeqB



The infinitesimal transformation of a higher spin object with the first $p$ indices undotted and the last $q$ dotted would then be
\beqB
	\xi \to \xi' = \left(1 
		+  \sum\limits_{a=0}^p \frac{\gv{\sigma}_a \cdot \gv{\rapidity} }{2}
		- \sum\limits_{a=p+1}^{p+q} \frac{\gv{\sigma}_a \cdot \gv{\rapidity} }{2}
	\right ) \xi 
\eeqB
where $a$ denotes which spinor index of $\xi$ is operated on.


If we define 
\beq \label{eq:SigDef}
	\v{\Sigma} = \sum\limits_{a=0}^p \gv{\sigma}_a - \sum\limits_{a=p+1}^{p+q} \gv{\sigma}_a 
\eeq

Then the infinitesimal transformations would be
\beqB
	\xi \to \xi' = \left( 1 + \frac{\gv{\Sigma} \cdot \gv{\rapidity} }{2} \right) \xi
\eeqB
\beqB
	\eta \to \eta' = \left( 1 - \frac{\gv{\Sigma} \cdot \gv{\rapidity} }{2} \right) \eta
\eeqB
So the exact transformation should be
\beqB
		\xi \to \xi' = \exp\left( \frac{\gv{\Sigma} \cdot \gv{\rapidity} }{2} \right) \xi
\eeqB
\beqB
	\eta \to \eta' = \exp \left( -\frac{\gv{\Sigma} \cdot \gv{\rapidity} }{2} \right) \eta
\eeqB
  
Therefore, the bispinor of some particle boosted by $\gv{\phi}$ from rest will be
% TODO check passive v. active boost
\beq \label{eq:PsiByXi0}
\Psi = \frac{1}{\sqrt{2}} \begin{pmatrix} 
		\exp\left( \frac{\gv{\Sigma} \cdot \gv{\rapidity} }{2} \right)\xi_0 \\ 
		\exp \left( \frac{-\gv{\Sigma} \cdot \gv{\rapidity} }{2} \right) \xi_0 
	\end{pmatrix}
\eeq

%TODO fix sqrt(2) factors in the right place
In dealing with the relativistic theory, we'll want a basis that separates the particle and antiparticle parts of the wave function.  If we want the upper component to be the particle, then in the rest frame the lower component will vanish, and for low momentum will be small compared to the upper component.  The unitary transformation which accomplishes this is

\[
	\Psi' = \begin{pmatrix} \phi \\ \chi \end{pmatrix}
\]

\[
	\phi = \frac{1}{\sqrt{2}}(\xi + \eta)
\]
\[
	\chi = \frac{1}{\sqrt{2}}( \eta - \xi)
\]


Which is equivalent to
\[
	\Psi' = \frac{1}{\sqrt{2}} \begin{pmatrix}1 & 1 \\ -1 & 1 \end{pmatrix} \Psi
\]

Then,
\beq \label{eq:phiDef}
	\phi =  \cosh \left( \frac{\gv{\Sigma} \cdot \gv{\rapidity} }{2} \right ) \xi_0
\eeq

%Sign confusion compared to original equation again
\beq \label{eq:chiDef}
	\chi =  \sinh \left( \frac{\gv{\Sigma} \cdot \gv{\rapidity} }{2} \right ) \xi_0
\eeq

\subsubsection{Spinors for nonrelativistic theory}

We also need to discuss the nonrelativistic, single-particle theory.  This is much simpler: the spin state of particles in this theory is represented by symmetric spinors with $2s+1$ undotted indices.  The only operators we need to consider acting on this space are spin matrices and products of spin matrices. 

%Section on bilinear transforms
%
%FIXME check for possible typos in calculations (I seem to remember there were some here?)


\section{Transformations of bilinears in the case of general spin}
\label{chap:bilinear}
We have the transformation of the spinor under small boosts:
\beqa
	\Psi &\to& \Psi' = \Psi + \frac{\eta_i }{2} \begin{pmatrix} 0 & \Sigma_i \\ \Sigma_i & 0 \end{pmatrix}\Psi
\eeqa
\beqa
	\bar{\Psi} &\to& \bar{\Psi'} = \bar{\Psi} - \frac{\eta_i }{2} \bar{\Psi} \begin{pmatrix} 0 & \Sigma_i \\ \Sigma_i & 0 \end{pmatrix}
\eeqa

We can also see the transformation of the spinor under parity: simply put, because the upper component is even in $\gv{\Sigma} \cdot \v{p}$, whereas the lower component is odd, we obtain
\[
	\Psi \to \begin{pmatrix} 1 & 0 \\ 0 & -1 \end{pmatrix}\Psi
\]
\[	\bar{\Psi} \to \bar{\Psi} \begin{pmatrix} 1 & 0 \\ 0 & -1 \end{pmatrix}
\]
So
\[
	\bar{\Psi} \begin{pmatrix} A & B \\ C & D \end{pmatrix} \Psi
		\to
	\bar{\Psi} \begin{pmatrix} A & -B \\ -C & D \end{pmatrix} \Psi
\]

From these facts we can examine the general behavior of bilinears under Lorentz transformations.

Now we'll examine the behavior of bilinears under boosts.  We can write the general structure of the bilinear as
\[
	\bar{\Psi} T \Psi = 	\bar{\Psi} \begin{pmatrix} A + D & B+C \\ B-C & A - D \end{pmatrix} \Psi
\]
or using a different notation
\[
\bar{\Psi} T \Psi = 	\bar{\Psi} 
	\left [
			A \otimes \begin{pmatrix} 1 & 0 \\ 0 & 1 \end{pmatrix}
			+ D \otimes \begin{pmatrix} 1 & 0 \\ 0 & -1\end{pmatrix}			
			+ B \otimes \begin{pmatrix} 0 & 1 \\ 1 & 0 \end{pmatrix}
			+ C \otimes \begin{pmatrix} 0 & 1 \\ -1 & 0 \end{pmatrix}
	\right)]  \Psi
\]

Under an infitesimal Lorentz boost $\gv{\eta}$ this will transform into
\[
	\bar{\Psi} T \Psi \to \bar{\Psi} T \Psi
		+ \frac{\eta_i}{2} \bar{\Psi} \left (  \begin{pmatrix} A + D & B+C \\ B-C & A - D \end{pmatrix} \begin{pmatrix} 0 & \Sigma_i \\ \Sigma_i & 0 \end{pmatrix} - \begin{pmatrix} 0 & \Sigma_i \\ \Sigma_i & 0 \end{pmatrix} \begin{pmatrix} A + D & B+C \\ B-C & A - D \end{pmatrix} \right ) \Psi					
\]
We can express this in terms of commutators and anti-commutators
\[
	\bar{\Psi} T \Psi \to \bar{\Psi} T \Psi
		+ \frac{\eta_i}{2} \bar{\Psi} \left [
			[B, \Sigma_i] \otimes \begin{pmatrix} 1 & 0 \\ 0 & 1 \end{pmatrix}
			+ [A, \Sigma_i] \otimes \begin{pmatrix} 0 & 1 \\ 1 & 0 \end{pmatrix}
 			+ \{C, \Sigma_i\} \otimes \begin{pmatrix} 1 & 0 \\ 0 & -1\end{pmatrix}
			+ \{D, \Sigma_i\} \otimes \begin{pmatrix} 0 & 1 \\ -1 & 0 \end{pmatrix}
	\right)] \Psi
\]

We can note here that, using only the matrices $\gv{\Sigma}$ and $\gv{S}$ we can build three structures invariant under rotations: and $S^2$, $\Sigma^2$, and $\Sigma \cdot S$.  All three of these structures commute with both $S_i$ and $\Sigma_i$, and their value depends only on the particular representation we're working with.  So for our purposes here, they can just be treated as pure numbers.


\subsubsection{Scalar bilinears}
Since the scalar must be invariant to rotation, then by the logic above it's block elements are proportional to the identity.

It must also be unchanged under boosts.  We can see that this necessitates that $C=D=0$, while providing no constraint on A and B.  So the general form of a bilinear invariant under boosts is
\[
	\bar{\Psi} T \Psi = \bar{\Psi} \begin{pmatrix} A & B \\B & A \end{pmatrix} \Psi
\]
where A and B are proportional to the identity.

Under the discrete partiy transformation this will transform into
\[
	\bar{\Psi} T' \Psi = \bar{\Psi} \begin{pmatrix} A & -B \\-B & A \end{pmatrix} \Psi
\]
This shows that for a true scalar, $B=0$.


\subsubsection{Vector bilinears}
To attempt to construct a vector bilinear, we can start by considering the time-like part of it.  Since this must be invariant under spatial rotations, then by the same logic as above it must essentially be composed of four blocks proportional to the identity.  We also know that under boosts the time-like part is transformed into the spatial and vice versa, so we can use these linked transformations to obtain constraints on the bilinear.

If $T^\mu$ is a vector we know that, under an infinitesimal boost, it's transformation will be
\beqa
	T^0 &\to& T^0 + \eta_i T^i	\\
	T^i &\to& T^i + \eta_i T^0
\eeqa

We again write 
\[T^\mu = \begin{pmatrix}A^\mu + D^\mu & B^\mu+C^\mu \\ B^\mu-C^\mu & A^\mu - D^\mu  \end{pmatrix} \]

Then we see that under boost, $T^0$ transforms as

\[
	\bar{\Psi} T^0 \Psi \to 	\bar{\Psi} T^0 \Psi
	+  \eta_i \bar{\Psi} \left [
			C^0 \Sigma_i \otimes \begin{pmatrix} 1 & 0 \\ 0 & -1\end{pmatrix}
			+ D^0 \Sigma_i \otimes \begin{pmatrix} 0 & 1 \\ -1 & 0 \end{pmatrix}
	\right] \Psi
\]
where we've used that fact that all the components of $T^0$ commute with $\Sigma_i$.
This tells us that for $T^\mu$ to be a 4-vector, the following must be true.
\beqa
	A^i &=& 0	\\
	B^i &=& 0 	\\
	C^i &=& D^0 \Sigma^i	\\
	D^i &=& C^0 \Sigma^i	\\
\eeqa

We can now consider how $T^i$ changes under a boost, and discover

\beqa
\bar{\Psi} T^i \Psi 
	&\to& \bar{\Psi} T^i \Psi
		+ \frac{\eta_j}{2} \bar{\Psi} \left [
			\{C^i, \Sigma^j\} \otimes \begin{pmatrix} 1 & 0 \\ 0 & -1\end{pmatrix}
			+ \{D^i, \Sigma^j\} \otimes \begin{pmatrix} 0 & 1 \\ -1 & 0 \end{pmatrix}
		\right)] \Psi	\\
	&=& \bar{\Psi} T^i \Psi
		+ \frac{\eta_j}{2} \bar{\Psi} \left [
			D^0\{\Sigma^i, \Sigma^j\} \otimes \begin{pmatrix} 1 & 0 \\ 0 & -1\end{pmatrix}
			+ C^0\{\Sigma^i, \Sigma^j\} \otimes \begin{pmatrix} 0 & 1 \\ -1 & 0 \end{pmatrix}
		\right)] \Psi	\\
\eeqa
Again considering our demand that $T^\mu$ transform like a 4-vector, we get
\beqa
	A^0 &=& 0	\\
	B^0 &=& 0	\\
	C^0 \delta^{ij} &=&  C^0 \frac{1}{2}\{\Sigma^i, \Sigma^j\}	\\
	D^0 \delta^{ij} &=&  D^0 \frac{1}{2}\{\Sigma^i, \Sigma^j\}	\\
\eeqa
The last two constraints are met in the spin-1/2 case, but not for higher spins.  This tells us that there's no way to, in the higher spin case, construct a vector bilinear using only I, $\gv{\Sigma}$, and $\v{S}$.

For spin-1/2, where $\Sigma_i = \sigma_i$, we see that a true vector bilinear (with correct transformation properties under parity) will be proportional to 
\[
	(T^0, \v{T} ) = \left( \begin{pmatrix} 1 & 0 \\ 0 & -1 \end{pmatrix} , \begin{pmatrix} 0 & \gv{\sigma} \\ -\gv{\sigma} & 0 \end{pmatrix} \right )
\]
which, of course, is exactly what we knew already.

\subsubsection{Tensor bilinears}
Here we'll be a little less ambitious.  We can tell from the above considerations that, under boosts, we effectively mix A and B components seperately from the C and D blocks.  What's more, we need anti-commutation relationships to deal with the latter transformations.  So we'll just consider tensors that look like
\[
	T^{\mu\nu} = \begin{pmatrix} A^{\mu\nu} & B^{\mu\nu} \\ B^{\mu\nu} & A^{\mu\nu} \end{pmatrix}	
\]
Furthermore, we'll consider only anti-symmetric tensors for now.  Now we can basically procede as in the vector case, knowing how an anti-symmetric tensor should transform:
\beqa
	T^{0i} &\to& T^{0i} +  \eta_j T^{ji}	\\
	T^{ij}	&\to& T^{ij} + \eta_i T^{0j} + \eta_j T^{i0}	\\
\eeqa

Start by considering the components of $T^{0i} = -T{i0}$.  They must transform as vectors under rotations.  We have two vectors available to us, so we can write
\beqa
	A^{0i} &=& \alpha \Sigma^i + \beta S^i	\\
	B^{0i} &=& \gamma \Sigma^i + \delta S^i	\\
\eeqa

Under a boost, we find the relation that
\beqa
	A^{ji} &=& [B^{0i}, \Sigma^j]	\\
	B^{ji} &=& [A^{0i}, \Sigma^j]	\\	
\eeqa
And then looking at how $T^{ij}$ transforms, we get the constraint
\beqa
	\eta_k \left[ [A^{0i}, \Sigma^j], \Sigma^k \right] &=& \eta_j A^{0i} - \eta_i A^{0j}	\\
	\eta_k \left[ [B^{0i}, \Sigma^j], \Sigma^k \right] &=& \eta_j B^{0i} - \eta_i B^{0j}	\\
\eeqa
We're assuming that both $A^{0i}$ and $B^{0i}$ are linear combinations of $\Sigma_i$ and $S_i$.  So what we need are the relationships
\beqa
	[\Sigma^i, \Sigma^j] &=& 4 i \epsilon_{ijk} S^k	\\
	{}[S^i, \Sigma^j] &=& i\epsilon_{ijk} \Sigma^k 	\\
	{}[[\Sigma^i, \Sigma^j], \Sigma^k] 
		&=& 4 i\epsilon_{ij\ell} [S^\ell, \Sigma^k]	\\
		&=& -4 \epsilon_{ij\ell} \epsilon_{\ell k m} \Sigma^m \\		
		&=& -4 (\delta_{ik} \Sigma^j - \delta_{jk} \Sigma^i)	\\ 
	{}[[S^i, \Sigma^j], \Sigma^k] 
		&=&  i\epsilon_{ij\ell} [\Sigma^\ell, \Sigma^k]	\\
		&=& -4 \epsilon_{ij\ell} \epsilon_{\ell k m} S^m \\		
		&=& -4 (\delta_{ik} S^j - \delta_{jk} S^i)	\\ 
\eeqa
So we can see that, no matter what $\alpha$ and $\beta$ are, we get the relation
\[
	[ [A^{0i}, \Sigma^j], \Sigma^k]
		=
	-4 (\delta{ik} A^{0i} -\delta_{jk} A^{0j } 
\]
And so necessarily, 
\[
	\eta_k[ [A^{0i}, \Sigma^j], \Sigma^k]
		=
	4 (\eta_j A^{0i} -  \eta_i A^{0j} )
\]
So any arbitrary combination of $\v{S}$ and $\gv{\Sigma}$ will allow us to construct a bilinear that transforms as a tensor.

(In fact, it's not hard to generalise this to any operator expressable as a linear combination of $\sigma^A_i$, where the index A represents which spinor index $\sigma$ operates on.)





\subsection{Electromagnetic Interaction}
%TODO insert diagram, showing the type of interaction we're talking about, defining momenta of particles in question.
Knowing how the wave functions themselves behave, we want to see what that tells us about possible electromagnetic interaction.  Interaction with a single electromagnetic photon should take the form

\[
	M = A_\mu j^\mu 
\]
where $j^\mu$ is the electromagnetic current.


The electromagnetic current must be built out of the particle's momenta and bilinears of the charged particle fields in such a way that they have the correct Lorentz properties.  We must also demand current conservation: the equation $q_\mu j^\mu = 0$ must hold.  Above we already have shown that, in the case of general spin, there exist only two such bilinears, a scalar and a tensor.

There will be two permissible terms in the current.  We could consider a scalar bilinear coupled with a single power of external momenta.  In order to fulfill the current conservation requirement, it should be
\[
	\frac{p^\mu + p'^\mu}{2m} \Psibar^\dagger \Psi
\]
This will obey current conservation because $q = p' -p$, and $ (p+p')\cdot(p'-p) = p^2-p'^2=0$

We can also consider a tensor term contracted with a power of momenta.  To fulfill current conservation, we can demand that the tensor bilinear be antisymmetric, and contract it with $q$:
\[
	\frac{q_\nu}{2m} \Psibar^\dagger \TensBi^{\mu\nu} \Psi
\]

We don't need to worry about higher order tensor bilinears: they will necessitate too many powers of the external momenta.

So the most general current would look like
\beq \label{eq:khr_current}
	j^\mu = F_e \frac{p^\mu + p'^\mu}{2m} \Psibar^\dagger \Psi + F_m 	\frac{q_\nu}{2m} \Psibar^\dagger \TensBi^{\mu\nu} \Psi	
\eeq
In general the form factors might have quite complicated dependence on $q$, but these corrections will be too small compared to the type of result we're interested in.  At leading order $F_e$ will just be the electric charge of the particle in question, and $F_m$ will, as we'll see after connecting this result to the nonrelativistic limit, be related to the particle's $g$-factor.  So to the order we need, we can write the current as

%TODO check definition of form factors F_e and F_m
\beq 
	j^\mu =  e \frac{p^\mu + p'^\mu}{2m} \Psibar^\dagger \Psi +   e g \frac{q_\nu}{2m} \Psibar^\dagger \TensBi^{\mu\nu} \Psi
\eeq

%TODO Actually show this formally: that the two types of antisymmetric tensors aren't truly different
%The tensor bilinear $\Psibar^\dagger T^{\mu\nu} \Psi$ itself has some free parameters.  However, it'll turn out that, when computing actual nonrelativistic amplitudes, the two types of anti-symmetric tensors collapse into the same general form.

This captures the essence of the interaction between a charged particle of general-spin and a single photon.




\subsubsection{Connection between the spinors of the two theories}
Knowing something of how the relativistic theory behaves, we can find the connection between the relativistic and nonrelativistic spinors.  In the rest frame, there are two independent bispinors which represent particle and antiparticle states: 
\[
	\Psi = \begin{pmatrix} \xi_0 \\ 0 \end{pmatrix}
\]

or
\[
	\Psi = \begin{pmatrix} 0 \\ \xi_0 \end{pmatrix}
\]

However, when we consider a particle with zero momentum it is not the case that the upper component of the bispinor can be directly associated with the Schrodinger like wave-function of the particle --- for instance, it would not be correctly normalized, for there is some mixing with the lower component.

We can obtain a relation between $\xi_0$ and the Schrodinger amplitude $\phi_s$ by considering the current density at zero momentum transfer.  For $\phi_s$ it will be $j_0 =  \phi_s^\dagger \phi$.  For the relativistic theory we have, as calculated above:
\[
	j^0 = F_e \frac{p^0 + p'^0}{2m} \Psibar^\dagger \Psi + F_m 	\frac{q_\nu}{2m} \Psibar^\dagger T^{0\nu} \Psi	
\]
At $q=0$ the expression simplifies

\[
	j^0(q=0) = F_e \frac{p_0}{m} \Psibar^\dagger \Psi
\]

\[
	= F_e  \frac{p_0}{m}( \phi^\dagger \phi - \chi^\dagger \chi )
\]

$\phi$ and $\chi$ are both related to the rest frame spinor $\xi_0$.  So we can write instead
\[
	j^0 = F_e \frac{p_0}{m} \xi_0^\dagger \left \{ 
		\cosh^2( \frac{\gv{\Sigma} \cdot \gv{\rapidity} }{2})
		- \sinh^2( \frac{\gv{\Sigma} \cdot \gv{\rapidity} }{2})
	\right \} \xi_0  
		=	F_e \frac{p_0}{m} \xi_0^\dagger \xi_0
\]
where the last equality follows from the hyperbolic trig identity.


If we demand that the two current densities be equal to each other, we find
\[
	\frac{p_0}{m} \xi_0^\dagger \xi_0 = \phi_s^\dagger \phi_s
\]
Approximating
\[
	\left( 1 + \frac{\v{p}^2}{2m} \right) \xi_0^\dagger \xi_0 = \phi_s^\dagger \phi_s
\]

This will hold to the necessary order if we identify
\[
	\xi_0 = \left( 1 - \frac{\v{p}^2}{4m} \right) \phi_s
\]


%TODO fix cosh expansion

To write the relativistic bispinors in terms of $\phi_s$ we will also need approximations to $\cosh( \frac{\gv{\Sigma} \cdot \gv{\rapidity} }{2})$ and $\sinh( \frac{\gv{\Sigma} \cdot \gv{\rapidity} }{2})$.  We only need the rapidity to the leading order: $\gv{\rapidity} \approx \v{v} \approx \frac{\v{p} }{m}$. 

\[
	\cosh( \frac{\gv{\Sigma} \cdot \gv{\rapidity} }{2}) 
		\approx 1 + \frac{1}{2}\left( \frac{\gv{\Sigma} \cdot \v{p} }{2m} \right)^2
\]
\[
	\sinh( \frac{\gv{\Sigma} \cdot \gv{\rapidity} }{2}) 
		\approx   \frac{\gv{\Sigma} \cdot \v{p} }{2m}
\]
The the two bispinor components are
%Non relativistic expression for \phi and \chi
\begin{align}
\phi 
	&\approx  \left(  1 + \left[ \frac{1}{2}\frac{\gv{\Sigma} \cdot \v{p} }{2m} \right]^2 \right) \xi_0 \notag \\
	&\approx  \left(  1 + \frac{(\gv{\Sigma} \cdot \v{p})^2 }{8m^2} - \frac{\v{p}^2}{4m} \right ) \phis	 \label{eq:nrPhi} \\
 \chi
 	&\approx	\frac{\gv{\Sigma} \cdot \v{p} }{2m} \xi_0 \notag \\
 	&\approx	\frac{\gv{\Sigma} \cdot \v{p} }{2m} \phis  \label{eq:nrChi}
\end{align}




%TODO remember to hunt out all the \phi where I mean \rapidity
\subsection{Bilinears in terms of nonrelativistic theory}
The next step is to express the relativistic bilinears, built out of the bispinors $\Psi$, in terms of the Schrodinger like wave functions.

We have above written the bispinors in terms of $\phi_s$, so we can use those identities to express the bilinears in the same manner.


\subsubsection{Scalar bilinear}
\beqa
\Psibar^\dagger(p') \Psi(p)
	&=&	\phi^\dagger \phi - \chi^\dagger \chi	\\
	&=&	\phi_s^\dagger \left[1 + \frac{(\gv{\Sigma} \cdot \v{p'})^2 }{8m^2}  - \frac{\v{p'}^2}{4m^2} \right ]
			 \left[1 + \frac{(\gv{\Sigma} \cdot \v{p})^2 }{8m^2}  - \frac{\v{p}^2}{4m^2} \right ] \phi_s
		- \phi_s^\dagger \left[
			\frac{ ( \gv{\Sigma} \cdot \v{p'}) (\gv{\Sigma} \cdot \v{p}) }{4m^2}
		\right ] \phi_s	\\
	&=&	\phi_s^\dagger \left (
			1 - \frac{ \v{p}^2 + \v{p'}^2 }{4m^2}
			+ \frac{1}{8m^2} \left \{
				( \gv{\Sigma} \cdot \v{p'})^2 +  (\gv{\Sigma} \cdot \v{p})^2 
				 - 2 ( \gv{\Sigma} \cdot \v{p'}) (\gv{\Sigma} \cdot \v{p})
			\right \}
	\right ) \phi_s	\\
	&=& \phi_s^\dagger \left (
			1 - \frac{ \v{p}^2 + \v{p'}^2 }{4m^2}
			+ \frac{1}{8m^2} \left \{
				[ \gv{\Sigma} \cdot \v{p},  \gv{\Sigma} \cdot \v{q}]  + ( \gv{\Sigma} \cdot \v{q})^2 
			\right \}
	\right ) \phi_s	\\
	&=& \phi_s^\dagger \left (
			1 - \frac{ \v{p}^2 + \v{p'}^2 }{4m^2}
			+ \frac{1}{8m^2} \left \{
				[ 4 i \epsilon_{ijk} p_i q_j S_k  + ( \gv{\Sigma} \cdot \v{q})^2 
			\right \}
	\right ) \phi_s
\eeqa
%TODO at some point need to translate \Sigma functions into spin functions in NR theory, but where?

\subsubsection{Tensor $ij$ component}



In calculating the nonrelativistic limit of the antisymmetric tensor bilinear, we will treat the $0i$ and the $ij$ components seperately.  First let us consider $\Psibar \Sigma_{ij} \Psi$.

\beqa
	\Psibar \TensBi_{ij} \Psi 
		&=& \Psibar (-2\epsilon_{ijk} S_k) \Psi	\\
		&=&	-2i\epsilon_{ijk} ( \phi^\dagger S_k \phi - \chi^\dagger S_k \chi)	\\
		&=&	-2i\epsilon_{ijk} \Big( \phis^\dagger \left[ 1 + \frac{( \gv{\Sigma} \cdot \v{p'})^2}{8m^2}  -\frac{\v{p'}^2}{4m^2} \right] S_k \left[ 1 + \frac{( \gv{\Sigma} \cdot \v{p})^2}{8m^2} -\frac{\v{p}^2}{4m^2}\right ] \phis - \phis^\dagger \frac{ ( \gv{\Sigma} \cdot \v{p'})S_k ( \gv{\Sigma} \cdot \v{p })}{4m^2} \phis \Big )	\\
		&=&	-2i\epsilon_{ijk} \phis^\dagger \left \{
				S_k \left( 1 - \frac{ \v{p}^2 + \v{p'}^2}{4m^2}  \right )
				+ \frac{1}{8m^2} \Big[ ( \gv{\Sigma} \cdot \v{p'})^2 S_k + S_k ( \gv{\Sigma} \cdot \v{p})^2 - 2 ( \gv{\Sigma} \cdot \v{p'})S_k ( \gv{\Sigma} \cdot \v{p}) \Big ]
			\right \} \phis
\eeqa

We want to write the terms in square brackets explicitly in terms of $\v{p}$ and $\v{q}$.
\beqa
( \gv{\Sigma} \cdot \v{p'})^2 S_k + S_k ( \gv{\Sigma} \cdot \v{p})^2 - 2 ( \gv{\Sigma} \cdot \v{p'})S_k ( \gv{\Sigma} \cdot \v{p}) 
	&=& ( \gv{\Sigma} \cdot \v{p})^2 S_k + S_k ( \gv{\Sigma} \cdot \v{p}) -2 ( \gv{\Sigma} \cdot \v{p}) S_k ( \gv{\Sigma} \cdot \v{p})
	\\ &&	+ \{ ( \gv{\Sigma} \cdot \v{p}) ( \gv{\Sigma} \cdot \v{q}) + ( \gv{\Sigma} \cdot \v{q}) ( \gv{\Sigma} \cdot \v{p}) \}S_k
	\\ &&	- 2 ( \gv{\Sigma} \cdot \v{q}) S_k ( \gv{\Sigma} \cdot \v{p})
		+ ( \gv{\Sigma} \cdot \v{q})^2 S_k
\eeqa

We can express many of these terms as commutators
\[
	=	\gv{\Sigma} \cdot \v{p} [ \gv{\Sigma} \cdot \v{p}, S_k] + [S_k, \gv{\Sigma} \cdot \v{p}] \gv{\Sigma} \cdot \v{p}
		+ 2 \gv{\Sigma} \cdot \v{q} [ \gv{\Sigma} \cdot \v{p}, S_k] - [\gv{\Sigma} \cdot \v{q}, S_k] \gv{\Sigma} \cdot \v{p}
		+ (\gv{\Sigma} \cdot \v{q})^2 S_k
\]

\[
	= i\epsilon_{ijk} p_j \{ (\gv{\Sigma} \cdot \v{p})\Sigma_i - \Sigma_i (\gv{\Sigma} \cdot \v{p}) \}
		+ 2 i\epsilon_{ijk} \{ (\gv{\Sigma} \cdot \v{q}) \Sigma_i p_j -  \Sigma_i (\gv{\Sigma} \cdot \v{p}) q_j ) \}
		+ (\gv{\Sigma} \cdot \v{q})^2 S_k
\]

\[
	= i\epsilon_{ijk} p_j [ (\gv{\Sigma} \cdot \v{p}), \Sigma_i  ]
		+ 2 i\epsilon_{ijk} \{ (\gv{\Sigma} \cdot \v{q}) \Sigma_i p_j -  \Sigma_i (\gv{\Sigma} \cdot \v{p}) q_j ) \}
		+ (\gv{\Sigma} \cdot \v{q})^2 S_k
\]


\[
	= 4( \v{p}^2 S_k - (\v{S} \cdot \v{p}) p_k ) 
		+ 2 i\epsilon_{ijk} \{ (\gv{\Sigma} \cdot \v{q}) \Sigma_i p_j -  \Sigma_i (\gv{\Sigma} \cdot \v{p}) q_j ) \}
		+ (\gv{\Sigma} \cdot \v{q})^2 S_k
\]

Thus the whole bilinear is

\beq
-2i\epsilon_{ijk} \phis^\dagger \left \{
				S_k \left( 1 - \frac{ \v{p}^2 + \v{p'}^2}{4m^2}  \right )
				+ \frac{1}{8m^2} \Big[ 
				4( \v{p}^2 S_k - (\v{S} \cdot \v{p}) p_k ) 
				+ 2 i\epsilon_{\ell m k} \{ (\gv{\Sigma} \cdot \v{q}) \Sigma_\ell p_m -  \Sigma_\ell (\gv{\Sigma} \cdot \v{p}) q_m ) \}
				+ (\gv{\Sigma} \cdot \v{q})^2 S_k
			 \Big ]
			\right \} \phis
\eeq

\subsubsection{Tensor $\Sigma_{0i}$ component}

We calculate $\Psibar \Sigma_{0i} \Psi$.

\[
	\Psibar \Sigma_{0i} \Psi = \Psibar \begin{pmatrix} 0 & \Sigma_i \\ \Sigma_i & 0 \end{pmatrix} \Psi
\]

\[
	=	\phi^\dagger \Sigma_i \chi - \chi^\dagger \Sigma_i \phi
\]

We'll only need $\phi$ and $\chi$ to first order here.
\[
	= \phis^\dagger \left( \frac{\Sigma_i \Sigma_j p_j - \Sigma_j \Sigma_i p'j}{2m} \right ) \phis
\]
Using $p'=p+q$ the terms involving only $p$ can be simplified using the commutator of $\Sigma$ matrices.
\[
	=\phis^\dagger \left( \frac{ 4i\epsilon_{ijk} p_j S_k - \Sigma_j \Sigma_i q_j}{2m} \right )\phi
\]



%%%%%%%  Calculate the Current
%%%%%%%%%%%%%%%%%%%%%%%%%%%%%%%%%
\subsection{Current in terms of nonrelativistic wave functions}

%add ref to equation
We derived the four-current \eqref{eq:khr_current} above; in nonrelativistic notation it is: 
\beq
	j_0 =  F_e \frac{p_0 + p'_0}{2m} \Psibar^\dagger \Psi -  F_m \frac{q_j}{2m} \Psibar^\dagger \TensBi^{0j} \Psi
\eeq

\beq
	j_i =  F_e \frac{p_i + p'_i}{2m} \Psibar^\dagger \Psi -   F_m \frac{q_j}{2m} \Psibar^\dagger \TensBi^{ij} \Psi 
			+F_m \frac{q_0}{2m} \Psibar^\dagger \TensBi^{i0} \Psi
\eeq

We have expressions for the bilinears in terms of the nonrelativistic wave functions $\phis$, so it is fairly straight forward to apply them here.  The calculation of $j_0$ is straightforward:
\beqa
F_e \frac{p_0 + p'_0}{2m} \Psibar^\dagger \Psi 
	  &=& F_e \left(1 + \frac{\v{p}^2 + \v{p'}^2}{4m^2} \right)  \phis^\dagger \left (
			1 - \frac{ \v{p}^2 + \v{p'}^2 }{4m^2}
			+ \frac{1}{8m^2} \left \{
				 4 i \epsilon_{ijk} p_i q_j S_k  + ( \gv{\Sigma} \cdot \v{q})^2 
			\right \}
	\right ) \phis	\\
	&\approx& 	F_e   \phis^\dagger \left (
					1 + \frac{1}{8m^2} \left \{ 4i \v{S} \cdot \v{p} \times \v{q}  + ( \gv{\Sigma} \cdot \v{q})^2 \right \}
				\right ) \phis	\\
F_m \frac{q_j}{2m} \Psibar^\dagger \TensBi^{0j} \Psi
	&=& F_m \frac{q_i}{2m}\phis^\dagger \left( \frac{ 4i\epsilon_{ijk} p_j S_k - \Sigma_j \Sigma_i q_j}{2m} \right )\phis	\\
	&=&  F_m \phis^\dagger \left( \frac{ 4i \v{S} \cdot \v{q} \times \v{p} -  (\gv{\Sigma} \cdot \v{q})^2 }{4m^2} \right )\phis	\\
\eeqa

It turns out that both terms here have the same form, so combining them we get
\beq \label{eq:nrJ0}
j_0 =  	 \phis^\dagger \left (
			F_e + \frac{F_e + 2F_m}{8m^2} \left \{ 4i \v{S} \cdot \v{p} \times \v{q}  + ( \gv{\Sigma} \cdot \v{q})^2  \right \}
		\right ) \phis	\\
\eeq


%Justify dropping derivatives of magnetic field a bit better
To calculate $j_i$ we want to first simplify things by considering the constraints of our particular problem.  The term with $\Sigma_ij$ can be simplified by dropping terms with more than one power of $q$; these will turn into derivatives of the magnetic field, and our problem concerns only a constant field.  Further, we need only calculate elastic scattering, and so $q_0=0$.  With those simplifications
\[
\Psibar \Sigma_{ij} \Psi \approx
		-2i\epsilon_{ijk} \phis^\dagger \left \{
			S_k \left( 1 - \frac{ \v{p}^2 + \v{p'}^2}{4m^2}  \right )
			+ \frac{\v{p}^2 S_k - (\v{S} \cdot \v{p}) p_k}{2m^2}  
		\right \} \phis
\]


\beqa
F_e \frac{p_i + p'_i}{2m} \Psibar^\dagger \Psi
	&=&			F_e \frac{p_i + p'_i}{2m}  \phis^\dagger \left (
					1 - \frac{ \v{p}^2 + \v{p'}^2 }{4m^2}
					+ \frac{1}{8m^2} \left \{  4 i \epsilon_{\ell jk} p_\ell q_j S_k  + ( \gv{\Sigma} \cdot \v{q})^2	\right \}
				\right ) \phis	\\
	&\approx& 	F_e \frac{p_i + p'_i}{2m}  \phis^\dagger \left (
					1 + \frac{1}{8m^2} \left \{ 4 i \epsilon_{\ell jk} p_\ell q_j S_k  \right \}
				\right ) \phis	\\
F_m \frac{q_j}{2m} \Psibar^\dagger \TensBi^{ij} \Psi
	&=& 		- F_m\frac{ i\epsilon_{ijk} q_j}{m} \phis^\dagger \left \{
					S_k \left( 1 - \frac{ \v{p}^2 + \v{p'}^2}{4m^2}  \right )
					+ \frac{\v{p}^2 S_k - (\v{S} \cdot \v{p}) p_k}{2m^2}  
				\right \} \phis
\eeqa

So the full spatial part of the current is
\beq \label{eq:nrJi}
j_i	=	\phis^\dagger \Bigg \{
			F_e \frac{p_i + p'_i}{2m} \left (
				1 + \frac{ i \epsilon_{\ell jk} p_\ell q_j S_k   }{2m^2}  \right)
			+ F_m   \frac{i\epsilon_{ijk} q_j}{m} \left( 
				S_k \left( 1 - \frac{ \v{p}^2 + \v{p'}^2}{4m^2}  \right )
				+ \frac{\v{p}^2 S_k - (\v{S} \cdot \v{p}) p_k}{2m^2} \right)	
		\Bigg \} \phis
\eeq

\subsection{Scattering off external field}
To compare to the NRQED Lagrangian, we want to calculate scattering off an external field for an arbitrary spin particle.  We already have the current, so the scattering is just
\[
	M = e j_\mu A^\mu = e j_0 A_0 - e \v{j} \cdot \v{A}
\]
Above we have expressions for both $j_0$ \eqref{eq:nrJ0} and $\v{j}$ \eqref{eq:nrJi}.  So we can write down the parts of the amplitude directly:
\beq
\begin{split}
	e j_0 A_0 = 
		& eA_0 \phis^\dagger \left (
			F_e + \frac{F_e + 2F_m}{8m^2} \left \{ 4i \v{S} \cdot \v{p} \times \v{q}  + ( \gv{\Sigma} \cdot \v{q})^2  \right \}
		\right ) \phis	\\
   e\v{j} \cdot \v{A} =
		& A_i \phis^\dagger \Bigg \{
			F_e \frac{p_i + p'_i}{2m} \left (
				1 + \frac{ i \epsilon_{\ell jk} p_\ell q_j S_k   }{2m^2}  \right)
			+ F_m   \frac{i\epsilon_{ijk} q_j}{m} \left( 
				S_k \left( 1 - \frac{ \v{p}^2 + \v{p'}^2}{4m^2}  \right )
				+ \frac{\v{p}^2 S_k - (\v{S} \cdot \v{p}) p_k}{2m^2} \right)	
		\Bigg \} \phis
\end{split}
\eeq  


As much as possible we want to express the result in terms of gauge invariant quantities $\v{B}$ and $\v{E}$.  We write the relations between these fields and $A_\mu$ in position space and the equivalent equation in momentum space.
%FIXME add reference to gauge
\beq
\begin{split}
	\v{B} &= \grad \times \v{A}	\notag \to i\v{q} \times \v{A} \\
	\v{E} &= -\grad A_0	 \to -i\v{q} A_0 		\notag
\end{split}
\eeq

There is one term above that can only be put into gauge-invariant form by considering the kinematic constraints of elastic scattering.  If the scattering is elastic, we have $\v{q} \cdot (\v{p'} + \v{p}) = \v{p'}^2 - \v{p}^2=0$.  We can use this identity on the term $q_j (p'_i + p_i) A_i$ as follows:
\beqa
	\epsilon_{ijk} B_k &=&
		 \partial_i A_j - \partial_j A_i = i(q_i A_j - q_j A_i)	\\
	(p_i + p'_i) \epsilon B_k 
		&=& i (p_i + p'_i) (q_i A_j - q_j A_i)		\\
		&=& - i (p_i + p'_i) q_j A_i	
\eeqa
So we have the identity
\beq \label{eq:ppqAid}
	i (p_i + p'_i) q_j A_i = - \epsilon_{ijk}B_k (p_i + p'_i)
\eeq 

Now we can write each term involving $q$ in terms of position space quantities.
%TODO maybe write quad term more explicitly in terms of quad moment and term with trace
\beqa
i \v{S} \cdot \v{p} \times \v{q} A_0 
		&=&	-\v{S} \cdot \v{p} \times \v{E}	\\
( \gv{\Sigma} \cdot \v{q})^2 A_0	
		&=&		\Sigma_i \Sigma_j q_i q_j A_0 		\\
		&=&		 \Sigma_i \Sigma_j \partial_i E_j	\\
i\epsilon_{ijk} A_i q_j	
		&=&	-i (\v{q} \times \v{A})_k 			\\
		&=&= - B_k	\\
A_i (p_i + p'_i)  i \epsilon_{\ell jk} p_\ell q_j S_k  
		&=&	\epsilon_{\ell jk}p_\ell S_k  i(p_i + p'_i) q_j A_i		\\
		&=&	- \epsilon_{\ell jk}p_\ell S_k \{ \epsilon_{ijm}B_m (p_i + p'_i) \}			\\
		&=& -(\delta_{\ell i} \delta_{km} - \delta{\ell m} \delta_{ik})p_\ell S_k \{ \epsilon_{ijm}B_m (p_i + p'_i) \}	\\
		&=& 2\{ (\v{B} \cdot \v{p})  (\v{S} \cdot \v{p}) - (\v{B} \cdot \v{S}) \v{p}^2  \}  
\eeqa

Using these
\beqB
	ej_0 A_0 = e\phis^\dagger \left\{
					A_0 + \frac{1 - 2F_2}{8m^2}\left( 4 \v{S} \cdot \v{E} \times \v{p} + \Sigma_i \Sigma_j \partial_i E_j \right)
				\right \}
\eeqB

\beqB
	e \v{j} \cdot \v{A}	= e \phis^\dagger \left \{
			\frac{ \v{p} \cdot \v{A} }{m} + \frac{(\v{B} \cdot \v{p})  (\v{S} \cdot \v{p}) - (\v{B} \cdot \v{S}) \v{p}^2 }{m^2} 
			- F_m \left ( \frac{ \v{S} \cdot \v{B} }{m} \left\{ 1 - \frac{\v{p}^2 + \v{p'}^2}{4m^2} \right \} + \frac{(\v{B} \cdot \v{p})  (\v{S} \cdot \v{p}) - (\v{B} \cdot \v{S}) \v{p}^2 }{2m^2} \right ) \right \} \phi
\eeqB

\beqB
	=e\phis^\dagger \left\{
		 \frac{ \v{p} \cdot \v{A} }{m} + [1 -2F_m] \frac{ (\v{B} \cdot \v{p} )(\v{S} \cdot \v{p})}{m^2} 
				- \v{S} \cdot \v{B} \frac{ \v{p}^2 }{m^2} - \frac{F_m}{m} \v{S} \cdot \v{B} \right \}
\eeqB
From this we can see that our $F_m$ is actually $g/2$, so in such terms

\beqB	
	ej_0 A_0 = e\phis^\dagger \left\{
					A_0 - \frac{g-1}{2m^2}\left( \v{S} \cdot \v{E} \times \v{p} + \frac{1}{4}\Sigma_i \Sigma_j \partial_i E_j \right)
				\right \}
\eeqB

\beqB
	e \v{j} \cdot \v{A} = e\phis^\dagger \left\{
		 \frac{ \v{p} \cdot \v{A} }{m} - [g-1] \frac{ (\v{B} \cdot \v{p} )(\v{S} \cdot \v{p})}{m^2} 
				- \v{S} \cdot \v{B} \frac{ \v{p}^2 }{m^2} - \frac{g}{2m} \v{S} \cdot \v{B} \right \}
\eeqB



So the entire scattering process is

\beq \label{eq:fullScatter}
	e \phis^\dagger  \left \{
		A_0 - \frac{g-1}{2m^2}\left( \v{S} \cdot \v{E} \times \v{p} + \frac{1}{4}\Sigma_i \Sigma_j \partial_i E_j \right)
		 \frac{ \v{p} \cdot \v{A} }{m} - [g-1] \frac{ (\v{B} \cdot \v{p} )(\v{S} \cdot \v{p})}{m^2} 
				- \v{S} \cdot \v{B} \frac{ \v{p}^2 }{m^2} - \frac{g}{2m} \v{S} \cdot \v{B} 
	\right \} \phis
\eeq


\subsection{Fixing the nonrelativistic coefficients}

Having calculated the same process in both the relativistic theory and in the NRQED effective theory, the two amplitudes can be compared, thus fixing the coefficients of NRQED.

The NRQED amplitude \eqref{eq:nrqedScatter} is
\beq
\begin{split}
	iM =
		ie\phi^\dagger \Bigg( - A_0 +   \frac{ \v{A} \cdot \v{p} }{m} - \frac{  (\v{A} \cdot \v{p}) \v{p}^2   }{2m^3} 
		+ c_F  \frac{\v{S} \smalldot \v{B}} {2m}   	
		+ c_D \frac{ ( \partial_i E_i ) }{8m^2}	
		+ c_Q \frac{ Q_{ij} ( \partial_i E_j ) }{8m^2}	
	\\	+ c^{1}_S \frac{  \v{E} \times \v{p} }{4m^2}
		- (c_{W_1} -c_{W_2}) \frac{   (\v{S} \smalldot \v{B} ) \v{p}^2  }{4m^3}	
		-  c_{p'p} \frac{   (\v{S} \smalldot \v{p}) (\v{B} \smalldot \v{p})  }{4m^3} \Bigg )\phi
\end{split}
\eeq



While the relativistic amplitude was
\beq
\begin{split}
iM_{REL} = -ie \phi^\dagger \Big (
		 A_0  - \frac{\v{p}\cdot \v{A} }{m} + \frac{\v{p}\cdot \v{A} \v{p}^2}{2m^3}
		- \frac{g-1}{2m^3}\{ \grad \cdot \v{E} -  \v{S} \cdot \v{p} \times \v{E} - S_i S_j \grad_i E_j \}
		\\ - g\frac{1}{2m} \v{S} \cdot \v{B}
		+ \v{S} \cdot \v{B} \frac{\v{p}^2}{2m^3}
		+ \frac{g-2}{4m^3}(\v{S} \cdot \v{p} )( \v{B} \cdot \v{p})
	\Big ) \phi
\end{split}
\eeq


The term $\grad \cdot \v{E}  - S_i S_j \grad_i E_j$ should be rewritten using the quadrupole moment tensor $Q_{ij} = \frac{1}{2} ( S_i S_j + S_j S_i - \frac{2}{3}\v{S}^2 )$.

Remembering that $\nabla_i E_j$ is actually symmetric under exchange of $i$ and $j$, 
\[
	S_i S_j \nabla_i E_j = \frac{1}{2} (S_i S_j + S_j S_i) = (Q_{ij} + \frac{1}{3} \v{S}^2 \delta_{ij}) \nabla_i E_j
\]
\[
	= Q_{ij} \nabla_i E_j + \frac{2}{3} \grad \cdot \v{E}
\]
Written in that form, 
\beq
\begin{split}
iM_{REL} = -ie \phi^\dagger \Big (
		 A_0  - \frac{\v{p}\cdot \v{A} }{m} + \frac{\v{p}\cdot \v{A} \v{p}^2}{2m^3}
		- \frac{g-1}{2m^3}\{ \frac{1}{3}\grad \cdot \v{E} -  \v{S} \cdot \v{p} \times \v{E} - Q_{ij} \grad_i E_j \}
		\\ - g\frac{1}{2m} \v{S} \cdot \v{B}
		+ \v{S} \cdot \v{B} \frac{\v{p}^2}{2m^3}
		+ \frac{g-2}{4m^3}(\v{S} \cdot \v{p} )( \v{B} \cdot \v{p})
	\Big ) \phi
\end{split}
\eeq


Comparing the two, the coefficients are:
\beqa
	c_F &=& g \\
	c_D &=&	\frac{4(g-1)}{3}	\\
	c_Q &=&	-4(g-1)	\\
	c^1_S &=& 2 (g-1)	\\
	(c_{W_1} - c_{W_2}) &=&	2	\\
	c_{p'p}	&=& (g-2)		\\
\eeqa
