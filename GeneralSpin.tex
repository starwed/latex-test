
\chapter{Particles of arbitrary spin}

%name? 
Above, two particular relativistic theories have been considered: those sectors of QED involving spin half and spin one particles.  A nonrelativistic theory for each was derived in two ways, each starting from the relativistic Lagrangian; once by solving the equations of motion, the other by comparing scattering amplitudes so as to determine NRQED coefficients.

Although the two nonrelativistic Lagrangians do differ, it turns out that all terms contributing to the bound $g$-factor coincide, at least at the second order in $\alpha^2$.  This offers the hope that this remains true for all charged particles, no matter their spin.

The obvious obstacle is that in the previous chapters a specific, known relativistic Lagrangian was the starting point.  However, a key point is that, in the NRQED approach, only the one-photon vertex for the charged particle was necessary to fix all relevant NRQED coefficients.  If this holds true for the case of general spin, then only that portion of a relativistic Lagrangian representing the one photon interaction need be considered.  And there are enough constraints on this interaction to render the problem quite tractable.

First, the form of the nonrelativistic Lagrangian and fields will be constructed for general spin.  Then the relativistic fields will be treated similarly.  Finally, by considering physical results from both theories, the coefficients will be fixed for the general spin Lagrangian.



\section{The nonrelativistic Lagrangian}
\subsection{Spinors for nonrelativistic theory}
The first question is how to represent the fields in a nonrelativistic theory.  A single particle of spin $s$ has $2s+1$ degrees of freedom.  A regular spinor of rank $2s$ has the correct transformation properties under rotation, and by considering only such spinors which are symmetric in all indices, the necessary number of degrees of freedom is obtained.


\subsection{Building the NRQED Lagrangian}
%TODO add refs back to them
Previously, NRQED Lagrangians were developed for spin half and spin one.  They included all the terms that could arise at the required order for such theories.  As discussed in developing the Lagrangian for spin one, new terms will arise because the increasing degrees of spin freedom allow a larger set of spin operators.  So to describe the NREQED Lagrangian for general spin, a complete description of spin space operators are needed.


%%%%%%%%%%%%%%%%%%%%%%%%%%%%%
%   Spin space operators input
%%%%%%%%%%%%%%%%%%%%%%%%%%%%%
\subsection{Properties of spin operators}
%TODO in the context of a field theory, is talking about the state-space the correct way of thinking about it?
In formulating NRQED for general spin particles, we need to consider all the possible operators might show up in the Lagrangian.  The state-space of a spin-$s$ particle is the direct product of its spin-state and all the other state information.  Because the spaces are orthogonal, we can treat separately operators in the two spaces.  The operators and fields which exist in position space are the same for a particle of any spin, but unsurpsingly the operators allowed in spin space do depend upon the spin of the particle.  As the spin of the particle is increased, and thus its spin degrees of freedom rise, there are more ways to mix these components, and thus a greater number of spin operators to consider.

For a particular representation, we can always write a bilinear as the spin operators and other operators acting between two spinors:
\beq
	\Psi^\dagger \mathcal{O}_S \mathcal{O}_X \Psi
\eeq

The two types of operators will always commute, since they act on orthogonal spaces, so it doesn't matter what order they're written in.  All such bilinears must be Galilean invariant, but individual operators might not be.  The non-spin operators we consider, such as $\v{D}$, $\v{B}$ or contractions with the tensor $\epsilon_{ijk}$ are already all written as 3-vectors or (in combination) as higher rank tensors.  Therefore, it will be most convenient to write spin-operators with well defined properties under Galilean transformations.  In that way, writing Galilean-invariant combinations of the two types of operators is done just by contracting indices. 

Even though the number of spin operators does depend upon the spin of the particle, it is still possible to proceed in such a way that the same notation may be used no matter the spin.  There are a few requirements:
\begin{itemize}
  \item We write all high spin operators in terms of combinations of $S_i$, since these have universal properties regardless of the representation they are written in.
  \item If an operator exists and is non-zero in the representation of spin-$s$, it also exists in spin-$s+1$
  \item All operators introduced to account for the additional degrees of freedom in higher spin representations vanish when written in a lower spin theory.  (As an example of the last point, the operator $S_i S_j + S_j S_i - \delta_{ij} S^2$ is needed to account for the degrees of freedom in a spin-1 theory, but vanishes in spin-1/2.)
\end{itemize}
If these requirements are met a consistent spin-agnostic notation can be adopted.  Now we attempt to construct operators that meet these conditions.

The spinors $\Psi$ are written with $2s+1$ independent components.  The spin operators will be isomorphic to matrices acting on these components, which for a spin-$s$ particle would be $(2s+1) \times (2s+1)$ matrices.  The combined operator $\mathcal{O}_S \mathcal{O}_X$ must be Hermitian, but without loss of generality we can require any $\mathcal{O}_S$, $\mathcal{O}_X$ to be Hermitian separately.  So there is the additional constraint that these matrices be Hermitian, and this means a total of $(2s+1)^2$ degrees of freedom.
%TODO: could include derivation of degrees of freedom?


For spin-$0$ there is only one component to the spinor, so the only possible operator is equivalent to the identity.

For spin-$1/2$ we have, in addition to the identity, the spin matrices $s_i = \frac{1}{2} \sigma_i$.  This is a set of four independent matrices, and since the space has $(2s+1)^2 = 4$ degrees of freedom, exactly spans the space of all spin-operators.  If we try to construct terms which are bilinear in spin matrices, they just reduce through the identity $\sigma_i \sigma_j = \delta_{ij} + \epsilon_{ijk}\sigma_k$, which we can already construct through combinations of the four operators we already have.  Since those four operators form a basis for the space, independent bilinears were forbidden even without an explicit form for the equation.


What about spin-$1$?  We need 9 independent operators to span the space.  All the operators that exist in spin-$1/2$ will work here as well, though the spin matrices will have a different representation.  That leaves 5 operators to construct.  It is natural to try to construct these from bilinear combinations of spin matrices.  Naively $S_i S_j$ would itself be 9 independent structures, but clearly some of these are expressible in terms of the lower order operators.  (By the order of a spin operator we mean its greatest degree in $S_i$ )  %TODO multipole moment language better?

Regardless of their representation, the spin operators always fulfill certain identities based on their Lie group.  Namely
\beq
	S_i S_j \delta_{ij} \sim I, \; [S_i, S_j] = \epsilon_{ijk} S_k
\eeq
and it is these identities which allow certain combinations of $S_i S_j$ to be related to lower order operators.

If instead of general spin bilinears we consider only combinations which are
\begin{itemize}
  \item Symmetric in $i$, $j$ 
  \item Traceless
\end{itemize}
then such a structure will be independent of the set of operators $\{I, S_1, S_2, S_3\}$.  Because it is symmetric no combination may be related using the commutator, and because it is traceless there is no combination that reduces due to the other identity.  %TODO Name of id involving Killing operator \delta_{ij}?

This conditions form a set of 4 constraints, so from the original 9 degrees of freedom possessed by combinations of $S_i S_j$ are left only 5.  Together with the 4 lower order operators this is exactly enough to span the space.

We can explicitly write this symmetric, traceless structure as
\beq
	 S_i S_j + S_j S_i - \frac{2}{3} \delta_{ij} S^2
\eeq

Having explored how the procedure works for spin-1, move on to consider the general spin case.   The idea is to proceed inductively using the same rough attack as for the case of spin-$1$.  In addition to all the ``lower order'' operators which were used for lower spin representations introduce new operators which are of higher degree in the spin matrices and guaranteed to be independent of the lower spin operators.

So suppose that for a spin-$s-1$ particle we have a set of operators written as $\bar{S}^0$, $\bar{S}^1$ $\ldots \bar{S}^{(s-1)}$, where a structure $\bar{S}^n$ carries $n$ Galilean indices and is symmetric and traceless between any pair of indices, that is:
\beq
	\bar{S}^n_{..i..j..} = \bar{S}^n_{..j..i..}, \; \delta_{ij} \bar{S}^n_{..i..j..}=0
\eeq 
(From above, $\bar{S}^0=I$, $\bar{S}^1_i = S_i$, and $\bar{S}^2 = S_i S_j + S_j S_i - \frac{2}{3}\delta_{ij} S^2$.) 

The objects $\bar{S}^n$ are built as follows: start with all combinations involving the product of exactly $n$ spin matrices.  (There are $3^n$ such structures.)  Form them into combinations which are symmetric in all indices.  Each index has three possible values, so we can label each structure by how many indices are equal to 1 and 2.  If $a$ is the number of indices equal to $1$, and $b$ the number of indices equal to $2$, then for a given $a$ there are $n+1-a$ possible choices for $b$.  The total number of symmetric structures is then
\beq
	\sum^n_{a=0} (n-a+1) = \frac{1}{2} (n+1) (n+2)
\eeq
We want to apply the additional constraint that the $\bar{S}^n$ be traceless in all indices.  This will involve subtracting all the lower order structures which result when the trace of the completely symmetric combinations is taken.  %TODO explicate {\it exactly} how this is accomplished.

It introduces an additional constraint on $\Sb^n$ for each pair of indices, and there are $n (n-1)/2$ distinct pairs of indices.  The total degrees of freedom left are

\beq
	\frac{1}{2} (n+1) (n+2) - \frac{1}{2} n(n-1) 
		= \frac{1}{2}\left( n^2+3n +2 - n^2 +n\right )
		= 2n+1
\eeq

%TODO expand upon going from spin s particle to spin s-1
In combination with the lower order spin operators, this is exactly the number of independent operators we need to span the space.  Combined with the lower order operators this is a complete basis, so we know we haven't missed any terms.  Because they are constructed to be independent from all the lower order operators, it must necessarily be true that they will vanish in lower spin representations.

Using this notation we can write down terms in the Lagrangian that are valid for particles of any spin.  By writing all spin operators in terms of $S_i$ they are representation agnostic, and by construction they will vanish for low spin particles where they do not ``fit''.


\subsection{New terms arising at higher spin}
With the general set of spin operators in hand, what new terms can occur in the Lagrangian involving them?  Although a large number of such operators exist for high spin, only a small number need be considered for the NRQED Lagrangian.  That is because only position space operators up to a certain order are considered.

The different combinations of position space operators were catalogued in describing the spin half Lagrangian.  The first new spin structure to consider is $\Sb_{ijk}$, which has three indices.  The only allowed set of position space operators that can be contracted with this operator is $\v{B}, \v{D}, \v{D}$.  Considering Hermitian combinations, two new terms can arise:
\beq \label{eq:Sg:Lnewterms}
  c_{T_1} \frac{ e\Sb_{ijk} D_i B_j D_k }{8m^3}  + c_{T_2} \frac{ e\Sb_{ijk} (D_i D_j B_k + B_i D_j D_i) }{8m^2} 
\eeq   

Only one combination of position space operators existed at the needed order, $D_i D_j D_k D_\ell$.  This cannot be coupled to $\Sb_{ijk\ell}$ because it would spoil the kinetic term.  So the only new terms are those in  \eqref{eq:Sg:Lnewterms}.  Adding these to the Lagrangian \eqref{??} found for spin one, the result is

\beq \label{eq:Sg:nrLFull}
\begin{split}
\mathcal{L}_{NRQED} = & \fnrb \Bigg\{
		iD_0 +  \frac{\v{D}^2}{2m}  + 	\frac{\v{D}^4}{8m^2}
		 + c_F \frac{e}{m} \v{S} \cdot \v{B}
		+ c_D \frac{e (\v{D} \cdot \v{E} - \v{E} \cdot \v{D})}{8m^2} 
		+ c_Q \frac{eQ_{ij}(D_i E_j - E_i D_j)}{8m^2}
\\	& + c_S \frac{ i e \v{S} \cdot(\v{D} \times \v{E} - \v{E} \times \v{D}}{8m^2}
		+ c_{W1} \frac{ e \v{D}^2 \v{S} \cdot \v{B} + \v{S} \cdot \v{B} \v{D}^2 }{8m^3}
		- c_{W2} \frac{e D_i (\v{S} \cdot \v{B}) D_i}{4m^3}
\\	&		+c_{p'p} \frac{ e [ (\v{S}\cdot \v{D})(\v{B} \cdot \v{D}) + (\v{B} \cdot \v{D})(\v{S}\cdot \v{D})]}{8m^3}
 	+ c_{T_1} \frac{ e \bar{S}_{ijk} (D_i D_j B_k + B_k D_j D_i)}{8m^3}
		+ c_{T_2} \frac{ e \bar{S}_{ijk} D_i B_j D_k }{8m^3} 
		\Bigg \} \fnr
\end{split}
\eeq

\subsection{Scattering off an external field in NRQED}

To calculate scattering off an external field, that part of the Lagrangian involving only a single power of $A$ is needed.  The terms which arise for spin greater than one are the only modifications from \eqref{??}
\small
\beq
\begin{split}
\mathcal{L}_A =& \fnrb (  -eA_0 - ie  \frac{ \{\nabla_i, A_i \} }{2m} -ie \frac{ \{\grad^2, \{\nabla_i, A_i \}  \} }{8m^3} 
		+ c_F e \frac{\v{S} \smalldot \v{B}} {2m}   	
		+ c_D \frac{ e(\v{\grad} \smalldot \v{E} - \v{E} \smalldot \v{\grad} ) }{8m^2}	
	\\&	+ c_Q \frac{e Q_{ij} (\nabla_i E_j - E_i \nabla_j) }{8m^2}	
		+ c^{1}_S \frac{ ie \v{S} \smalldot ( \v{\grad} \times \v{E} - \v{E} \times \v{\grad} )}{8m^2}
		+ c_{W_1} \frac{ e [ \v{\grad}^2 (\v{S} \smalldot \v{B} ) + (\v{S} \smalldot \v{B} ) \v{\grad}^2] }{8m^3}	
	\\&	- c_{W_2} \frac{ e \nabla^i (\v{S} \smalldot \v{B} ) \nabla^i }{4m^3}
		+ c_{p'p} \frac{ e [ (\v{S} \smalldot \v{\grad}) (\v{B} \smalldot \v{\grad}) + (\v{B} \smalldot \v{\grad})(\v{S} \smalldot \v{\grad}) }{8m^3} 
	\\&	+ c_{T_1} \frac{ e \bar{S}_{ijk} (D_i D_j B_k + B_k D_j D_i)}{8m^3}
		+ c_{T_2} \frac{ e \bar{S}_{ijk} D_i B_j D_k }{8m^3} 
		\big )\fnr
\end{split}
\eeq
\normalsize

The calculation of the scattering amplitude goes just as before, with a couple of extra terms.  The result is
\beq 
\begin{split} \label{eq:Sg:nrqedScatter}
	iM_{\text{NR}} =
		ie\wnrb \Bigg(  -A_0 +  \frac{ \v{A} \cdot \v{p} }{m} - \frac{  (\v{A} \cdot \v{p}) \v{p}^2   }{2m^3} 
		+ c_F  \frac{\v{S} \smalldot \v{B}} {2m}   	
		+ c_D \frac{ ( \partial_i E_i ) }{8m^2}	
		+ c_Q \frac{ Q_{ij} ( \partial_i E_j ) }{8m^2}	
		+ c^{1}_S \frac{  \v{E} \times \v{p} }{4m^2}
	\\	- (c_{W_1} -c_{W_2}) \frac{   (\v{S} \smalldot \v{B} ) \v{p}^2  }{4m^3}	
		-  c_{p'p} \frac{   (\v{S} \smalldot \v{p}) (\v{B} \smalldot \v{p})  }{4m^3} 
		+ (c_{T_1} + c_{T_2} )\frac{ e \bar{S}_{ijk} D_i B_j D_k }{8m^3}
		\Bigg )\wnr 
\end{split}
\eeq

%Such a representation is equivalent to just the sum of several spin half representations.  If $\eta$ is a spinor of rank $2s$, then, the spin operator is
%\beq
%	\v{S}  = \frac{1}{2}\displaystyle \sum_{n=1}^{2s} \gv{\sigma}_n 	
%\eeq
%where $\gv{\sigma}_n$ is a Pauli spin matrix, acting on the $n$th index of $\eta$.


\section{Relativistic theory for general spin particles}


\subsection{Relativistic bispinors}
To work out a relativistic theory that describes particles of arbitrary spin, the first step will be to consider how the fields should be represented.  The tactic here will be to represent the spin state by an object that looks like a generalization of the Dirac bispinor.


It is easiest to start with the chiral basis, where the upper and lower components of the bispinor are objects of opposite helicity, each transforming as an object of spin $s$.

To that end define an object

\beq \label{eq:PsiDef}
\Psig  = \frac{1}{\sqrt{2}} \begin{pmatrix} \xi \\ \eta \end{pmatrix}
\eeq

the components of which will have the desired properties.  Each component should transform as a particle of spin $s$, but with opposite helicity.  Under reflection the upper and lower components transform into each other.

Spinors which are separately symmetric in dotted and undotted indices are representations of the proper Lorentz group.  If $\xi$ is an object with $p$ undotted and $q$ dotted indices
\beq
	\xi = \{ \xi^{\alpha_1 \ldots \alpha_p}_{\dot\beta_1 \ldots \dot\beta_q} \}
\eeq
Then this can be a representation of a particle of spin $s = (p+q)/2$.

There is some free choice in how to partition the dotted/undotted indices --- the same scheme won't work for all spin as long as both types of indices are present.  However, separately consistent choices can be made for integral and half-integral spin.  For integral spin let $p=q=s$, while for the half-integral case choose $p=s+\frac{1}{2}$, $q=s-\frac{1}{2}$.

$\xi$ and $\eta$ should transform as objects of opposite helicity.  Under reflection they will transform into each other.  So the choices made for $\xi$ then dictate that
\beq
	\eta = \{ \eta_{\dot \alpha_1 \ldots \dot \alpha_p}^{\beta_1 \ldots \beta_q} \}
\eeq




In the rest frame of the particle, there is no helicity.  Both objects will have clearly defined and identical properties under rotation.    The rest frame spinors are equivalent to rank $2s$ nonrelativistic spinors.  So the bispinor in the rest frame looks like
\beq \label{eq:PsiRest}
\Psig = \frac{1}{\sqrt{2}} \begin{pmatrix} \xi_0 \\ \xi_0 \end{pmatrix}
\eeq

where
\beq \label{eq:xi0def}
	\xi_0 = \{ (\xi_0)_{\alpha_1 \ldots \alpha_p \beta_1 \ldots \beta_q}  \}
\eeq
and is symmetric in all indices.

We can obtain the spinors in an arbitrary frame by boosting from the rest frame.  The upper and lower components we have defined to have opposite helicity, and so will act in opposite ways under boost by some rapidity $\rapidity$:
\beq \label{eq:xi0boosted}
	\xi = \exp{ \left (\frac{\v{\Sigma} \cdot \gv{\rapidity}}{2} \right ) } \xi_0,  
	\hspace{3em} 
	\eta = \exp{ \left(-\frac{\v{\Sigma} \cdot \gv{\rapidity}}{2} \right) } \xi_0
\eeq

%TODO check dotted/undotted transformations are correct
What form should the operator $\v{\Sigma}$ have?  Under an infinitesimal boost by a rapidity $\phi$, a spinor with a single undotted index is transformed as
\beqB
	\xi_\alpha \to \xi'_\alpha = \left(\delta_{\alpha \beta} + \frac{\gv{\rapidity}\cdot \gv{\sigma}_{\alpha \beta} }{2} \right) \xi_\beta 
\eeqB
while one with a dotted index will transform as
\beqB
\xi_{\dot\alpha} \to \xi'_{\dot\alpha} = \left(\delta_{\dot \alpha \dot \beta} - \frac{\gv{\rapidity}\cdot \gv{\sigma}_{\dot \alpha \dot\beta} }{2} \right) \xi_{\dot \beta}
\eeqB
The infinitesimal transformation of a higher spin object with the first $p$ indices undotted and the last $q$ dotted would then be
\beqB
	\xi \to \xi' = \left(1 
		+  \sum\limits_{a=0}^p \frac{\gv{\sigma}_a \cdot \gv{\rapidity} }{2}
		- \sum\limits_{a=p+1}^{p+q} \frac{\gv{\sigma}_a \cdot \gv{\rapidity} }{2}
	\right ) \xi 
\eeqB
where $a$ denotes which spinor index of $\xi$ is operated on.


So if $\gv{\Sigma}$ is defined as
\beq \label{eq:SigDef}
	\v{\Sigma} = \sum\limits_{a=0}^p \gv{\sigma}_a - \sum\limits_{a=p+1}^{p+q} \gv{\sigma}_a 
\eeq
then the infinitesimal transformations would be
\beqB
	\xi \to \xi' = \left( 1 + \frac{\gv{\Sigma} \cdot \gv{\rapidity} }{2} \right) \xi
\eeqB
\beqB
	\eta \to \eta' = \left( 1 - \frac{\gv{\Sigma} \cdot \gv{\rapidity} }{2} \right) \eta
\eeqB
That is satisfied if the exact transformation are
\beqB
		\xi \to \xi' = \exp\left( \frac{\gv{\Sigma} \cdot \gv{\rapidity} }{2} \right) \xi
\eeqB
\beqB
	\eta \to \eta' = \exp \left( -\frac{\gv{\Sigma} \cdot \gv{\rapidity} }{2} \right) \eta
\eeqB
  
So the bispinor of some particle boosted by $\gv{\phi}$ from rest will be
% TODO check passive v. active boost
\beq \label{eq:PsiByXi0}
\Psig = \frac{1}{\sqrt{2}} \begin{pmatrix} 
		\exp\left( \frac{\gv{\Sigma} \cdot \gv{\rapidity} }{2} \right)\xi_0 \\ 
		\exp \left( \frac{-\gv{\Sigma} \cdot \gv{\rapidity} }{2} \right) \xi_0 
	\end{pmatrix}
\eeq

%TODO fix sqrt(2) factors in the right place
The helicial basis made sense to examine the transformation properties of the bispinors.  But in dealing with the relativistic theory, a basis that separates the particle and antiparticle parts of the field will be more convenient.  If the upper component is supposed to be the particle, then in the rest frame the lower component will vanish, and for low momentum will be small compared to the upper component.


  A unitary transformation which accomplishes this is 
Define the new bispinor as
\[
	\Psig' = \begin{pmatrix} \phi \\ \chi \end{pmatrix}
\]
The components of this bispinor are, in terms of the old components $\xi$ and $\eta$
\[
	\phi = \frac{1}{\sqrt{2}}(\xi + \eta)
\]
\[
	\chi = \frac{1}{\sqrt{2}}( \eta - \xi)
\]


And $\Psig'$ can obtained by a unitary transformation:
\[
	\Psig' = \frac{1}{\sqrt{2}} \begin{pmatrix}1 & 1 \\ -1 & 1 \end{pmatrix} \Psig
\]

In terms of the original rest frame spinors, the components of $\Psig'$ are
\beq \label{eq:phiDef}
	\phi =  \cosh \left( \frac{\gv{\Sigma} \cdot \gv{\rapidity} }{2} \right ) \xi_0
\eeq

%TODO consider: Sign confusion compared to original equation again?
\beq \label{eq:chiDef}
	\chi =  \sinh \left( \frac{\gv{\Sigma} \cdot \gv{\rapidity} }{2} \right ) \xi_0
\eeq


\subsection{Properties of $\gv{\Sigma}$ matrices}

The matrices $\Sigma_i$ act on wave functions with $p+q$ indices.  If we treat the two sets of indices as two seperate spaces, with spins $\frac{p}{2}, \frac{q}{2}$ respectively, we can write it as the sum of those spaces spin operators: $\Sigma_i = 2(S^P_i - S^Q_i)$.  We can then write the total spin operator as $S^I_i = S^P_i + S^Q_i$.  (The case of spin one-half is degenerate: $\Sigma_i =  2 S_i = \sigma_i $.)

Because the two spaces are orthogonal, $[S^P_i, S^Q_j]=0$.  Using that and the above definitions, the algebra of these matrices is found.  
\beq \label{eq:Sg:comSSig}
	[S^I_i, \Sigma_j] = i\epsilon_{ijk}\Sigma_k
\eeq
\beq \label{eq:Sg:comSigSig}
	[\Sigma_i, \Sigma_j] = 4i\epsilon_{ijk}S^I_k
\eeq

Wave functions of definite spin are also eigenfunctions of $\Sigma^2$ and $\Sigma \cdot S^I$.  This can be shown by calculating various scalar quantities:
\beqa
	{S^P}^2 &=& \frac{p}{2} ( \frac{p}{2} + 1)	\\
	{S^Q}^2 &=& \frac{q}{2} ( \frac{q}{2} + 1)	\\
	(S^I)^2 	&=& \frac{p+q}{2} ( \frac{p+q}{2} + 1 )	\\
		&=& \frac{p}{2} (\frac{p}{2} + 1) + \frac{q}{2}(\frac{q}{2} + 1 ) + 2 \frac{pq}{4}	\\
		&=& {S^P}^2 + 2\frac{pq}{4} + {S^Q}^2	\\
	 {S^P} \cdot {S^Q} &=& \frac{pq}{4}	\\
	\Sigma ^2 &=& 4(S^P - S^Q)^2	\\
		&=&	4({S^P}^2 + {S^Q}^2 - 2 S^P \cdot S^Q)	\\
		&=& 	4\left(
				\frac{p}{2} ( \frac{p}{2} + 1)
				+\frac{q}{2} ( \frac{q}{2} + 1)
				- \frac{pq}{2}
			\right )	\\
		&=&	p^2 + q^2 - 2pq + 2(p + q)	\\
		&=& 	(p-q)^2 + 2(p+q)	\\
	\Sigma \cdot S &=& 2 (S^P - S_Q) \cdot (S^P + S_Q)	\\
		&=& 2 (S^P)^2 - 2 (S^Q)^2	\\
		&=&	\frac{1}{2} ( p^2 - q^2 + p - q )	\\
\eeqa

Expressing these in terms of $I= \frac{p+q}{2}$ and $\Delta = p-q$ we can write
\beqa
	S^2 		&=&	I(I+1)	\\
	\Sigma^2	&=&	4I + \Delta	\\
	\Sigma \cdot S	&=&	\frac{\Delta}{2} ( 2I + 1)	\\
\eeqa

%TODO write properly in terms of matrix elements and so forth
It will eventually be necessary to take the matrix element of terms containing $\Sigma_i \Sigma_j$ in the nonrelativistic theory.  These must be expressible in terms of the spin operator's matrix element, because that operator spans that space.  Considering the traceless and symmetric structure, it must be proportional to the simlilar structure composed of spin matrices.
\beqa
	\avg{\Sigma_i \Sigma_j + \Sigma_j \Sigma_i - \frac{2}{3} \delta_{ij} \gv{\Sigma}^2}
		&=&	\lambda \avg{ \left \{ I_i I_j + I_j I_i - \frac{2}{3} \delta_{ij} \v{I}^2 \right\} }	\\
\eeqa
To determine the constant $\lambda$, consider the zz component of the tensor structure acting on a wave function with all spin projected in the z direction.  Then $I_3 \to \frac{p+q}{2} = I$, and $\Sigma_3 \to p-q = \Delta$, giving
\beqa
	2 \Delta^2 - \frac{2}{3} (4I + \Delta) 
		&=& \lambda  \left \{ 2 I^2 -\frac{2}{3} (I^2 + I) \right \}	\\
		&=& \lambda \frac{2}{3}I(2I - 1 )
\eeqa
For integer spin $\Delta = 0$, giving
\beqa
	-\frac{8}{3} I &=& \lambda_1 \frac{2}{3}I (2I -1)	\\
	\lambda_1 &=& -\frac{4}{2I-1}	\\
\eeqa
For half spin we have $\Delta = 1$ 
\beqa
	\frac{4}{3} (1  - 2I) &=& \lambda_{\frac{1}{2}}	 \frac{2}{3}I (2I -1)	\\
	\lambda_{\frac{1}{2}} &=&-\frac{4}{2I}	\\
\eeqa



\subsection{Bilinears in the relativistic theory}
Recall that in the case of spin one-half it was possible to catalogue a complete set of field bilinears with definite Lorentz transformation properties.  The structure of the electron vertex was necessarily expressible in terms of those bilinears.  These bilinears were built out of the $4\times 4$ $\gamma^\mu$ matrices, which were themselves built out of $2 \times 2$ $\sigma$ matrices.

What is the generalisation of these bilinears for the case of genearl spin?  The fields have been written as bispinors, with upper and lower components spinors with transformation properties defined above.  Whereas in the case of spin one-half $\{ \mathcal{I}, \sigma_i\}$ formed a basis for operators acting on these spinors, here more complicated quantities built out of $S$ and $\Sigma$ are allowed.

It is still possible to write down a set of bilinears with definite transformation properties.  The greater the spin (and thus the degrees of freedom of the spinors) the larger their number and rank.  However, for the purposes of constructing the electromagnetic current and then calculating its nonrelativistic limit, only bilinears of up to rank 2 tensors need be considered.

In the appendix \ref{chap:bilinear} the transformations of various bilinears are worked out.  There are two necessary for constructing the electromagnetic current; the simple scalar $\Psigbar \Psi$ and the tensor $\Psigbar \TensBi_{\mu\nu} \Psig$.
 

%Section on bilinear transforms
%
%FIXME check for possible typos in calculations (I seem to remember there were some here?)


\section{Transformations of bilinears in the case of general spin}
\label{chap:bilinear}
We have the transformation of the spinor under small boosts:
\beqa
	\Psi &\to& \Psi' = \Psi + \frac{\eta_i }{2} \begin{pmatrix} 0 & \Sigma_i \\ \Sigma_i & 0 \end{pmatrix}\Psi
\eeqa
\beqa
	\bar{\Psi} &\to& \bar{\Psi'} = \bar{\Psi} - \frac{\eta_i }{2} \bar{\Psi} \begin{pmatrix} 0 & \Sigma_i \\ \Sigma_i & 0 \end{pmatrix}
\eeqa

We can also see the transformation of the spinor under parity: simply put, because the upper component is even in $\gv{\Sigma} \cdot \v{p}$, whereas the lower component is odd, we obtain
\[
	\Psi \to \begin{pmatrix} 1 & 0 \\ 0 & -1 \end{pmatrix}\Psi
\]
\[	\bar{\Psi} \to \bar{\Psi} \begin{pmatrix} 1 & 0 \\ 0 & -1 \end{pmatrix}
\]
So
\[
	\bar{\Psi} \begin{pmatrix} A & B \\ C & D \end{pmatrix} \Psi
		\to
	\bar{\Psi} \begin{pmatrix} A & -B \\ -C & D \end{pmatrix} \Psi
\]

From these facts we can examine the general behavior of bilinears under Lorentz transformations.

Now we'll examine the behavior of bilinears under boosts.  We can write the general structure of the bilinear as
\[
	\bar{\Psi} T \Psi = 	\bar{\Psi} \begin{pmatrix} A + D & B+C \\ B-C & A - D \end{pmatrix} \Psi
\]
or using a different notation
\[
\bar{\Psi} T \Psi = 	\bar{\Psi} 
	\left [
			A \otimes \begin{pmatrix} 1 & 0 \\ 0 & 1 \end{pmatrix}
			+ D \otimes \begin{pmatrix} 1 & 0 \\ 0 & -1\end{pmatrix}			
			+ B \otimes \begin{pmatrix} 0 & 1 \\ 1 & 0 \end{pmatrix}
			+ C \otimes \begin{pmatrix} 0 & 1 \\ -1 & 0 \end{pmatrix}
	\right)]  \Psi
\]

Under an infitesimal Lorentz boost $\gv{\eta}$ this will transform into
\[
	\bar{\Psi} T \Psi \to \bar{\Psi} T \Psi
		+ \frac{\eta_i}{2} \bar{\Psi} \left (  \begin{pmatrix} A + D & B+C \\ B-C & A - D \end{pmatrix} \begin{pmatrix} 0 & \Sigma_i \\ \Sigma_i & 0 \end{pmatrix} - \begin{pmatrix} 0 & \Sigma_i \\ \Sigma_i & 0 \end{pmatrix} \begin{pmatrix} A + D & B+C \\ B-C & A - D \end{pmatrix} \right ) \Psi					
\]
We can express this in terms of commutators and anti-commutators
\[
	\bar{\Psi} T \Psi \to \bar{\Psi} T \Psi
		+ \frac{\eta_i}{2} \bar{\Psi} \left [
			[B, \Sigma_i] \otimes \begin{pmatrix} 1 & 0 \\ 0 & 1 \end{pmatrix}
			+ [A, \Sigma_i] \otimes \begin{pmatrix} 0 & 1 \\ 1 & 0 \end{pmatrix}
 			+ \{C, \Sigma_i\} \otimes \begin{pmatrix} 1 & 0 \\ 0 & -1\end{pmatrix}
			+ \{D, \Sigma_i\} \otimes \begin{pmatrix} 0 & 1 \\ -1 & 0 \end{pmatrix}
	\right)] \Psi
\]

We can note here that, using only the matrices $\gv{\Sigma}$ and $\gv{S}$ we can build three structures invariant under rotations: and $S^2$, $\Sigma^2$, and $\Sigma \cdot S$.  All three of these structures commute with both $S_i$ and $\Sigma_i$, and their value depends only on the particular representation we're working with.  So for our purposes here, they can just be treated as pure numbers.


\subsubsection{Scalar bilinears}
Since the scalar must be invariant to rotation, then by the logic above it's block elements are proportional to the identity.

It must also be unchanged under boosts.  We can see that this necessitates that $C=D=0$, while providing no constraint on A and B.  So the general form of a bilinear invariant under boosts is
\[
	\bar{\Psi} T \Psi = \bar{\Psi} \begin{pmatrix} A & B \\B & A \end{pmatrix} \Psi
\]
where A and B are proportional to the identity.

Under the discrete partiy transformation this will transform into
\[
	\bar{\Psi} T' \Psi = \bar{\Psi} \begin{pmatrix} A & -B \\-B & A \end{pmatrix} \Psi
\]
This shows that for a true scalar, $B=0$.


\subsubsection{Vector bilinears}
To attempt to construct a vector bilinear, we can start by considering the time-like part of it.  Since this must be invariant under spatial rotations, then by the same logic as above it must essentially be composed of four blocks proportional to the identity.  We also know that under boosts the time-like part is transformed into the spatial and vice versa, so we can use these linked transformations to obtain constraints on the bilinear.

If $T^\mu$ is a vector we know that, under an infinitesimal boost, it's transformation will be
\beqa
	T^0 &\to& T^0 + \eta_i T^i	\\
	T^i &\to& T^i + \eta_i T^0
\eeqa

We again write 
\[T^\mu = \begin{pmatrix}A^\mu + D^\mu & B^\mu+C^\mu \\ B^\mu-C^\mu & A^\mu - D^\mu  \end{pmatrix} \]

Then we see that under boost, $T^0$ transforms as

\[
	\bar{\Psi} T^0 \Psi \to 	\bar{\Psi} T^0 \Psi
	+  \eta_i \bar{\Psi} \left [
			C^0 \Sigma_i \otimes \begin{pmatrix} 1 & 0 \\ 0 & -1\end{pmatrix}
			+ D^0 \Sigma_i \otimes \begin{pmatrix} 0 & 1 \\ -1 & 0 \end{pmatrix}
	\right] \Psi
\]
where we've used that fact that all the components of $T^0$ commute with $\Sigma_i$.
This tells us that for $T^\mu$ to be a 4-vector, the following must be true.
\beqa
	A^i &=& 0	\\
	B^i &=& 0 	\\
	C^i &=& D^0 \Sigma^i	\\
	D^i &=& C^0 \Sigma^i	\\
\eeqa

We can now consider how $T^i$ changes under a boost, and discover

\beqa
\bar{\Psi} T^i \Psi 
	&\to& \bar{\Psi} T^i \Psi
		+ \frac{\eta_j}{2} \bar{\Psi} \left [
			\{C^i, \Sigma^j\} \otimes \begin{pmatrix} 1 & 0 \\ 0 & -1\end{pmatrix}
			+ \{D^i, \Sigma^j\} \otimes \begin{pmatrix} 0 & 1 \\ -1 & 0 \end{pmatrix}
		\right)] \Psi	\\
	&=& \bar{\Psi} T^i \Psi
		+ \frac{\eta_j}{2} \bar{\Psi} \left [
			D^0\{\Sigma^i, \Sigma^j\} \otimes \begin{pmatrix} 1 & 0 \\ 0 & -1\end{pmatrix}
			+ C^0\{\Sigma^i, \Sigma^j\} \otimes \begin{pmatrix} 0 & 1 \\ -1 & 0 \end{pmatrix}
		\right)] \Psi	\\
\eeqa
Again considering our demand that $T^\mu$ transform like a 4-vector, we get
\beqa
	A^0 &=& 0	\\
	B^0 &=& 0	\\
	C^0 \delta^{ij} &=&  C^0 \frac{1}{2}\{\Sigma^i, \Sigma^j\}	\\
	D^0 \delta^{ij} &=&  D^0 \frac{1}{2}\{\Sigma^i, \Sigma^j\}	\\
\eeqa
The last two constraints are met in the spin-1/2 case, but not for higher spins.  This tells us that there's no way to, in the higher spin case, construct a vector bilinear using only I, $\gv{\Sigma}$, and $\v{S}$.

For spin-1/2, where $\Sigma_i = \sigma_i$, we see that a true vector bilinear (with correct transformation properties under parity) will be proportional to 
\[
	(T^0, \v{T} ) = \left( \begin{pmatrix} 1 & 0 \\ 0 & -1 \end{pmatrix} , \begin{pmatrix} 0 & \gv{\sigma} \\ -\gv{\sigma} & 0 \end{pmatrix} \right )
\]
which, of course, is exactly what we knew already.

\subsubsection{Tensor bilinears}
Here we'll be a little less ambitious.  We can tell from the above considerations that, under boosts, we effectively mix A and B components seperately from the C and D blocks.  What's more, we need anti-commutation relationships to deal with the latter transformations.  So we'll just consider tensors that look like
\[
	T^{\mu\nu} = \begin{pmatrix} A^{\mu\nu} & B^{\mu\nu} \\ B^{\mu\nu} & A^{\mu\nu} \end{pmatrix}	
\]
Furthermore, we'll consider only anti-symmetric tensors for now.  Now we can basically procede as in the vector case, knowing how an anti-symmetric tensor should transform:
\beqa
	T^{0i} &\to& T^{0i} +  \eta_j T^{ji}	\\
	T^{ij}	&\to& T^{ij} + \eta_i T^{0j} + \eta_j T^{i0}	\\
\eeqa

Start by considering the components of $T^{0i} = -T{i0}$.  They must transform as vectors under rotations.  We have two vectors available to us, so we can write
\beqa
	A^{0i} &=& \alpha \Sigma^i + \beta S^i	\\
	B^{0i} &=& \gamma \Sigma^i + \delta S^i	\\
\eeqa

Under a boost, we find the relation that
\beqa
	A^{ji} &=& [B^{0i}, \Sigma^j]	\\
	B^{ji} &=& [A^{0i}, \Sigma^j]	\\	
\eeqa
And then looking at how $T^{ij}$ transforms, we get the constraint
\beqa
	\eta_k \left[ [A^{0i}, \Sigma^j], \Sigma^k \right] &=& \eta_j A^{0i} - \eta_i A^{0j}	\\
	\eta_k \left[ [B^{0i}, \Sigma^j], \Sigma^k \right] &=& \eta_j B^{0i} - \eta_i B^{0j}	\\
\eeqa
We're assuming that both $A^{0i}$ and $B^{0i}$ are linear combinations of $\Sigma_i$ and $S_i$.  So what we need are the relationships
\beqa
	[\Sigma^i, \Sigma^j] &=& 4 i \epsilon_{ijk} S^k	\\
	{}[S^i, \Sigma^j] &=& i\epsilon_{ijk} \Sigma^k 	\\
	{}[[\Sigma^i, \Sigma^j], \Sigma^k] 
		&=& 4 i\epsilon_{ij\ell} [S^\ell, \Sigma^k]	\\
		&=& -4 \epsilon_{ij\ell} \epsilon_{\ell k m} \Sigma^m \\		
		&=& -4 (\delta_{ik} \Sigma^j - \delta_{jk} \Sigma^i)	\\ 
	{}[[S^i, \Sigma^j], \Sigma^k] 
		&=&  i\epsilon_{ij\ell} [\Sigma^\ell, \Sigma^k]	\\
		&=& -4 \epsilon_{ij\ell} \epsilon_{\ell k m} S^m \\		
		&=& -4 (\delta_{ik} S^j - \delta_{jk} S^i)	\\ 
\eeqa
So we can see that, no matter what $\alpha$ and $\beta$ are, we get the relation
\[
	[ [A^{0i}, \Sigma^j], \Sigma^k]
		=
	-4 (\delta{ik} A^{0i} -\delta_{jk} A^{0j } 
\]
And so necessarily, 
\[
	\eta_k[ [A^{0i}, \Sigma^j], \Sigma^k]
		=
	4 (\eta_j A^{0i} -  \eta_i A^{0j} )
\]
So any arbitrary combination of $\v{S}$ and $\gv{\Sigma}$ will allow us to construct a bilinear that transforms as a tensor.

(In fact, it's not hard to generalise this to any operator expressable as a linear combination of $\sigma^A_i$, where the index A represents which spinor index $\sigma$ operates on.)





\subsection{Electromagnetic Interaction}
%TODO insert diagram, showing the type of interaction we're talking about, defining momenta of particles in question.
Knowing how the fields themselves behave, what form might the electromagnetic interaction take?  In general, it can be written
\[
	M = e A_\mu j^\mu 
\]
where $j^\mu$ is the electromagnetic current.


The electromagnetic current must be built out of the particle's momenta and bilinears of the charged particle fields in such a way that they have the correct Lorentz properties.  It must also obey current conservation: the equation $q_\mu j^\mu = 0$ must hold.  Above it was shown that, in the case of general spin, there exist two such bilinears, a scalar and a tensor.  A vector bilinear could not be constructed.

%TODO address concern about p_0 terms
As higher spins are considered, higher order bilinears will appear.  But necessarily, they will be coupled to a greater number of powers of the external momenta, and so suppressed in the nonrelativistic limit.

So there will be just two relevant terms in the current that transform as Lorentz vectors.  One is a scalar bilinear coupled with a single power of external momenta.  In order to fulfill the current conservation requirement, the exact combination will be
\[
	\frac{p^\mu + p'^\mu}{2m} \Psigbar^\dagger \Psig
\]
This obeys current conservation because $q = p' -p$, and $ (p+p')\cdot(p'-p) = p^2-p'^2=0$

The other type of term will be a tensor term contracted with a power of momenta.  To fulfill current conservation, the tensor must be antisymmetric, and contracted with $q$:
\[
	\frac{q_\nu}{2m} \Psigbar^\dagger \TensBi^{\mu\nu} \Psig
\]

There is no need to consider higher order tensor bilinears; they will necessitate additional powers of the external momenta.

So the most general current would look like
\beq \label{eq:khr_current}
	j^\mu = F_e \frac{p^\mu + p'^\mu}{2m} \Psigbar^\dagger \Psig + F_m 	\frac{q_\nu}{2m} \Psigbar^\dagger \TensBi^{\mu\nu} \Psig	
\eeq
The form factors might have quite complicated dependence on $q$, but these corrections will be too small.  They will be suppressed beyond the order of the calculation.  At leading order $F_e$ will just be the electric charge of the particle in question, and $F_m$ will, as will be seen after calculating the nonrelativistic limit, be related to the particle's $g$-factor.  So to the order needed, the current can be written
%TODO check definition of form factors F_e and F_m
\beq 
	j^\mu =  e \frac{p^\mu + p'^\mu}{2m} \Psigbar^\dagger \Psig +   e g \frac{q_\nu}{2m} \Psigbar^\dagger \TensBi^{\mu\nu} \Psig
\eeq

%TODO Actually show this formally: that the two types of antisymmetric tensors aren't truly different
%The tensor bilinear $\Psigbar^\dagger T^{\mu\nu} \Psig$ itself has some free parameters.  However, it'll turn out that, when computing actual nonrelativistic amplitudes, the two types of anti-symmetric tensors collapse into the same general form.

This captures the essence of the interaction between a charged particle of general-spin and a single photon.


\section{Comparison of scattering amplitudes}

\subsection{Connection between the spinors of the two theories}
Having sufficient knowledge of the relativistic theory, the approach of NRQED can be applied.  In NRQED, the amplitude for scattering off an external field has been calculated.  To compare the same process in this relativistic theory, first a way to write the nonrealtivistic spinors in terms of the relativistic ones is needed.

In the rest frame, there are two independent bispinors which represent particle and antiparticle states.  The particle state is represented by
\beq
	\Psig = \begin{pmatrix} \xi_0 \\ 0 \end{pmatrix}
\eeq
the antiparticle by
\beq
	\Psig = \begin{pmatrix} 0 \\ \xi_0 \end{pmatrix}
\eeq

However, when considering a particle with non-zero momentum it is not the case that the upper component of the bispinor can be directly associated with the Schrodinger like wave-function of the particle --- probability would not be properly conserved, for there is some mixing with the lower component, and thus a nonzero chance of the particle being in such a state.

To obtain a relation between $\xi_0$ and the Schrodinger amplitude $\phi_s$,  consider the current density at zero momentum transfer.  For $\phis$ it will be $j_0 =  \phis^\dagger \phis$.  For the relativistic theory, as calculated above:
\beq
	j^0 = F_e \frac{p^0 + p'^0}{2m} \Psigbar^\dagger \Psig + F_m 	\frac{q_\nu}{2m} \Psigbar^\dagger T^{0\nu} \Psig	
\eeq
At $q=0$ the expression simplifies

\beq
	j^0(q=0) = F_e \frac{p_0}{m} \Psigbar^\dagger \Psig
	= F_e  \frac{p_0}{m}( \phi^\dagger \phi - \chi^\dagger \chi )
\eeq

$\phi$ and $\chi$ are both related to the rest frame spinor $\xi_0$.  In such terms, $j^0$ is given by
\beq
	j^0 = F_e \frac{p_0}{m} \xi_0^\dagger \left \{ 
		\cosh^2( \frac{\gv{\Sigma} \cdot \gv{\rapidity} }{2})
		- \sinh^2( \frac{\gv{\Sigma} \cdot \gv{\rapidity} }{2})
	\right \} \xi_0  
		=	F_e \frac{p_0}{m} \xi_0^\dagger \xi_0
\eeq
where the last equality follows from the hyperbolic trig identity.


Demanding that the two current densities be equal to each other, the result is that
\beq
	\frac{p_0}{m} \xi_0^\dagger \xi_0 = \phi_s^\dagger \phi_s
\eeq
As throughout the calculation, only corrections of order $1/m^2$ are needed.  So appoximating
\beq
	\left( 1 + \frac{\v{p}^2}{2m} \right) \xi_0^\dagger \xi_0 = \phi_s^\dagger \phi_s
\eeq

This will hold to the necessary order if
\beq
	\xi_0 = \left( 1 - \frac{\v{p}^2}{4m} \right) \phi_s
\eeq


%TODO fix cosh expansion

That relates the the Schrodinger like wave functions to the quantities $\xi_0$.  To write the relativistic bispinors in terms of $\phis$, approximations for $\cosh( \frac{\gv{\Sigma} \cdot \gv{\rapidity} }{2})$ and $\sinh( \frac{\gv{\Sigma} \cdot \gv{\rapidity} }{2})$ are also needed.  The rapidity is needed only to the leading order: $\gv{\rapidity} \approx \v{v} \approx \frac{\v{p} }{m}$. 

\beq
	\cosh( \frac{\gv{\Sigma} \cdot \gv{\rapidity} }{2}) 
		\approx 1 + \frac{1}{2}\left( \frac{\gv{\Sigma} \cdot \v{p} }{2m} \right)^2
\eeq
\beq
	\sinh( \frac{\gv{\Sigma} \cdot \gv{\rapidity} }{2}) 
		\approx   \frac{\gv{\Sigma} \cdot \v{p} }{2m}
\eeq
Using this, the two bispinor components are
%Non relativistic expression for \phi and \chi
\begin{align}
\phi 
	&\approx  \left(  1 + \left[ \frac{1}{2}\frac{\gv{\Sigma} \cdot \v{p} }{2m} \right]^2 \right) \xi_0 \notag \\
	&\approx  \left(  1 + \frac{(\gv{\Sigma} \cdot \v{p})^2 }{8m^2} - \frac{\v{p}^2}{4m} \right ) \phis	 \label{eq:nrPhi} \\
 \chi
 	&\approx	\frac{\gv{\Sigma} \cdot \v{p} }{2m} \xi_0 \notag \\
 	&\approx	\frac{\gv{\Sigma} \cdot \v{p} }{2m} \phis  \label{eq:nrChi}
\end{align}




%TODO remember to hunt out all the \phi where I mean \rapidity
\subsection{Bilinears in terms of nonrelativistic theory}
The next step is to express the relativistic bilinears, built out of the bispinors $\Psig$, in terms of the Schrodinger like wave functions.

Above the bispinors were written terms of $\phi_s$, so those identities can be used to express the bilinears in the same manner.  


\subsubsection{Scalar bilinear}
%FIXME  proper reference to sigma commutator
Calculating the scalar bilinear is just straightforward substitution and expansion.  To express the commutator of $\Sigma$ operators, the identity \eqref{eq:Sg:comSigSig} is needed.
\beqa
\Psigbar^\dagger(p') \Psig(p)
	&=&	\phi^\dagger \phi - \chi^\dagger \chi	\\
	&=&	\phi_s^\dagger \left[1 + \frac{(\gv{\Sigma} \cdot \v{p'})^2 }{8m^2}  - \frac{\v{p'}^2}{4m^2} \right ]
			 \left[1 + \frac{(\gv{\Sigma} \cdot \v{p})^2 }{8m^2}  - \frac{\v{p}^2}{4m^2} \right ] \phi_s
		- \phi_s^\dagger \left[
			\frac{ ( \gv{\Sigma} \cdot \v{p'}) (\gv{\Sigma} \cdot \v{p}) }{4m^2}
		\right ] \phi_s	\\
	&=&	\phi_s^\dagger \left (
			1 - \frac{ \v{p}^2 + \v{p'}^2 }{4m^2}
			+ \frac{1}{8m^2} \left \{
				( \gv{\Sigma} \cdot \v{p'})^2 +  (\gv{\Sigma} \cdot \v{p})^2 
				 - 2 ( \gv{\Sigma} \cdot \v{p'}) (\gv{\Sigma} \cdot \v{p})
			\right \}
	\right ) \phi_s	\\
	&=& \phi_s^\dagger \left (
			1 - \frac{ \v{p}^2 + \v{p'}^2 }{4m^2}
			+ \frac{1}{8m^2} \left \{
				[ \gv{\Sigma} \cdot \v{p},  \gv{\Sigma} \cdot \v{q}]  + ( \gv{\Sigma} \cdot \v{q})^2 
			\right \}
	\right ) \phi_s	\\
	&=& \phi_s^\dagger \left (
			1 - \frac{ \v{p}^2 + \v{p'}^2 }{4m^2}
			+ \frac{1}{8m^2} \left \{
				[ 4 i \epsilon_{ijk} p_i q_j S_k  + ( \gv{\Sigma} \cdot \v{q})^2 
			\right \}
	\right ) \phi_s
\eeqa
%TODO at some point need to translate \Sigma functions into spin functions in NR theory, but where?

\subsubsection{Tensor $ij$ component}


%FIXME eqref problem
In calculating the nonrelativistic limit of the antisymmetric tensor bilinear, the $0i$ and the $ij$ components will be treated separately.  First consider $\Psigbar \Sigma_{ij} \Psig$.  The value of $\TensBi_{ij}$ itself was written in \eqref{??}

\beqa
	\Psigbar \TensBi_{ij} \Psig 
		&=& \Psigbar (-2\epsilon_{ijk} S_k) \Psig	\\
		&=&	-2i\epsilon_{ijk} ( \phi^\dagger S_k \phi - \chi^\dagger S_k \chi)	\\
		&=&	-2i\epsilon_{ijk} \Big( \phis^\dagger \left[ 1 + \frac{( \gv{\Sigma} \cdot \v{p'})^2}{8m^2}  -\frac{\v{p'}^2}{4m^2} \right] S_k \left[ 1 + \frac{( \gv{\Sigma} \cdot \v{p})^2}{8m^2} -\frac{\v{p}^2}{4m^2}\right ] \phis - \phis^\dagger \frac{ ( \gv{\Sigma} \cdot \v{p'})S_k ( \gv{\Sigma} \cdot \v{p })}{4m^2} \phis \Big )	\\
		&=&	-2i\epsilon_{ijk} \phis^\dagger \left \{
				S_k \left( 1 - \frac{ \v{p}^2 + \v{p'}^2}{4m^2}  \right )
				+ \frac{1}{8m^2} \underbrace{\Big[ ( \gv{\Sigma} \cdot \v{p'})^2 S_k + S_k ( \gv{\Sigma} \cdot \v{p})^2 - 2 ( \gv{\Sigma} \cdot \v{p'})S_k ( \gv{\Sigma} \cdot \v{p}) \Big ]}_{T_k}
			\right \} \phis
\eeqa

Call the term in square brackets $T_k$.  It should be written explicitly in terms of $\v{p}$ and $\v{q}$.
\beqa
T_k	&=& ( \gv{\Sigma} \cdot \v{p})^2 S_k + S_k ( \gv{\Sigma} \cdot \v{p}) -2 ( \gv{\Sigma} \cdot \v{p}) S_k ( \gv{\Sigma} \cdot \v{p})
		+ \{ ( \gv{\Sigma} \cdot \v{p}) ( \gv{\Sigma} \cdot \v{q}) 
	\\ &&	+ ( \gv{\Sigma} \cdot \v{q}) ( \gv{\Sigma} \cdot \v{p}) \}S_k
		- 2 ( \gv{\Sigma} \cdot \v{q}) S_k ( \gv{\Sigma} \cdot \v{p})
		+ ( \gv{\Sigma} \cdot \v{q})^2 S_k
\eeqa

Many of these terms may be expressed as commutators, and then these commutators simplified.
\beqa
T_k	 &=&	\gv{\Sigma} \cdot \v{p} [ \gv{\Sigma} \cdot \v{p}, S_k] + [S_k, \gv{\Sigma} \cdot \v{p}] \gv{\Sigma} \cdot \v{p}
		+ 2 \gv{\Sigma} \cdot \v{q} [ \gv{\Sigma} \cdot \v{p}, S_k] - [\gv{\Sigma} \cdot \v{q}, S_k] \gv{\Sigma} \cdot \v{p}
		+ (\gv{\Sigma} \cdot \v{q})^2 S_k
	\\&=& i\epsilon_{ijk} p_j \{ (\gv{\Sigma} \cdot \v{p})\Sigma_i - \Sigma_i (\gv{\Sigma} \cdot \v{p}) \}
		+ 2 i\epsilon_{ijk} \{ (\gv{\Sigma} \cdot \v{q}) \Sigma_i p_j -  \Sigma_i (\gv{\Sigma} \cdot \v{p}) q_j ) \}
		+ (\gv{\Sigma} \cdot \v{q})^2 S_k
	\\&=& i\epsilon_{ijk} p_j [ (\gv{\Sigma} \cdot \v{p}), \Sigma_i  ]
		+ 2 i\epsilon_{ijk} \{ (\gv{\Sigma} \cdot \v{q}) \Sigma_i p_j -  \Sigma_i (\gv{\Sigma} \cdot \v{p}) q_j ) \}
		+ (\gv{\Sigma} \cdot \v{q})^2 S_k
	\\&=& 4( \v{p}^2 S_k - (\v{S} \cdot \v{p}) p_k ) 
		+ 2 i\epsilon_{ijk} \{ (\gv{\Sigma} \cdot \v{q}) \Sigma_i p_j -  \Sigma_i (\gv{\Sigma} \cdot \v{p}) q_j ) \}
		+ (\gv{\Sigma} \cdot \v{q})^2 S_k
\eeqa
Thus the whole bilinear is
\beq \begin{split}
\Psigbar \TensBi_{ij} \Psig = 
	-2i\epsilon_{ijk} \phis^\dagger \Bigg \{
				S_k \left( 1 - \frac{ \v{p}^2 + \v{p'}^2}{4m^2}  \right )
				+ \frac{1}{8m^2} \Big[ 
				4( \v{p}^2 S_k - (\v{S} \cdot \v{p}) p_k ) 
			\\	+ 2 i\epsilon_{\ell m k} \{ (\gv{\Sigma} \cdot \v{q}) \Sigma_\ell p_m -  \Sigma_\ell (\gv{\Sigma} \cdot \v{p}) q_m ) \}
				+ (\gv{\Sigma} \cdot \v{q})^2 S_k
			 \Big ]
			\Bigg \} \phis
\end{split}
\eeq

\subsubsection{Tensor $\Sigma_{0i}$ component}

Now calculate the $0i$ component, $\Psigbar \Sigma_{0i} \Psig$.

\beq
	\Psigbar \Sigma_{0i} \Psig = \Psigbar \begin{pmatrix} 0 & \Sigma_i \\ \Sigma_i & 0 \end{pmatrix} \Psig
\eeq

\beq
	=	\phi^\dagger \Sigma_i \chi - \chi^\dagger \Sigma_i \phi
\eeq

Because this tensor structure is coupled to $q_i/m$, $\phi$ and $\chi$ are needed only to first order here.
\beq
	\Psigbar \Sigma_{0i} \Psig = \phis^\dagger \left( \frac{\Sigma_i \Sigma_j p_j - \Sigma_j \Sigma_i p'_j}{2m} \right ) \phis
\eeq
Using $p'=p+q$ the terms involving only $p$ can be simplified using the commutator of $\Sigma$ matrices.
\beq
	\Psigbar \Sigma_{0i} \Psig =\phis^\dagger \left( \frac{ 4i\epsilon_{ijk} p_j S_k - \Sigma_j \Sigma_i q_j}{2m} \right )\phis
\eeq



%%%%%%%  Calculate the Current
%%%%%%%%%%%%%%%%%%%%%%%%%%%%%%%%%
\subsection{Current in terms of nonrelativistic wave functions}

%add ref to equation
The four-current \eqref{eq:khr_current} was derived above; in Galilean form it is 
\beq
	j_0 =  F_e \frac{p_0 + p'_0}{2m} \Psigbar^\dagger \Psig -  F_m \frac{q_j}{2m} \Psigbar^\dagger \TensBi^{0j} \Psig
\eeq

\beq
	j_i =  F_e \frac{p_i + p'_i}{2m} \Psigbar^\dagger \Psig -   F_m \frac{q_j}{2m} \Psigbar^\dagger \TensBi^{ij} \Psig 
			+F_m \frac{q_0}{2m} \Psigbar^\dagger \TensBi^{i0} \Psig
\eeq

Expressions for the bilinears in terms of the nonrelativistic wave functions $\phis$ have been calcaluted, so it is fairly straight forward to apply them here.  The calculation of $j_0$ goes:
\beqa
F_e \frac{p_0 + p'_0}{2m} \Psigbar^\dagger \Psig 
	  &=& F_e \left(1 + \frac{\v{p}^2 + \v{p'}^2}{4m^2} \right)  \phis^\dagger \left (
			1 - \frac{ \v{p}^2 + \v{p'}^2 }{4m^2}
			+ \frac{1}{8m^2} \left \{
				 4 i \epsilon_{ijk} p_i q_j S_k  + ( \gv{\Sigma} \cdot \v{q})^2 
			\right \}
	\right ) \phis	\\
	&\approx& 	F_e   \phis^\dagger \left (
					1 + \frac{1}{8m^2} \left \{ 4i \v{S} \cdot \v{p} \times \v{q}  + ( \gv{\Sigma} \cdot \v{q})^2 \right \}
				\right ) \phis	\\
F_m \frac{q_j}{2m} \Psigbar^\dagger \TensBi^{0j} \Psig
	&=& F_m \frac{q_i}{2m}\phis^\dagger \left( \frac{ 4i\epsilon_{ijk} p_j S_k - \Sigma_j \Sigma_i q_j}{2m} \right )\phis	\\
	&=&  F_m \phis^\dagger \left( \frac{ 4i \v{S} \cdot \v{q} \times \v{p} -  (\gv{\Sigma} \cdot \v{q})^2 }{4m^2} \right )\phis	\\
\eeqa

It turns out that both terms here have the same form, so combining them, 
\beq \label{eq:nrJ0}
j_0 =  	 \phis^\dagger \left (
			F_e + \frac{F_e + 2F_m}{8m^2} \left \{ 4i \v{S} \cdot \v{p} \times \v{q}  + ( \gv{\Sigma} \cdot \v{q})^2  \right \}
		\right ) \phis	\\
\eeq


%Justify dropping derivatives of magnetic field a bit better
To calculate $j_i$ it helps to first simplify things by considering the constraints of this particular problem.  The term with $\Sigma_ij$ can be simplified by dropping terms with more than one power of $q$; these will turn into derivatives of the magnetic field, and this problem concerns only a constant field.  Further, only elastic scattering is considered, and so $q_0=0$.  With those simplifications
\beq
\Psigbar \Sigma_{ij} \Psig \approx
		-2i\epsilon_{ijk} \phis^\dagger \left \{
			S_k \left( 1 - \frac{ \v{p}^2 + \v{p'}^2}{4m^2}  \right )
			+ \frac{\v{p}^2 S_k - (\v{S} \cdot \v{p}) p_k}{2m^2}  
		\right \} \phis
\eeq


\beqa
F_e \frac{p_i + p'_i}{2m} \Psigbar^\dagger \Psig
	&=&			F_e \frac{p_i + p'_i}{2m}  \phis^\dagger \left (
					1 - \frac{ \v{p}^2 + \v{p'}^2 }{4m^2}
					+ \frac{1}{8m^2} \left \{  4 i \epsilon_{\ell jk} p_\ell q_j S_k  + ( \gv{\Sigma} \cdot \v{q})^2	\right \}
				\right ) \phis	\\
	&\approx& 	F_e \frac{p_i + p'_i}{2m}  \phis^\dagger \left (
					1 + \frac{1}{8m^2} \left \{ 4 i \epsilon_{\ell jk} p_\ell q_j S_k  \right \}
				\right ) \phis	\\
F_m \frac{q_j}{2m} \Psigbar^\dagger \TensBi^{ij} \Psig
	&=& 		- F_m\frac{ i\epsilon_{ijk} q_j}{m} \phis^\dagger \left \{
					S_k \left( 1 - \frac{ \v{p}^2 + \v{p'}^2}{4m^2}  \right )
					+ \frac{\v{p}^2 S_k - (\v{S} \cdot \v{p}) p_k}{2m^2}  
				\right \} \phis
\eeqa

So the full spatial part of the current is
\beq \label{eq:nrJi}
j_i	=	\phis^\dagger \Bigg \{
			F_e \frac{p_i + p'_i}{2m} \left (
				1 + \frac{ i \epsilon_{\ell jk} p_\ell q_j S_k   }{2m^2}  \right)
			+ F_m   \frac{i\epsilon_{ijk} q_j}{m} \left( 
				S_k \left \{ 1 - \frac{ \v{p}^2 + \v{p'}^2}{4m^2}  \right \}
				+ \frac{\v{p}^2 S_k - (\v{S} \cdot \v{p}) p_k}{2m^2} \right)	
		\Bigg \} \phis
\eeq

\subsection{Scattering off an external field}
To compare to the NRQED Lagrangian, scattering off an external field needs to be calculated for an arbitrary spin particle.  The hardest part was calculating the current; with that in hand the amplitude can be found just as
\[
	M = e j_\mu A^\mu = e j_0 A_0 - e \v{j} \cdot \v{A}
\]
With the expressions for both $j_0$ \eqref{eq:nrJ0} and $\v{j}$ \eqref{eq:nrJi} both parts of the amplitude can be written down directly.
\beq
\begin{split}
	e j_0 A_0 = 
		& eA_0 \phis^\dagger \left (
			F_e + \frac{F_e + 2F_m}{8m^2} \left \{ 4i \v{S} \cdot \v{p} \times \v{q}  + ( \gv{\Sigma} \cdot \v{q})^2  \right \}
		\right ) \phis	\\
   e\v{j} \cdot \v{A} =
		& A_i \phis^\dagger \Bigg \{
			F_e \frac{p_i + p'_i}{2m} \left (
				1 + \frac{ i \epsilon_{\ell jk} p_\ell q_j S_k   }{2m^2}  \right)
			+ F_m   \frac{i\epsilon_{ijk} q_j}{m} \left( 
				S_k \left( 1 - \frac{ \v{p}^2 + \v{p'}^2}{4m^2}  \right )
				+ \frac{\v{p}^2 S_k - (\v{S} \cdot \v{p}) p_k}{2m^2} \right)	
		\Bigg \} \phis
\end{split}
\eeq  


To compare to the NRQED result, as much as possible terms should be expressed in gauge invariant quantities $\v{B}$ and $\v{E}$.  The relations between these fields and $A_\mu$ in position space and the equivalent equation in momentum space are:
%FIXME add reference to gauge
\beq
\begin{split}
	\v{B} &= \grad \times \v{A}	\notag \to i\v{q} \times \v{A} \\
	\v{E} &= -\grad A_0	 \to -i\v{q} A_0 		\notag
\end{split}
\eeq

There is one term above that can only be put into gauge-invariant form by considering the kinematic constraints of elastic scattering. As previously derived in \eqref{eq:Sh:pqA}, the identity is
\beq \label{eq:Sg:ppqAid}
	i (p_i + p'_i) q_j A_i = - \epsilon_{ijk}B_k (p_i + p'_i)
\eeq 

Now each term involving $q$ can be written in terms of $E$ or $B$.
%TODO maybe write quad term more explicitly in terms of quad moment and term with trace
\beqa
i \v{S} \cdot \v{p} \times \v{q} A_0 
		&=&	-\v{S} \cdot \v{p} \times \v{E}	\\
( \gv{\Sigma} \cdot \v{q})^2 A_0	
		&=&		\Sigma_i \Sigma_j q_i q_j A_0 		\\
		&=&		 \Sigma_i \Sigma_j \partial_i E_j	\\
i\epsilon_{ijk} A_i q_j	
		&=&	-i (\v{q} \times \v{A})_k 			\\
		&=&= - B_k	\\
A_i (p_i + p'_i)  i \epsilon_{\ell jk} p_\ell q_j S_k  
		&=&	\epsilon_{\ell jk}p_\ell S_k  i(p_i + p'_i) q_j A_i		\\
		&=&	- \epsilon_{\ell jk}p_\ell S_k \{ \epsilon_{ijm}B_m (p_i + p'_i) \}			\\
		&=& -(\delta_{\ell i} \delta_{km} - \delta{\ell m} \delta_{ik})p_\ell S_k \{ \epsilon_{ijm}B_m (p_i + p'_i) \}	\\
		&=& 2\{ (\v{B} \cdot \v{p})  (\v{S} \cdot \v{p}) - (\v{B} \cdot \v{S}) \v{p}^2  \}  
\eeqa

Using these
\beqB
	ej_0 A_0 = e\phis^\dagger \left\{
					A_0 + \frac{1 - 2F_2}{8m^2}\left( 4 \v{S} \cdot \v{E} \times \v{p} + \Sigma_i \Sigma_j \partial_i E_j \right)
				\right \}
\eeqB

\beqB
	e \v{j} \cdot \v{A}	= e \phis^\dagger \left \{
			\frac{ \v{p} \cdot \v{A} }{m} + \frac{(\v{B} \cdot \v{p})  (\v{S} \cdot \v{p}) - (\v{B} \cdot \v{S}) \v{p}^2 }{m^2} 
			- F_m \left ( \frac{ \v{S} \cdot \v{B} }{m} \left\{ 1 - \frac{\v{p}^2 + \v{p'}^2}{4m^2} \right \} + \frac{(\v{B} \cdot \v{p})  (\v{S} \cdot \v{p}) - (\v{B} \cdot \v{S}) \v{p}^2 }{2m^2} \right ) \right \} \phi
\eeqB

\beqB
	=e\phis^\dagger \left\{
		 \frac{ \v{p} \cdot \v{A} }{m} + [1 -2F_m] \frac{ (\v{B} \cdot \v{p} )(\v{S} \cdot \v{p})}{m^2} 
				- \v{S} \cdot \v{B} \frac{ \v{p}^2 }{m^2} - \frac{F_m}{m} \v{S} \cdot \v{B} \right \}
\eeqB
From this it can be seen that $F_m$ is actually $g/2$, so in such terms

\beqB	
	ej_0 A_0 = e\phis^\dagger \left\{
					A_0 - \frac{g-1}{2m^2}\left( \v{S} \cdot \v{E} \times \v{p} + \frac{1}{4}\Sigma_i \Sigma_j \partial_i E_j \right)
				\right \}
\eeqB

\beqB
	e \v{j} \cdot \v{A} = e\phis^\dagger \left\{
		 \frac{ \v{p} \cdot \v{A} }{m} - [g-1] \frac{ (\v{B} \cdot \v{p} )(\v{S} \cdot \v{p})}{m^2} 
				- \v{S} \cdot \v{B} \frac{ \v{p}^2 }{m^2} - \frac{g}{2m} \v{S} \cdot \v{B} \right \}
\eeqB

The complete expression for the amplitude is
\beq \label{eq:fullScatter}
  \begin{split} M= 	e \phis^\dagger  \Bigg \{
		A_0 - \frac{g-1}{2m^2}\left( \v{S} \cdot \v{E} \times \v{p} + \frac{1}{4}\Sigma_i \Sigma_j \partial_i E_j \right)
		 \frac{ \v{p} \cdot \v{A} }{m} 
			\\ - [g-1] \frac{ (\v{B} \cdot \v{p} )(\v{S} \cdot \v{p})}{m^2} 
				- \v{S} \cdot \v{B} \frac{ \v{p}^2 }{m^2} - \frac{g}{2m} \v{S} \cdot \v{B} 
	\Bigg \} \phis
\end{split}
\eeq


\subsection{Fixing the nonrelativistic coefficients}

Having calculated the same process in both the relativistic theory and in the NRQED effective theory, the two amplitudes can be compared, thus fixing the coefficients of NRQED.

The NRQED amplitude \eqref{eq:nrqedScatter} is
\beq
\begin{split}
	iM =
		ie\phi^\dagger \Bigg( - A_0 +   \frac{ \v{A} \cdot \v{p} }{m} - \frac{  (\v{A} \cdot \v{p}) \v{p}^2   }{2m^3} 
		+ c_F  \frac{\v{S} \smalldot \v{B}} {2m}   	
		+ c_D \frac{ ( \partial_i E_i ) }{8m^2}	
		+ c_Q \frac{ Q_{ij} ( \partial_i E_j ) }{8m^2}	
	\\	+ c^{1}_S \frac{  \v{E} \times \v{p} }{4m^2}
		- (c_{W_1} -c_{W_2}) \frac{   (\v{S} \smalldot \v{B} ) \v{p}^2  }{4m^3}	
		-  c_{p'p} \frac{   (\v{S} \smalldot \v{p}) (\v{B} \smalldot \v{p})  }{4m^3} \Bigg )\phi
\end{split}
\eeq



While the relativistic amplitude was
\beq
\begin{split}
iM_{REL} = -ie \phi^\dagger \Big (
		 A_0  - \frac{\v{p}\cdot \v{A} }{m} + \frac{\v{p}\cdot \v{A} \v{p}^2}{2m^3}
		- \frac{g-1}{2m^3}\{ \grad \cdot \v{E} -  \v{S} \cdot \v{p} \times \v{E} - S_i S_j \grad_i E_j \}
		\\ - g\frac{1}{2m} \v{S} \cdot \v{B}
		+ \v{S} \cdot \v{B} \frac{\v{p}^2}{2m^3}
		+ \frac{g-2}{4m^3}(\v{S} \cdot \v{p} )( \v{B} \cdot \v{p})
	\Big ) \phi
\end{split}
\eeq


The term $\grad \cdot \v{E}  - S_i S_j \grad_i E_j$ should be rewritten using the quadrupole moment tensor $Q_{ij} = \frac{1}{2} ( S_i S_j + S_j S_i - \frac{2}{3}\v{S}^2 )$.

Remembering that $\nabla_i E_j$ is actually symmetric under exchange of $i$ and $j$, 
\[
	S_i S_j \nabla_i E_j = \frac{1}{2} (S_i S_j + S_j S_i) = (Q_{ij} + \frac{1}{3} \v{S}^2 \delta_{ij}) \nabla_i E_j
\]
\[
	= Q_{ij} \nabla_i E_j + \frac{2}{3} \grad \cdot \v{E}
\]
Written in that form, 
\beq
\begin{split}
iM_{REL} = -ie \phi^\dagger \Big (
		 A_0  - \frac{\v{p}\cdot \v{A} }{m} + \frac{\v{p}\cdot \v{A} \v{p}^2}{2m^3}
		- \frac{g-1}{2m^3}\{ \frac{1}{3}\grad \cdot \v{E} -  \v{S} \cdot \v{p} \times \v{E} - Q_{ij} \grad_i E_j \}
		\\ - g\frac{1}{2m} \v{S} \cdot \v{B}
		+ \v{S} \cdot \v{B} \frac{\v{p}^2}{2m^3}
		+ \frac{g-2}{4m^3}(\v{S} \cdot \v{p} )( \v{B} \cdot \v{p})
	\Big ) \phi
\end{split}
\eeq


Comparing the two, the coefficients are:
\beqa
	c_F &=& g \\
	c_D &=&	\frac{4(g-1)}{3}	\\
	c_Q &=&	-4(g-1)	\\
	c^1_S &=& 2 (g-1)	\\
	(c_{W_1} - c_{W_2}) &=&	2	\\
	c_{p'p}	&=& (g-2)		\\
\eeqa
