\subsection{The NRQED Lagrangian}

We want to construct an effective Lagrangian in the nonrelativistic limit.  Our goal is to calculate the leading order corrections to the $g$-factor, which are corrections of order $\alpha^2$.  To this end, we need terms in the effective nonrelativistic Lagrangian which are equivalent corrections.
% is this name really right?
\subsubsection{position space operators}

\begin{tabular}{l|c|ccc}
& Order	&	P	&	T	&	$\dagger$	\\
\hline
$eE_i$	&$m^2v^3$	&	-	& 	+	&	+		\\
$eB_i$	&$m^2v^2$	&	+	&   -	&	+		\\
$D_i$		& mv	&	-	&	+	&	-		\\
$D_0$		& mv	&	+	&	-	&	-		\\
$S_i$		& 1		&	+	&	-	&	+		\\
$i$		& 1		&	+	&	-	&	-		\\
\end{tabular}


There are several symmetries that terms in the Lagrangian must preserve: spatial reflection, time reversal, gauge invariance, and Hermiticity.  Gauge invariance is taken care of by only considering gauge-invariant operators.

All of the possible spin-space structures preserve spatial parity.  If the term in the Lagrangian is to obey such a symmetry, then any allowed collection of position-space operators must be themselves invariant under reflection.  

Each term must also be invariant under time reversal.  The spin operator $S_i$ flips sign under time reversal.  Thus, $\bar{S}_{ij}$ is even, and $\bar{S}_{ijk}$ odd, under the same transformation.  $E_i$ and $D_i$ are even, while $B_i$ is odd.  Any set of operators with definite behavior under time reversal, though, can be made invariant by including an extra factor of $i$.  So even if some term $ABC$ is odd, $iABC$ will be even.

Finally, any sequence of operators can be made Hermitian by simply adding the Hermitian conjugate; if we wish to include a term $ABC$ in the Lagrangian, we add $C^\dagger B^\dagger A^\dagger$.  Since all the operator	s under consideration are either Hermitian or anti-Hermitian, this means that exactly one of $ABC \pm CBA$ will be allowed.  The spin space operators are all Hermitian of themselves, and commute with all other operators, so only the position space operators are nontrivial.

Thus for any possible set of operators, first it is determined if such a set is allowed by considerations of parity.  Then, whether an additional factor of $i$ is needed is determined by examining the properties under time reversal.  Finally, the allowed Hermitian terms are enumerated.


%TODO add bit about order of operators as well, either before this or after it.
Of the ``building blocks'' for the general Lagrangian, those odd in parity are also odd in $v$.  And those even in parity are of an order even in $v$.  So only terms of an order even in $v$ can possibly be allowed under parity, and at the same time every such term will have the correct symmetry properties under spatial reflection.

First consider terms of leading order $mv^2$.  What collections of position space operators exist?  $E$ is already too high of an order, leaving only that with two powers of $D$:
%TODO clean up notation -- shouldn't use O here, probably
\beq
 \text{Sets of } \mathcal{O}(mv^2) \text{ combinations} = \{ \frac{1}{m} D_i D_j \} 
\eeq
This is in addition to the single permitted term containing $e\Phi$ which is not, by itself, gauge-invariant.
 
 Next consider terms of order $mv^4$.  We could have up to a single power of $E$, or only the long derivative operators:
 
\beq
 \text{Set of } \mathcal{O}(mv^2) \text{ combinations} = \{ \frac{1}{m^3} D_i D_j D_k D_\ell, \, \frac{1}{m^2} e E_i D_j \} 
\eeq

The leading order term in $B$ is just $\frac{e}{m}B_i$.

The order $\frac{e}{m} B_i v^2$ terms are drawn from:
\beq
 \text{Set of } \mathcal{O}(\frac{e}{m} B v^2) \text{ combinations} = \{ \frac{e}{m^3} D_i D_j B_k \}
\eeq


\subsubsection{Contraction of terms}

In this nonrelativistic theory the Lagrangian need not be Lorentz invariant, but must still be Galilean invariant.  All the operators we consider transform as 3-vectors or higher order 3-tensors.  To form allowed terms, all indices must be somehow contracted.

Above, all relevant sets of position space operators are considered for each order of term.  The greatest number of indices free was four.  These must be contracted with order unity structures, which as well as the spin operators with up to four indices ($S_i$, $\Sb_{ij}$, $\Sb_{ijk}$,$\Sb_{ijk\ell}$) include $\delta_{ij}$ and the completely antisymmetric tensor $\epsilon_{ijk}$.

\begin{itemize}
\item The only structure with one index is just $S_i$.

\item With two indices, there are the simple structures $\Sb_{ij}$ and $\delta_{ij}$, and also $S_i \epsilon_{ijk}$.

\item With three indices, there are $\epsilon_{ijk}$ and $\Sb_{ijk}$ as well as $\delta_{ij} S_k$ or $\Sb_{ij} \epsilon_{jk\ell}$.

\item The only position space operator with four indices is just $D_i D_j D_k D_\ell$.  While there are a fairly large number of order unity structures with four indices, any with spin operators are forbidden because of the kinetic nature of the term.  So only $\delta_{ij} \delta_{k\ell}$ need be considered.
\end{itemize}

Above are categorized the possible sets of position space operators for each order, and the ways of contracting them by number of indices.  The next step is to write all Hermitian combinations we can form from these operators.  This will form a complete catalogue of allowed terms in the Lagrangian.

%TODO address that D_i D_j \epsilon_{ijk} \sim B_k
The term $D_i D_j$ has two indices.  There are three ways to contract it: with $\delta_{ij}$, $\Sb_{ij}$, or $S_k \epsilon_{ijk}$.  Since kinetic terms with spin are not considered, that only leaves one possible term: $\v{D}^2$.  This term is Hermitian in and of itself.  It has mass dimension two, so the term as it appears in the Lagrangian will be:
\beq
	 D_i D_j  \to \frac{\v{D^2}}{2m}
\eeq 

Terms from the set $E_i D_j$ also have two indices.  It can be contracted with all three of $\delta_{ij}$, $\Sb_{ij}$, or $i S_k \epsilon_{ijk}$.   
$E_i$, $\delta_{ij}$ and $\Sb_{ij}$ are Hermitian, while $D_j$ and $i S_k \epsilon_{ijk}$ are anti-Hermitian.  It is convenient to use that $[D_j, E_i] = \grad_j E_i$.

The Hermitian combination with $\delta_{ij}$ is 
\beq
	(D_i E_j - E_j D_i) \delta_{ij} \to \frac{ \grad \cdot \v{E}  }{4m^2} = \frac{ \grad \cdot \v{E} }{4m^2}
\eeq

The Hermitian combination with $\Sb_{ij}$ is
\beq
	(D_i E_j - E_j D_i)  \Sb_{ij} \to \frac{ \Sb_{ij} \partial_i E_j }{4m^2} 
\eeq

The Hermitian combination with $i S_k \epsilon_{ijk}$ is
\beq
	(D_i E_j + E_j D_i) \to i S_k \epsilon_{ijk} \to  \frac{ i\v{S} \cdot ( \v{D} \times \v{E} + \v{E} \times \v{D} ) }{4m^2}
\eeq

A single power of the magnetic field, $B_i$ may only be contracted with $S_i$.  So the term is
\beq
	\frac{e}{m} \v{S} \cdot \v{B}
\eeq 

The second order terms involving the magnetic field are drawn from the set $D_i D_j B_k$.  There are four order-unity structures which have three indices, but only two of those need be considered.  This is because $\v{D} \times \v{D} \sim \v{B}$ and $\v{D} \times \v{B} =0$.  If is not assumed that $[D_i, B_j] = 0$ then the allowed Hermitian combinations will be with $D_i D_j B_k + B_k D_j D_i$ and $D_i B_j D_k$.

While $[D_i, D_j] \neq 0$, it does produce another term proportional to $B$ and thus, smaller than considered.  So for instance, $D_i D_j B_i$ need not be considered separate from $D_j D_i B_i$.   
\beqa
(D_i D_j B_k + B_k D_j D_i) \delta_{ij} S_k 
	&\to&	 \frac{e}{m}\frac{ \v{D}^2 (\v{S} \cdot \v{B}) + (\v{S} \cdot \v{B}) \v{D}^2 }{4m^2}	\\  
(D_i D_j B_k + B_k D_j D_i) \delta_{jk} S_i 
	&\to&	 \frac{e}{m}\frac{ (\v{S} \cdot \v{D}) (\v{D} \cdot \v{B}) + (\v{B} \cdot \v{D}) (\v{S} \cdot \v{D}) }{4m^2}	\\
D_i B_j D_k \delta_{ij} S_k  
	&\to&	 \frac{e}{m}\frac{ (\v{S} \cdot \v{D}) (\v{B} \cdot \v{D}) + (\v{D} \cdot \v{B}) (\v{S} \cdot \v{B}) }{4m^2}		\\ 
\eeqa

\beqa
(D_i D_j B_k + B_k D_j D_i) \Sb_{ijk} 
	&\to&	\frac{e}{m}\frac{ \Sb_{ijk} (D_i D_j B_k + B_k D_j D_i)   }{4m^2}	\\
D_i B_j D_k \Sb_{ijk} 
	&\to&	\frac{e}{m}\frac{ \Sb_{ijk} D_i B_j D_k  }{4m^2}	\\  
\eeqa

Finally, there is only one way to contract $D_i D_j D_k D_\ell$, which is as $\v{D}^4$.  Again, variations such as $D_i \v{D}^2 D_i$ need not be considered because they just reproduce already considered terms with $B$.
\beqa
	D_i D_j D_k D_\ell \delta_{ij} \delta_{k \ell} 
		&\to& \frac{\v{D}^4}{8m^3}
\eeqa
   
   
\beq \label{eq:nrLFirstOrder}
	\mathcal{L}_{NRQED} = \fnrb \Bigg\{ iD_0 +  \frac{\v{D}^2}{2m}  +  c_F \frac{e}{m} \v{S} \cdot \v{B}\Bigg \} \fnr
\eeq 

%%%%%%%% Lagrangian
\beq \label{eq:nrLv4}
	\mathcal{L}_{mv^4} = \fnrb \Bigg\{
		\frac{\v{D}^4}{8m^2}
		+ c_D \frac{e (\v{D} \cdot \v{E} - \v{E} \cdot \v{D})}{8m^2} 
		+ c_Q \frac{eQ_{ij}(D_i E_j - E_i D_j)}{8m^2}
		+ c_S \frac{ i e \v{S} \cdot(\v{D} \times \v{E} - \v{E} \times \v{D}}{8m^2} \Bigg \} \fnr
\eeq


\beq \label{eq:nrLBv2} \begin{split}
	\mathcal{L}_{Bv^2} = &
		\fnrb \Bigg\{
			c_{W1} \frac{ e \v{D}^2 \v{S} \cdot \v{B} + \v{S} \cdot \v{B} \v{D}^2 }{8m^3}
			- c_{W2} \frac{e D_i (\v{S} \cdot \v{B}) D_i}{4m^3}
			+c_{p'p} \frac{ e [ (\v{S}\cdot \v{D})(\v{B} \cdot \v{D}) + (\v{B} \cdot \v{D})(\v{S}\cdot \v{D})]}{8m^3}
\\ &		+ c_{T_1} \frac{ e \bar{S}_{ijk} (D_i D_j B_k + B_k D_j D_i)}{8m^3}
		+ c_{T_2} \frac{ e \bar{S}_{ijk} D_i B_j D_k }{8m^3} \Bigg \} \fnr
\end{split}\eeq

\subsubsection{Full Lagrangian}
The full Lagrangian we consider is then:

\beq \label{eq:nrLFull}
\begin{split}
\mathcal{L}_{NRQED} = & \fnrb \Bigg\{
		iD_0 +  \frac{\v{D}^2}{2m}  + 	\frac{\v{D}^4}{8m^2}
		 + c_F \frac{e}{m} \v{S} \cdot \v{B}
		+ c_D \frac{e (\v{D} \cdot \v{E} - \v{E} \cdot \v{D})}{8m^2} 
		+ c_Q \frac{eQ_{ij}(D_i E_j - E_i D_j)}{8m^2}
\\	& + c_S \frac{ i e \v{S} \cdot(\v{D} \times \v{E} - \v{E} \times \v{D}}{8m^2}
		+ c_{W1} \frac{ e \v{D}^2 \v{S} \cdot \v{B} + \v{S} \cdot \v{B} \v{D}^2 }{8m^3}
		- c_{W2} \frac{e D_i (\v{S} \cdot \v{B}) D_i}{4m^3}
\\	&		+c_{p'p} \frac{ e [ (\v{S}\cdot \v{D})(\v{B} \cdot \v{D}) + (\v{B} \cdot \v{D})(\v{S}\cdot \v{D})]}{8m^3}
 	+ c_{T_1} \frac{ e \bar{S}_{ijk} (D_i D_j B_k + B_k D_j D_i)}{8m^3}
		+ c_{T_2} \frac{ e \bar{S}_{ijk} D_i B_j D_k }{8m^3} 
		\Bigg \} \fnr
\end{split}
\eeq


One of the features of this Lagrangian is that every coefficient is fixed by the one-photon interaction.  Although some terms might represent two-photon interactions, they are terms like $\v{S} \cdot \v{A} \times \v{E}$, whose coefficient is fixed by the gauge-invariant term $\v{S} \cdot \v{D} \times \v{E}$.  This in turn means that we can calculate the corrections to the $g$-factor by considering only one-photon interactions.

%%%%%%  OLD OLD OLD
\subsubsection{Order of terms}
%TODO Discussion of energy scales -- possibly move this into introduction, since it applies to each calculation
We consider constant, infinitesimal external magnetic fields, so we need only consider terms linear in $\v{B}$.

The velocity of the particles in our bound state system will be $v \sim \alpha$.

The electric field we consider is the Coulomb field, so $e\Phi \sim m Z\alpha^2 \sim mv^2$, and $eE \sim m^2v^3$.

Each derivative of the electric field will add an additional factor of $mv$, so the operator $\v{D}$ can be taken to be of this order.

We need to keep terms up to order $mv^4$ and $\frac{B}{m} v^2$ in order to calculate the $g$-factor to the necessary precision.  We include $mv^4$ terms so we can be sure that there are no effects entering from second-order perturbation theory.


%TODO insert qualifications on external field
\subsubsection{Constraints on the form of the Lagrangian}
The Lagrangian is constrained to obey several symmetries.  It must be invariant under the symmetries of parity and time reversal.  It must also be invariant under Galilean transformations.  The Lagrangian must also be Hermitian, and gauge invariant.

What are the gauge invariant building blocks we can use to construct this Lagrangian?  We have the external fields $\v{E}$ and $\v{B}$, the spin operators $\v{S}$, and the long derivative $\v{D} = \v{\partial} - i e\v{A}$.  The fields should always be accompanied by the charge $e$ of the particle.

When considering the case of higher spin particles, we might consider terms quadratic and above in spin operators.  For a particle of spin $s$, there must be $(2s+1)^2$ independent hermitian operators.  We can span this set of operators by considering products of up to $2s$ spin matrices which are symmetric and traceless in every vector index.  For example, for spin-$1$ we have quadratic, in addition to $I$ and $S_i$, five independent structures of the form $ S_i S_j + S_j S_i + \delta_{ij} \v{S}$.




We also have the scalar $D_0$, however, we need only include a single such term because we insist on having only one power of the time derivative.

To consider possible terms, we need to know how each of the above behave under the discrete transformations and Hermitian conjugate.  The signs under these transformations are listed in the table below.  (Also included is the imaginary number $i$.)


\begin{tabular}{l|c|ccc}
& Order	&	P	&	T	&	$\dagger$	\\
\hline
$eE_i$	&$m^2v^3$	&	-	& 	+	&	+		\\
$eB_i$	&$m^2v^2$	&	+	&   -	&	+		\\
$D_i$		& mv	&	-	&	+	&	-		\\
$D_0$		& mv	&	+	&	-	&	-		\\
$S_i$		& 1		&	+	&	-	&	+		\\
$i$		& 1		&	+	&	-	&	-		\\
\end{tabular}

\subsection{Properties of spin operators}
%TODO in the context of a field theory, is talking about the state-space the correct way of thinking about it?
In formulating NRQED for general spin particles, we need to consider all the possible operators might show up in the Lagrangian.  The state-space of a spin-$s$ particle is the direct product of its spin-state and all the other state information.  Because the spaces are orthogonal, we can treat separately operators in the two spaces.  The operators and fields which exist in position space are the same for a particle of any spin, but unsurpsingly the operators allowed in spin space do depend upon the spin of the particle.  As the spin of the particle is increased, and thus its spin degrees of freedom rise, there are more ways to mix these components, and thus a greater number of spin operators to consider.

For a particular representation, we can always write a bilinear as the spin operators and other operators acting between two spinors:
\beq
	\Psi^\dagger \mathcal{O}_S \mathcal{O}_X \Psi
\eeq

The two types of operators will always commute, since they act on orthogonal spaces, so it doesn't matter what order they're written in.  All such bilinears must be Galilean invariant, but individual operators might not be.  The non-spin operators we consider, such as $\v{D}$, $\v{B}$ or contractions with the tensor $\epsilon_{ijk}$ are already all written as 3-vectors or (in combination) as higher rank tensors.  Therefore, it will be most convenient to write spin-operators with well defined properties under Galilean transformations.  In that way, writing Galilean-invariant combinations of the two types of operators is done just by contracting indices. 

Even though the number of spin operators does depend upon the spin of the particle, it is still possible to proceed in such a way that the same notation may be used no matter the spin.  There are a few requirements:
\begin{itemize}
  \item We write all high spin operators in terms of combinations of $S_i$, since these have universal properties regardless of the representation they are written in.
  \item If an operator exists and is non-zero in the representation of spin-$s$, it also exists in spin-$s+1$
  \item All operators introduced to account for the additional degrees of freedom in higher spin representations vanish when written in a lower spin theory.  (As an example of the last point, the operator $S_i S_j + S_j S_i - \delta_{ij} S^2$ is needed to account for the degrees of freedom in a spin-1 theory, but vanishes in spin-1/2.)
\end{itemize}
If these requirements are met a consistent spin-agnostic notation can be adopted.  Now we attempt to construct operators that meet these conditions.

The spinors $\Psi$ are written with $2s+1$ independent components.  The spin operators will be isomorphic to matrices acting on these components, which for a spin-$s$ particle would be $(2s+1) \times (2s+1)$ matrices.  The combined operator $\mathcal{O}_S \mathcal{O}_X$ must be Hermitian, but without loss of generality we can require any $\mathcal{O}_S$, $\mathcal{O}_X$ to be Hermitian separately.  So there is the additional constraint that these matrices be Hermitian, and this means a total of $(2s+1)^2$ degrees of freedom.
%TODO: could include derivation of degrees of freedom?


For spin-$0$ there is only one component to the spinor, so the only possible operator is equivalent to the identity.

For spin-$1/2$ we have, in addition to the identity, the spin matrices $s_i = \frac{1}{2} \sigma_i$.  This is a set of four independent matrices, and since the space has $(2s+1)^2 = 4$ degrees of freedom, exactly spans the space of all spin-operators.  If we try to construct terms which are bilinear in spin matrices, they just reduce through the identity $\sigma_i \sigma_j = \delta_{ij} + \epsilon_{ijk}\sigma_k$, which we can already construct through combinations of the four operators we already have.  Since those four operators form a basis for the space, independent bilinears were forbidden even without an explicit form for the equation.


What about spin-$1$?  We need 9 independent operators to span the space.  All the operators that exist in spin-$1/2$ will work here as well, though the spin matrices will have a different representation.  That leaves 5 operators to construct.  It is natural to try to construct these from bilinear combinations of spin matrices.  Naively $S_i S_j$ would itself be 9 independent structures, but clearly some of these are expressible in terms of the lower order operators.  (By the order of a spin operator we mean its greatest degree in $S_i$ )  %TODO multipole moment language better?

Regardless of their representation, the spin operators always fulfill certain identities based on their Lie group.  Namely
\beq
	S_i S_j \delta_{ij} \sim I, \; [S_i, S_j] = \epsilon_{ijk} S_k
\eeq
and it is these identities which allow certain combinations of $S_i S_j$ to be related to lower order operators.

If instead of general spin bilinears we consider only combinations which are
\begin{itemize}
  \item Symmetric in $i$, $j$ 
  \item Traceless
\end{itemize}
then such a structure will be independent of the set of operators $\{I, S_1, S_2, S_3\}$.  Because it is symmetric no combination may be related using the commutator, and because it is traceless there is no combination that reduces due to the other identity.  %TODO Name of id involving Killing operator \delta_{ij}?

This conditions form a set of 4 constraints, so from the original 9 degrees of freedom possessed by combinations of $S_i S_j$ are left only 5.  Together with the 4 lower order operators this is exactly enough to span the space.

We can explicitly write this symmetric, traceless structure as
\beq
	 S_i S_j + S_j S_i - \frac{2}{3} \delta_{ij} S^2
\eeq

Having explored how the procedure works for spin-1, move on to consider the general spin case.   The idea is to proceed inductively using the same rough attack as for the case of spin-$1$.  In addition to all the ``lower order'' operators which were used for lower spin representations introduce new operators which are of higher degree in the spin matrices and guaranteed to be independent of the lower spin operators.

So suppose that for a spin-$s-1$ particle we have a set of operators written as $\bar{S}^0$, $\bar{S}^1$ $\ldots \bar{S}^{(s-1)}$, where a structure $\bar{S}^n$ carries $n$ Galilean indices and is symmetric and traceless between any pair of indices, that is:
\beq
	\bar{S}^n_{..i..j..} = \bar{S}^n_{..j..i..}, \; \delta_{ij} \bar{S}^n_{..i..j..}=0
\eeq 
(From above, $\bar{S}^0=I$, $\bar{S}^1_i = S_i$, and $\bar{S}^2 = S_i S_j + S_j S_i - \frac{2}{3}\delta_{ij} S^2$.) 

The objects $\bar{S}^n$ are built as follows: start with all combinations involving the product of exactly $n$ spin matrices.  (There are $3^n$ such structures.)  Form them into combinations which are symmetric in all indices.  Each index has three possible values, so we can label each structure by how many indices are equal to 1 and 2.  If $a$ is the number of indices equal to $1$, and $b$ the number of indices equal to $2$, then for a given $a$ there are $n+1-a$ possible choices for $b$.  The total number of symmetric structures is then
\beq
	\sum^n_{a=0} (n-a+1) = \frac{1}{2} (n+1) (n+2)
\eeq
We want to apply the additional constraint that the $\bar{S}^n$ be traceless in all indices.  This will involve subtracting all the lower order structures which result when the trace of the completely symmetric combinations is taken.  %TODO explicate {\it exactly} how this is accomplished.

It introduces an additional constraint on $\Sb^n$ for each pair of indices, and there are $n (n-1)/2$ distinct pairs of indices.  The total degrees of freedom left are

\beq
	\frac{1}{2} (n+1) (n+2) - \frac{1}{2} n(n-1) 
		= \frac{1}{2}\left( n^2+3n +2 - n^2 +n\right )
		= 2n+1
\eeq

%TODO expand upon going from spin s particle to spin s-1
In combination with the lower order spin operators, this is exactly the number of independent operators we need to span the space.  Combined with the lower order operators this is a complete basis, so we know we haven't missed any terms.  Because they are constructed to be independent from all the lower order operators, it must necessarily be true that they will vanish in lower spin representations.

Using this notation we can write down terms in the Lagrangian that are valid for particles of any spin.  By writing all spin operators in terms of $S_i$ they are representation agnostic, and by construction they will vanish for low spin particles where they do not ``fit''.






