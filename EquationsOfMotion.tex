
\section{Spin-1 equations of motion}
\subsection{Derivation of relativistic Hamiltonian}

We want to derive a "Schrodinger-like" equation governing the nonrelativistic motion of the vector particle; an equation of the form $i\partial_t \Psi = \hat{H} \Psi$.  Since it is a vector particle, $\Psi$ is a spinor with three components.  We will first find such an equation for a bispinor in the relativistic theory, having the form:
\[
i\partial_t 
\begin{pmatrix}
	\Psi_u	\\	\Psi_l
\end{pmatrix} 
=
\hat{H}
\begin{pmatrix}
	\Psi_u	\\	\Psi_l
\end{pmatrix}
\]

If we have a Lagrangian describing the interaction, such an equation can be obtained from the Euler-Lagrange equations.  


We have such a Lagrangian:
\begin{eqnarray*}
\mathcal{L} 
	&=&	-\frac{1}{2} (D \times W)^\dagger \cdot (D \times W) 
				+ m_w^2 W^\dagger W 
				- i e {W^\dagger}^\mu W^\nu F_{\mu \nu}	\\
\end{eqnarray*}

Where
\begin{eqnarray*}
		D^\mu	=	\partial^\mu - i e A^\mu ,
	&&
		D \times W = D^\mu W^\nu - D^\nu W^\mu
\end{eqnarray*}
The last term in the Lagrangian, $-ie{W^\dagger}^\mu W^\nu F_{\mu\nu}$, is needed to produce the "natural" magnetic moment of g=2.  We can introduce an anomalous magnetic moment by changing the coefficient of that term, from 1 to $1+(g-2)$.  For now call this coefficient $\lambda$.  Now
\begin{eqnarray*}
\mathcal{L} 
	&=&	-\frac{1}{2} (D \times W)^\dagger \cdot (D \times W) 
				+ m_w^2 W^\dagger W 
				- \lambda i e  {W^\dagger}^\mu W^\nu F_{\mu \nu}	\\
\end{eqnarray*}


We can find the equations of motion with the Euler-Lagrange method.
\begin{eqnarray*} \pd{ \mathcal{L}}{{W^\dagger}^\alpha} - \frac{d}{dt} \frac{\partial \mathcal{L}} {\partial (\partial_t {W^\dagger}^ \alpha)} 
	&=& 0	\\
\frac{ \partial}{\partial  {W^\dagger}^\alpha} (D \times W)^\dagger_{\mu \nu} (D \times W)^{\mu \nu}
	&=&	\frac{ \partial}{\partial  {W^\dagger}^\alpha}
				(D_\mu W_\nu - D_\nu W_\mu)^\dagger(D^\mu W^\nu - D^\nu W^\mu) \\
	&=&	\frac{ \partial}{\partial  {W^\dagger}^\alpha}
				2(D_\mu W_\nu)^\dagger(D^\mu W^\nu - D^\nu W^\mu) \\
	&=& 	\frac{ \partial}{\partial  {W^\dagger}^\alpha}
				2W^\dagger_\mu D_\nu (D^\mu W^\nu - D^\nu W^\mu) \\
	&=& 	2g_\mu^\alpha D_\nu(D^\mu W^\nu - D^\nu W^\mu) \\
	&=& 	2 D_\nu(D^\alpha W^\nu - D^\nu W^\alpha) \\ 
\frac{ \partial}{\partial  {W^\dagger}^\alpha}  m_W^2 {W^\dagger}^\mu W_\mu
	&=&	m_W^2 W_\alpha	\\
\frac{ \partial}{\partial  {W^\dagger}^\alpha} \lambda i e {W^\dagger}^\mu W^\nu F_{\mu \nu}s
	&=&	\lambda i e g^{\alpha \mu} W^\nu F_{\mu \nu}	\\
	&=&	- \lambda i e W_\nu F^{\nu \mu}
\end {eqnarray*}


To obtain a coupled set of first order equations, we introduce the fields $W^{\mu \nu} = D^\mu W^\nu - D^\nu W^\mu$.
\begin{eqnarray}
	D_\mu W^{\mu \nu} + m_W^2 W^\nu + \lambda i e W_\mu F^{\mu \nu} = 0	\\
	W^{\mu \nu} = D^\mu W^\nu - D^\nu W^\mu	
\end{eqnarray}
$W^{\mu\nu}$ is antisymmetric and so has six degrees of freedom, corresponding to six independent fields.  Together with $W^\mu$ this represents a total of ten fields.  However, upon examination only some of these fields are dynamic.  The  fields $W^{0i}$ and $W^{i}$ appear in the equations with time derivatives, while the fields $W^{ij}$ and $W^0$ never do.  So it is only necessary to consider the former six fields.  So that these six fields all have the same dimension, we will define $\frac{W^{i0}}{m} = i \eta^i$.

We will now eliminate the extraneous fields and solve for $iD_0 W^i$, $iD_0 \eta^i$.

Consider (1) with $\nu=0$:

\begin{equation*}
-D^i W ^{i0} + m^2 W^0 - \lambda i e W^{i} F^{i0} = 0
\end{equation*}
or equivalently, using $F^{i0} = -E^i$:
\begin{eqnarray*}
W^0	&=&	\frac{1}{m} \left [ D^i \frac{W^{i0}}{m} + \lambda ie \frac{\v{W}}{m} \cdot \v{E}  \right ]	\\
	&=&	\frac{i}{m} \left [ \v{D} \cdot \gv{\eta} + \lambda ie \frac{\v{W}}{m} \cdot \v{E}  \right ]
\end{eqnarray*}

Now consider (1) with $\nu=i$:
\begin{equation*}
D^0 W^{0i} - D^j W^{ji} + m^2 W^{i} + \lambda ieW^0 F^{0i} - \lambda ieW^{j} F^{ji} = 0
\end{equation*}
Using the definition of $W^{ij}$, the previously defined $\eta^i$, and that $F^{ij} = \epsilon_{ijk} B_k$
\begin{equation*}
-i D^0 m \eta^i - D^j (D^j W^i - D^i W^j) + m^2 W^i - \lambda ieW^0 E^i  - \lambda ieW^{j} \epsilon_{jik} B_k = 0
\end{equation*}

This gives
\begin{equation} \label{eq:eta1}
iD^0 \eta^i = \frac{1}{m} \left (\v{D} \times (\v{D} \times \v {W}) \right)^i + m W^i + \lambda\frac{e}{m^2} E^i \left( \v{D} \cdot \gv{\eta} + \lambda \frac{e}{m} \v{W} \cdot \v{E} \right) + \lambda \frac{ie}{m} (\v{W} \times \v{B} )^i
\end{equation}

Now, look at (2) with $\nu=0, \mu=i$:

\begin{eqnarray*}
W^{i0} 
	&=& 	D^i W^0 - D^0 W^i		\\
	&=&	 \frac{i}{m} D^i \left [ \v{D} \cdot \gv{\eta} + \lambda ie \frac{\v{W}}{m} \cdot \v{E}  \right ] - D^0 W^i
\end{eqnarray*}

This yields an equation for $iD^0 W^i$:
\begin{equation}\label{eq:W1}
i D^0 W^i = 
	m \eta^i - \frac{1}{m} D^i \left [ \v{D} \cdot \gv{\eta} + \lambda ie \frac{\v{W}}{m} \cdot \v{E}  \right ]
\end{equation}

We have two equations for $i D_0 W^i$ and $i D_0 \eta^i$.  But we need these equations written so that each term should be written as an operator acting on either $\v{W}$ or $\gv{\eta}$. For this purpose, we introduce into the equations the spin matrices $S_i$.  Then, using the identities shown in the appendix, we can rewrite \eqref{eq:eta1} it as:

\begin{equation}
\begin{split}
iD^0 \gv{\eta} =&	-\frac{1}{m} (\v{S} \cdot \v{D})^2 \v{W} 
								+ m\v{W} 
								+ \lambda \frac{e}{m^2} \left [ \v{E} \cdot \v{D} - (\v{S} \cdot \v{E})( \v{S} \cdot \v{D}) + i \v{S} \cdot \v{E} \times \v{D} \right ] \gv{\eta} 	\\
							&	- \lambda \frac{e}{m} \v{S} \cdot \v{B} \v{W}
								+ \lambda^2 \frac{e^2}{m^3} \left [ \v{E}^2 - (\v{S} \cdot \v{E})^2 \right ] \v{W}
\end{split}
\end{equation}

Likewise, write \eqref{eq:W1} as:
\begin{equation}
iD^0 \v{W} =	m\gv{\eta}
		- \frac{1}{m}\left [ \v{D}^2 - (\v{S} \cdot \v{D})^2 + e \v{S} \cdot \v{B} \right ] \v{\eta}
		- \lambda \frac{e}{m^2} \left [ \v{D} \cdot \v{E} - (\v{S} \cdot \v{D}) (\v{S} \cdot \v{E}) + i \v{S} \cdot \v{D} \times \v{E} \right ] \v{W}
\end{equation}

Now, these two equations can be written together in matrix form, from which the Hamiltonian can be easily obtained.
\begin{equation*}
iD_0
\begin{pmatrix}
\eta	\\	W
\end{pmatrix} 
=
\begin{pmatrix}
	\lambda \frac{e}{m^2} \left [ \v{E} \cdot \v{D} - (\v{S} \cdot \v{E})( \v{S} \cdot \v{D}) + i \v{S} \cdot \v{E} \times \v{D} \right ]
&
	m
	-\frac{1}{m} (\v{S} \cdot \v{D})^2 
	- \lambda \frac{e}{m} \v{S} \cdot \v{B}
	+ \lambda^2 \frac{e^2}{m^3} \left [ \v{E}^2 - (\v{S} \cdot \v{E})^2 \right ] 
\\
	m
	- \frac{1}{m}\left [ \v{D}^2 - (\v{S} \cdot \v{D})^2 + e \v{S} \cdot \v{B} \right ] 
&
	- \lambda \frac{e}{m^2} \left [ \v{D} \cdot \v{E} - (\v{S} \cdot \v{D}) (\v{S} \cdot \v{E}) + i \v{S} \cdot \v{D} \times \v{E} \right ] 
\end{pmatrix}
\begin{pmatrix}
\eta	\\	W
\end{pmatrix}
\end{equation*}

\subsubsection*{Current}

We can also derive the conserved current from the Lagrangian:
\begin{eqnarray*}
\mathcal{L} 
	&=&	-\frac{1}{2} (D \times W)^\dagger \cdot (D \times W) 
				+ m_w^2 W^\dagger W 
				-\lambda i e {W^\dagger}^\mu W^\nu F_{\mu \nu}	\\
\end{eqnarray*}

Where
\begin{eqnarray*}
		D^\mu	=	\partial^\mu - i e A^\mu ,
	&&
		D \times W = D^\mu W^\nu - D^\nu W^\mu
\end{eqnarray*}


We want the conserved current corresponding to the transformation $ W_i \to e^{i \alpha}W_i $, which in infinitesimal form is:
\begin{equation*}
	W_\mu \to W_\mu + i \alpha W_\mu, \;
	W_\mu^\dagger \to W_\mu^\dagger - i \alpha W_\mu^{\dagger}
\end{equation*}

The 4-current density will be:
\begin{equation*}
j^{\sigma} = -i \pd {\mathcal{L} }{W_{\mu, \sigma}} W_\mu  +  i\pd {\mathcal{L} }{W^\dagger_{\mu, \sigma}} W^\dagger_\mu
\end{equation*}

Only one term contains derivatives of the field:

\begin{eqnarray*}
 \pd {\mathcal{L} }{W_{\alpha, \sigma}} 
		&=& \pd{}{W_{\alpha, \sigma}} \left \{ - \frac{1}{2} (D_\mu W_\nu - D_\nu W_\mu)^\dagger(D^\mu W^\nu - D^\nu W^\mu) \right \}\\
		&=& - \frac{1}{2} (D_\mu W_\nu - D_\nu W_\mu)^\dagger (g_{\sigma \mu} g_{\alpha \nu} - g_{\sigma \nu}g_{\alpha \mu})\\
		&=& - (D_\alpha W_\sigma - D_\sigma W_\alpha)^\dagger
\end{eqnarray*}
Likewise:
\begin{eqnarray*}
	\pd {\mathcal{L} }{W_{\alpha, \sigma}^\dagger} 
		&=& - (D_\alpha W_\sigma - D_\sigma W_\alpha)
\end{eqnarray*}

If  we define $W_{\mu \nu} =  D^\mu W^\nu - D^\nu W^\mu$ then the 4-current and charge density are:

\begin{eqnarray*}
	j_\sigma &=& i W_{\sigma \mu}^\dagger W^{\mu} - i W_{\sigma \mu} {W^{\dagger}}^\mu \\
	j_0 	&=& i W_{0 \mu}^\dagger W^{\mu} - i W_{0 \mu} {W^{\dagger}}^\mu \\
		&=& i W_{0 i}^\dagger W^i - i W_{0 i} {W^{\dagger}}^i \\
\end{eqnarray*}
Where the last equality follows from the antisymmetry of $W_{\mu \nu}$.

Now, we defined the fields $\eta_i = -i \frac{W_{i0}}{m}$.  In terms of these fields,
$j_0 =  m (\eta_i^\dagger  W^i + \eta_i {W^\dagger}^i )$. 

We can do the same to find the vector part of the current.
\begin{eqnarray*}
	j_i &=& i W_{i \mu}^\dagger W^{\mu} - i W_{i \mu} {W^{\dagger}}^\mu 	\\
	&=&	i W_j^\dagger W_{ij}  + i W_{i0}^\dagger W_0 + c.c.
\end{eqnarray*}

We have $W_{ij} = D_i W_j - D_j W_i$.  Using the identities developed in the appendix, we can obtain
\[ D_j W_i = D_i W_j - D_k(S_i S_k)_{ja} W_a 	\]
Then
\[ W_{ij} = D_k (S_i S_k)_{ja} W_a 	\]

In the absence of an electric field E, $W_{0} = \frac{i}{m} D_j \eta_j$, with $W_{i0} = i m \eta_i$.
\[	W_{i0}^\dagger W_0 = - \eta_i^\dagger D_j \eta_j 	\]
Again we introduce spin matrices to get the equation in the desired form, and obtain
\[	W_{i0}^\dagger W_0 = - \eta_j^\dagger D_k (\delta_{ik} - S_k S_i) \eta_j 	\]

This leads to
\[ j_i = i W_j^\dagger D_k (S_i S_k W)_j - i \eta_j^\dagger D_k ([\delta_{ik} - S_k S_i]\eta)_j + c.c. \]

Writing this in terms of the bispinor $\begin{pmatrix}\eta \\ W\end{pmatrix}$, the expression for the current is

\begin{equation}	j_i	=
		\frac{i}{2} \begin{pmatrix}\eta^\dagger && W^\dagger \end{pmatrix} \left [
		(\{S_i, S_j\} - \delta_{ij})  
		\begin{pmatrix} 
			1 & 0 \\
			0 & 1 \\ 
		\end{pmatrix}
		- ([S_i, S_j] +\delta_{ij})	\begin{pmatrix} 1 & 0 \\ 0 & -1 \\ \end{pmatrix}
		\right ]
		D_j \begin{pmatrix}\eta \\ W\end{pmatrix} + c.c.
\end{equation}


\subsubsection*{Hermiticity of Hamiltonian}


It's clear from inspection that the above Hamiltonian is not Hermitian in the standard sense.  (Of the operators in use, the only one which is not self adjoint is $D_i^\dagger = - D_i$.)  Noticing that 
\[
	\left [ \v{E} \cdot \v{D} - (\v{S} \cdot \v{E})( \v{S} \cdot \v{D}) + i \v{S} \cdot \v{E} \times \v{D} \right]^\dagger 
		= - \left [ \v{D} \cdot \v{E} - (\v{S} \cdot \v{D}) (\v{S} \cdot \v{E}) + i \v{S} \cdot \v{D} \times \v{E} \right ] 
\]
we can see it has the general form 
\[
	H = 
\begin{pmatrix}
	A	&	B	\\
	C	&	A^\dagger
\end{pmatrix}
\]
where the off diagonal blocks are Hermitian in the normal sense: $B^\dagger = B$ and $C^\dagger=C$.

At this point we should consider that an operator is defined as Hermitian with respect to a particular inner product.  In quantum mechanics this inner product is normally defined as:
\[	<\Psi, \phi> = \int d^3x \Psi^\dagger \phi	\]
This definition, however, does not produce sensible results for the states in question.  We need the inner product of a state with itself to be conserved; in other words, it should play the role of a conserved charge.  From the considerations above we already have one such quantity: the conserved charge $\int d^3x j_0$, where 

\[	
	j_0 = m [\eta^\dagger W + W^\dagger \eta] 
		= 	m \begin{pmatrix} \eta^\dagger & W^\dagger \end{pmatrix}
			\begin{pmatrix} 0 & 1 \\ 1 & 0 \end{pmatrix}
			\begin{pmatrix} \eta \\ W \end{pmatrix}
\]

A more general definition of the inner product includes some weight M: 
\[	<\Psi, \phi> = \int d^3x \Psi^\dagger M \phi	\]
In normal quantum mechanics M would be the identity matrix, but here, as implied by the charge density, we want $M=\begin{pmatrix} 0 & 1 \\ 1 & 0 \end{pmatrix}$.  Such a definition will lead to the inner product $<\Psi, \Psi>$ being conserved.

An operator H is hermitian with respect to this inner product if
\[ <H\Psi, \phi> = <\Psi, H\phi> \to
		\int d^3x \Psi^\dagger H^\dagger M \phi	=	\int d^3x \Psi^\dagger M H \phi
\]
For this equality to hold, it is sufficient for $H^\dagger M = M H$.  With $M=\begin{pmatrix} 0 & 1 \\ 1 & 0 \end{pmatrix}$ and $H=\begin{pmatrix} A & B \\ C & D \end{pmatrix}$ this condition reduces to 
\[
	\begin{pmatrix} A^\dagger & C^\dagger \\ B^\dagger & D^\dagger \end{pmatrix}
	=\begin{pmatrix} D & C \\ B & A \end{pmatrix}
\]
Our Hamiltonian fulfills exactly this requirement, and so is Hermitian with respect to this particular inner product.




\subsection{Non-relativistic Hamiltonian}
Now we will consider the non-relativistic limit of the above Hamiltonian.  To work in this regime constrains the order of both the momentum and the electromagnetic field strength.
\begin{eqnarray*}
	D	&\sim&	mv	\\
	\Phi	&\sim&	mv^2	\\
	E	&\sim&	m^2v^3	\\
	B	&\sim&	m^2v^2
\end{eqnarray*}

We can write the Hamiltonian matrix in terms of the basis of 2x2 Hermitian matrices: $({\bf I}, \rho_i)$:

\begin{eqnarray*}
H &=&	a_0 {\bf I} + a_i \rho_i \\
a_0  	&=& 
 \frac{1}{2}(H_{11}+H_{22}) =
			e\Phi + 
				\lambda \frac{e}{2m^2} 
				\left [ 
					\v{E} \cdot \v{D} - (\v{S} \cdot \v{E})( \v{S} \cdot \v{D}) + i \v{S} \cdot \v{E} \times \v{D}
					- \v{D} \cdot \v{E} + (\v{S} \cdot \v{D}) (\v{S} \cdot \v{E}) - i \v{S} \cdot \v{D} \times \v{E} 
				\right ]\\
a_3 	&=&
 \frac{1}{2}(H_{11}-H_{22}) =
				\lambda s\frac{e}{2m^2} 	\left [ 
					\v{E} \cdot \v{D} - (\v{S} \cdot \v{E})( \v{S} \cdot \v{D}) + i \v{S} \cdot \v{E} \times \v{D}
					+ \v{D} \cdot \v{E} - (\v{S} \cdot \v{D}) (\v{S} \cdot \v{E}) + i \v{S} \cdot \v{D} \times \v{E} 
				\right ]\\
ia_2	&=& 
 \frac{1}{2}(H_{21}-H_{12}) =
				-\left [	
					\frac{\v{D}^2}{2m} - \frac{1}{m} (\v{S} \cdot \v{D})^2 
					-\frac{\lambda-1}{2} \frac{e}{m} \v{S} \cdot \v{B}
					+ \frac{e^2}{2m^3}(\v{E}^2 -(\v{S}\cdot\v{E})^2)
				\right ]	\\
a_1		&=&	
 \frac{1}{2}(H_{12}-H_{21}) =
				\left [
					m - \frac{\v{D}^2}{2m} - \frac{1+\lambda}{2}\frac{e}{m}\v{S} \cdot \v{B}
					+ \frac{e^2}{2m^3}(\v{E}^2 -(\v{S}\cdot\v{E})^2)
				\right ]
\end{eqnarray*}

We can see that to leading order, the Hamiltonian is
\begin{equation*}
H = m \rho_1 = 
\begin{pmatrix}
0	&	m \\
m	&	0	\\
\end{pmatrix}
\end{equation*}
Since we wish to separate positive and negative energy states, this poses a problem.  So we first switch to a basis where at least the rest energies are separate.  Then, remaining off-diagonal elements can be treated as perturbations.

An appropriate transformation which meets our requirements is a "rotation" in the space spanned by $\rho_i$ matrices, about the $\rho_1 + \rho_3$ axis.  This has the explicit form:
\begin{equation}
U =\frac{1}{\sqrt {2}}  (\rho_1 + \rho_3)	
	= \frac{1}{\sqrt {2}}
\begin{pmatrix}
1	&	1	\\
1	&	-1	\\
\end{pmatrix}
\end{equation}
Any transformation U can be described by it's action on the basis of 4x4 matrices.   This takes:
\begin{eqnarray*}
{\bf I} & \to &	{\bf I}	\\
\rho_1	& \to &	\rho_3	\\
\rho_3	& \to &	\rho_1	\\
\rho_2	& \to &	-\rho_2	\\
\begin{pmatrix}
	\eta	\\	W
\end{pmatrix}
&	\to	&
\begin{pmatrix}
\Psi_u \\	\Psi_\ell
\end{pmatrix}
=	\frac{1}{\sqrt{2}}
\begin{pmatrix}
\eta	+ W \\	\eta - W
\end{pmatrix}
\end{eqnarray*}

(This transformation will transform the current to
\[	j_0 =  m (\Psi_u^\dagger \Psi_u  - \Psi_\ell^\dagger \Psi_\ell)		\]
and the weight M, used in the inner product, to
\[	M \to M' = U^\dagger M U = \begin{pmatrix} 1 & 0 \\ 0 & -1 \end{pmatrix}	\]
The transformed Hamiltonian will of course be Hermitian with respect to the transformed inner product.)

Our equation is now of the following form:
\begin{eqnarray*}
i\partial_0 \begin{pmatrix} \Psi_u \\	\Psi_\ell \end{pmatrix}  
	& =&
	H' \begin{pmatrix} \Psi_u \\	\Psi_\ell  \end{pmatrix}
\end{eqnarray*}

We can see that the Hamiltonian still contains off-diagonal elements, so this represents a pair of coupled equations for the upper and lower components of $\gv{\Psi}$.  But the off-diagonal terms are small, so we can consider the case where $\Psi_\ell$ is small compared to $\Psi_u$. Solving for $\Psi_\ell$ in terms of $\Psi_u$: 

\begin{eqnarray*}
	E \Psi_\ell &=& 	H'_{21} \Psi_u + H'_{22} \Psi_\ell \\
	\Psi_\ell 	&=& 	(E - H'_{22})^{-1} H'_{21} \Psi_u
\end{eqnarray*}

This gives the exact formula:
\begin{eqnarray*}
	E \Psi_u 	&=&		\left( H'_{11} + H'_{12}[E-H'_{22}]^{-1} H'_{21} \right) \Psi_u
\end{eqnarray*}

However, we only need corrections to the magnetic moment of order $v^2$.  With $\frac{e}{m}\v{S} \cdot \v{B} \sim mv^2$, this means we only need the Hamiltonian to at most order $mv^4$.  Examining the leading order terms of the matrix H', the diagonal elements are order m while the off-diagonal elements are order $mv^2$.  To leading order the term $[E-H'_{22}]^{-1}=\frac{1}{2m}$. So we'll need $H'_{11}$ to $\mathcal{O}(v^4)$, and $H'_{12}$, $H'_{21}$, and $[E-H'_{22}]^{-1}$ each to only leading order.


\[	E \Psi_u 	=		\left( H'_{11} + \frac{1}{2m}H'_{12} H'_{21} + \mathcal{O}(mv^6)\right) \Psi_u \]

The needed terms of H are, using $\lambda=g-1$
\begin{eqnarray*}
	H'_{11} 	= a_0+ a_3
			&=& m + e\Phi - \frac{D^2}{2m} - \frac{g}{2}\frac{e}{m} \v{S} \cdot \v{B}	\\
			&& +(g-1)\frac{e}{2m^2} 
				\left [ 
					\v{E} \cdot \v{D} - (\v{S} \cdot \v{E})( \v{S} \cdot \v{D}) + i \v{S} \cdot \v{E} \times \v{D}
					- \v{D} \cdot \v{E} + (\v{S} \cdot \v{D}) (\v{S} \cdot \v{E}) - i \v{S} \cdot \v{D} \times \v{E} 
				\right ]
			+\mathcal{O}(mv^6)	\\
	H'_{12} 	= a_1 - ia_2
			&=& \frac{D^2}{2m} - \frac{1}{m}(\v{S} \cdot \v{D})^2 - \frac{g-2}{2}\frac{e}{m} \v{S} \cdot \v{B}
			+\mathcal{O}(mv^4)	\\
	H'_{21}  = a_1 + ia_2
			&=&  -\frac{D^2}{2m} + \frac{1}{m}(\v{S} \cdot \v{D})^2 + \frac{g-2}{2}\frac{e}{m} \v{S} \cdot \v{B}
			+\mathcal{O}(mv^4)
\end{eqnarray*}

The product $H'_{12}H'_{21}$ is calculated in the appendix.  To first order in the magnetic field strength it is:
\begin{eqnarray*}
\frac{1}{2m}H'_{12}H'_{21}
	&=&	-\frac{1}{2m^3}\left(  \frac {\v{D}^2} {2} -  (\v{S} \cdot \v{D})^2  -  \frac{g-2}{2}  e\v{S} \cdot \v{B} \right )^2	\\	
	&=& 	-\frac{1}{2m^3}\left( 
				\frac{ \gv{\pi}^4 } {4}  -  e \v{p}^2  \v{S} \cdot \v{B}   
				-\frac{g-2}{2}e (\v{S} \cdot \v{p}) (\v{B} \cdot \v{p})
			\right)
\end{eqnarray*}

So finally,replacing all $\v{D}$ with $\gv{\pi} \equiv  \v{p} - e \v{A}$, we have a direct expression for $\Psi_u$:
\begin{eqnarray*}
	E \Psi_u 
		&=&\left \{ m + e\Phi + \frac{\pi^2}{2m} - \frac{g}{2}\frac{e}{m} \v{S} \cdot \v{B}
			- \frac{\pi^4 } {8m^3}  
			+ \frac{ e \v{p}^2  (\v{S} \cdot \v{B}) }{2m^3}
			+ (g-2)\frac{e}{4m^3} (\v{S} \cdot \v{p}) (\v{B} \cdot \v{p})
				 \right .	\\
		&&	\left . 
			+ (g-1)\frac{ie}{2m^2}  \left [ 
					\v{E} \cdot \gv{\pi} - (\v{S} \cdot \v{E}) (\v{S} \cdot \gv{\pi}) + i \v{S} \cdot \v{E} \times \gv{\pi}
					- \gv{\pi} \cdot \v{E} + (\v{S} \cdot \gv{\pi}) (\v{S} \cdot \v{E}) - i \v{S} \cdot \gv{\pi} \times \v{E} 
			\right ]			
			\right \}\Psi_u
\end{eqnarray*}
The complicated expression in square brackets can be cleaned up a bit:

\begin{eqnarray*}
\v{E} \cdot \gv{\pi} - \gv{\pi} \cdot \v{E}
	&=&	[E_i, \pi_i]			\\
	&=&	[E_i, -i\partial_i]	\\
	&=&	i(\partial_i E_i)		\\
(\v{S} \cdot \v{E}) (\v{S} \cdot \gv{\pi}) - (\v{S} \cdot \gv{\pi}) (\v{S} \cdot \v{E})
	&=&	S_i S_j E_i \pi_j - S_i S_j \pi_i E_j						\\
	&=&	(S_i S_j) (E_i \pi_j - E_j \pi_i - [\pi_i, E_j])					\\
	&=&	[S_i, S_j](E_i \pi_j) - (S_i S_j)(-i \nabla_i E_j)				\\
	&=&	(i\epsilon_{ijk}S_k)E_j \pi_i -  (S_i S_j)(-i \nabla_i E_j)		\\
	&=&	i \v{S} \cdot \v{E} \times \gv{\pi} + i S_i S_j \nabla_i E_j)	\\
(\v{E} \times \gv{\pi} - \gv{\pi} \times \v{E})_k
	&=&	\epsilon_{ijk}(E_i \pi_j - \pi_i E_j)		\\
	&=&	\epsilon_{ijk}(E_i \pi_j + \pi_j E_i)		\\
	&=&	\epsilon_{ijk}(2 E_i \pi_j + [\pi_i, E_j])	\\
	&=&	2 \epsilon_{ijk} E_i \pi_j				\\
	&=&	2 (\v{E} \times \gv{\pi})_k
\end{eqnarray*}

Using these identities and collecting terms, and then writing everything in terms of $g$, $g-2$
\begin{eqnarray*}
	E \Psi_u 
		&=&\left\{ m + e\Phi + \frac{\pi^2}{2m} - \frac{\pi^4 } {8m^3}
			- \frac{e}{m} \v{S} \cdot \v{B} \left ( \frac{g}{2} - \frac{p^2}{2m^2} \right )
			+ (g-2)\frac{e}{4m^3} (\v{S} \cdot \v{p}) (\v{B} \cdot \v{p})	\right. \\
		&&	\left.
			- (g-1)\frac{e}{2m^2} 
				\left [ 
					\v{\nabla} \cdot \v{E} 
					- S_i S_j \nabla_i E_j +\v{S} \cdot \v{E} \times \gv{\pi}
				\right ]
			\right\}\Psi_u	\\
		&=& \left\{ m + e\Phi + \frac{\pi^2}{2m} - \frac{\pi^4 } {8m^3}
			- \frac{g}{2}\frac{e}{m} \v{S} \cdot \v{B} \left ( 1 - \frac{p^2}{2m^2} \right )
			- \frac{g-2}{2} \frac{e}{m} \frac{p^2}{2m^2} \v{S} \cdot \v{B} 
			+ (g-2)\frac{e}{4m^3} (\v{S} \cdot \v{p}) (\v{B} \cdot \v{p})	\right.	\\
		&&	\left.
			- \left ( \frac{g}{2} + \frac{g-2}{2} \right) \frac{e}{2m^2} 
				\left [ 
					\v{\nabla} \cdot \v{E} 
					- S_i S_j \nabla_i E_j +\v{S} \cdot \v{E} \times \gv{\pi}
				\right ]
			\right\}\Psi_u
\end{eqnarray*}
			
We have a Hamiltonian for the upper component of the bispinor, as desired.  But is it truly Schrodinger-like?  In the general case we would need to perform the Fouldy-Wouthyusen transformation, to a representation where all the physics up to the desired order is contained in the single spinor equation.  But we can show that, at the order we are working at, the Hamiltonian above is correct.



\subsection{Normalization}
What we want is a single equation for the non-relativistic particle.  Because the lower component of the bispinor is small but nonzero, it is not necessarily true that the above equation accurately captures the physics.  The FW transformation to a basis where the lower component is truly negligible might be necessary.

However, here it can be shown that such a transformation will have no effect at the desired order.  The transformation U would have the form::
 $U = e^{iS} $ where S is Hermitian.  Because the Hamiltonian is diagonal at leading order, S must be small, and the transformation will affect $\Psi_u$ as
\begin{eqnarray*} 
\Psi_u \to \Psi'_u= (1 + \Delta)\Psi_u 
\end{eqnarray*}
where $\Delta$ is some small operator.  The probability density must be unaffected by this change, so:
On the one hand, with $\Psi_\ell = \epsilon \Psi_u$, $ \epsilon \sim \mathcal{O}(v^2) $

\begin{eqnarray*}
\int d^3 x (\Psi_u^\dagger \Psi_u - {\Psi^\dagger}_\ell \Psi_\ell)  
	&=& \int d^3 x (\Psi_u^\dagger \Psi_u - (\epsilon {\Psi_u})^\dagger \epsilon \Psi_u) \\
	&=& \int d^3 x \Psi_u^\dagger (1 + \mathcal{O}(v^4)) \Psi_u
\end{eqnarray*}

And on the other hand:
\begin{eqnarray*}
\int d^3 x {\Psi'}_u^\dagger \Psi'_u 
	&=& \int d^3 x ([1 + \Delta]\Psi_u)^\dagger (1+\Delta)\Psi_u \\
\end{eqnarray*}

Comparing the two, it can be seen that $\Delta$ must be no larger than $\mathcal{O}(v^4)$.  Now considering the new equation for $\Psi'_u$:
\begin{eqnarray*}
(E -m)\Psi_u 	
	&=&	\left(H'_{11} +  \frac{1}{2m}H'_{12} H'_{21} \right) \Psi_u\\
(E-m)(1-\Delta)\Psi'_u 
	&=&	\left(H'_{11} +  \frac{1}{2m}H'_{12} H'_{21} \right) (1-\Delta)\Psi'_u\\
(E-m)\Psi'_u 
	&=&	(1 + \Delta)\left(H'_{11} +  \frac{1}{2m}H'_{12} H'_{21} \right) (1-\Delta)\Psi'_u
\end{eqnarray*}

Since $H' \sim \mathcal{O}(mv^2)$ and $\Delta \sim \mathcal{O}(v^4)$, to $\mathcal{O}(v^4)$, $\Psi'_u$ obeys exactly the same equation as $\Psi_u$:
\begin{eqnarray*}
(E -m)\Psi'_u 	
	&=&  \left ( H'_{11} +  \frac{1}{2m} H'_{12} H'_{21}  \right ) \Psi'_u \\
\end{eqnarray*}


%%%%% Below needs to be moved elsewhere !

\section{Magnetic Moment}
Ultimately, we want to evaluate the magnetic moment of the bound vector particle.  In the presence of a weak magnetic field, the energy shift caused by the magnetic moment will be linearly proportional to the magnetic field strength, and will depend on the angle between the spin orientation and the direction of the magnetic field.

So our tactic will be to consider the contribution to the ground state energy level of terms which are linear in $\v{B}$ and contain at least one spin operator.  By evaluating such terms, we can calculate the magnetic moment.

To find the ground-state energy level we will use perturbation theory.  The leading order, unperturbed Hamiltonian will contain neither relativistic corrections nor terms due to the presence of the magnetic field.  It is given by:
\[ H_0 = e\Phi + \frac{p^2}{2m}  \]
Thus the unperturbed bound-state energy levels are just those of the hydrogen-like atom.  Since there are no spin operators in this expression, the unperturbed levels are degenerate in spin space.

The perturbation can be split into three types of terms: those of orders $\frac{B}{m}$, $mv^4$, and $\frac{B}{m}v^2$ respectively.

\[	V = V_{I} + V_{II} + V_{III}	\]

\[V_{I} = -g \frac{e}{2m} \v{S} \cdot \v{B} - e \frac{\v{p} \cdot \v{A} + \v{A} \cdot \v{p}}{2m} \]

\[V_{II} = \frac{p^4}{8m^3} - \frac{e}{2m^2}\left(\frac{g}{2} + \frac{g-2}{2} \right) \left [ (\delta_{ij} - S_i S_j ) \partial_i E_j + S_i E_j p_k \epsilon_{ijk} \right ]  \] 

\begin{eqnarray*}
 V_{III} &=& 
		e\frac{p^2 ( \v{p} \cdot \v{A} + \v{A} \cdot \v{p}) + (\v{p} \cdot \v{A} + \v{A} \cdot \v{p})p^2}{8m^3}
		+  \frac{e^2}{2m^2} \left(\frac{g}{2} + \frac{g-2}{2} \right)  S_i  E_j A_k \epsilon_{ijk}	\\
	&&	+ \frac{e}{2m}\left( g - [g-2] \right) \v{S} \cdot \v{B} \frac{p^2}{2m} 
		+ (g-2)\frac{e}{2m} \frac{ (\v{S} \cdot \v{p}) (\v{B} \cdot \v{p})}{2m^2}
\end{eqnarray*}

We want the shift to the energy levels given by this perturbation.  The general equation for the shift of energy level of some state n is
\[
	\Delta E_n = \matrixel{n}{V}{n}	- \Sigma_{m \neq n} \frac {\matrixel{n}{V}{m} \matrixel{m}{V}{n}} { E_m - E_n }
\]
where n and m index the set of bound states.

To evaluate the matrix elements of these operators we will need to abandon generality and consider our specific case, where $\v{E}$ is the Coulomb field. 
\[	\v{E} = -\frac{Ze}{4\pi} \frac{\v{r}}{r^3}	\]

For a constant $\v{B}$, we also chose a particularly convenient form for $\v{A}$:
\[	\v{A} = \frac{1}{2}\v{B} \times \v{r}		\]
We can take the direction of the magnetic field to be along the z axis, so that the term $\v{S} \cdot \v{B}$ is diagonal in spin space.

\subsection*{First order perturbation}
The first order perturbation term to the ground state energy is
\[	\Delta^{(0)}_0 = \matrixel{n}{V}{n}	\]


\begin{eqnarray*}  
H_{S \cdot B} &=& - \frac{e}{m} \left\{
				g \v{S} \cdot \v{B} \left ( 1 - \frac{p^2}{2m^2} \right )
				+ (g-2) \v{S} \cdot \v{B} \frac{p^2}{2m^2}
				- (g-2) \frac{ (\v{S} \cdot \v{p}) (\v{B} \cdot \v{p})}{2m^2}
				+ \frac{e}{m} \left(\frac{g}{2} + \frac{g-2}{2} \right)  \v{S} \cdot \v{E} \times \v{A}
			\right\}	
\end{eqnarray*}
With our definition of E and B, $\v{S} \cdot \v{E} \times \v{A} = - \frac{Ze}{8\pi} \frac{1}{r^3} \v{S} \cdot \v{r} \times [\v{B} \times \v{r}]$.


We take the matrix elements of these terms in the ground state.  The spherical symmetry of the unperturbed state dictates that $\avg{\frac{r_i r_j}{r^3}}=\delta_{ij}\frac{1}{3}\avg{\frac{1}{r}}$.
Using these identities, we find that:
\begin{eqnarray*}
\avg{\v{S} \cdot \v{B} p^2} 
	&=& \v{S} \cdot \v{B} \avg{p^2}	\\
 \avg{\frac{1}{r^3} \v{S} \cdot \v{r} \times (\v{B} \times \v{r})}
	&=&	\avg{\frac{1}{r^3} S_k R_i B_l r_m \epsilon_{jlm} \epsilon_{ijk}}	\\
	&=&	\avg{ \frac{1}{r} \v{S} \cdot \v{B}  - \frac{1}{r^3} B_i S_j r_i r_j }	\\
	&=&	\v{S} \cdot \v{B} \avg{\frac{1}{r}} - B_i S_j \avg{\frac{r_i r_j}{r^3}}	\\
	&=&	\v{S} \cdot \v{B}(\avg{\frac{1}{r}} - \frac{1}{3}\avg{\frac{1}{r}})	\\
	&=&	\frac{2}{3} \v{S} \cdot \v{B}\avg{\frac{1}{r}}\\
\avg{ (\v{S} \cdot \v{p}) (\v{B} \cdot \v{p}) }
	&=&	\avg{ S_i B_j p_i p_j }	\\
	&=&	S_i B_j \avg{p_i p_j}	\\
	&=&	S_i B_j \delta_{ij} \frac{\avg{p^2}}{3}  \\
	&=&	 \v{S} \cdot \v{B} \frac{\avg{p^2}}{3}
\end{eqnarray*}

For the ground state, $\avg{\frac{1}{r} }= mZ\alpha$, $\avg{p^2} = (mZ\alpha)^2$.  So 
\begin{eqnarray*}  
\avg{H_{S \cdot B} }
	&=& - \frac{e}{m} \v{S} \cdot \v{B} 
		\left ( 1 - \frac{\avg{p^2}}{2m^2}  + \frac{Ze^2}{6m}\avg{\frac{1}{r}}\right )	\\
	&=&  - \frac{e}{m} \v{S} \cdot \v{B} \left (1 - \frac{ (Z \alpha)^2}{2} +\frac{(Z\alpha)^2}{6} \right )	\\
	&=&  - \frac{e}{m} \v{S} \cdot \v{B} \left (1 - \frac{(Z\alpha)^2}{3} \right )	\\
\end{eqnarray*}


\subsection*{Second order perturbation}
We want the second order correction to the ground state energy level.  First we consider matrix elements between the ground level and excited levels, and show that none contribute to the magnetic moment.  Then we consider degenerate levels with differing spin, but it will turn out that such matrix elements also vanish.

Given that we throw away terms quadratic in the magnetic field, and of order greater than $mv^4$ or $\frac{B}{m}v^2$, the only terms contributing to second order in perturbation theory are

\[ \Delta^{(1)} E_0 = -2\Sigma_{m \neq 0} \frac {\matrixel{0}{V_{I}}{m} \matrixel{m}{V_{III}}{0}} { E_m - E_0 }	\]

Now, since the term $\v{S} \cdot \v{B}$ contains only constants and spin space operators, it clearly doesn't connect different energy levels.  So $\matrixel{0}{\v{S} \cdot \v{B}}{m} = 0$.

The term $\matrixel{0}{\v{p} \cdot \v{A} + \v{A} \cdot \v{p}}{m}$ does contain position space operators, but will vanish due to symmetry considerations.

\begin{eqnarray*}
 \v{p} \cdot \v{A} + \v{A} \cdot \v{p} 
	&=&	p_i (r_j B_k \epsilon_{ijk} ) + (r_j B_k \epsilon_{ijk}) p_i	\\
	&=&	[p_i, r_j] B_k \epsilon{ijk} + 2 B_k r_j p_i  \epsilon_{ijk}	\\
	&=&	2 B_k r_j p_i  \epsilon_{ijk}
\end{eqnarray*}
However, $ r_j p_i \epsilon_{ijk}$ is proportional to the angular momentum operator $L_k$.  When acting on the spherically symmetric ground state, this operator must vanish.

Therefore, for any excited state m, $\matrixel{m}{V_{I}}{0}$ vanishes.  Now we must consider the matrix element between two degenerate ground level states of differing spin.

\subsection{Degenerate}
Clearly any term not involving spin operators will not connect states of differing spin.  Likewise, with our convention that we label spin states by their projection along the direction of the magnetic field, the operator $\v{S} \cdot \v{B} = S_3 B_3$ is also diagonal in spin space.

The remaining terms involving spin operators are:
\[	 -\frac{e}{2m^2} ( S_i E_j (p_k - eA_k) \epsilon_{ijk} - S_i S_j \partial_i E_j) \]
By the same reasoning used above, $E_j p_k \epsilon_{ijk}$ is proportional to $L_i$, which will vanish when acting on the ground state.

Now consider the term containing $S_i E_j A_k \epsilon_{ijk}$.  Using the specific gauge chosen, and the particular form of $\v{E}$, it will be proportional to 

\begin{eqnarray*}
S_i \left( \frac{r_j}{r^3} \right) (r_l B_m \epsilon_{lmk}) \epsilon_{ijk}
	&=&	\frac{1}{r^3} S_i r_j r_l B_m (\delta_{li} \delta_{mj} - \delta_{lj} \delta_{mi} )	\\
	&=&	\frac{1}{r^3} (S_i r_i r_j B_j - S_i B_i r_j r_j)
\end{eqnarray*}	

Between two ground state wave functions $\avg{\frac{r_i r_j}{r^3}}=\delta_{ij}\frac{1}{3}\avg{\frac{1}{r}}$.  So after averaging over position space (but not yet calculating the spin part) the above reduces to
\[	 
	\frac{1}{r} (S_i \delta_{ij} B_j - S_i B_i \delta_{jj})
	= -2 \v{S} \cdot \v{B} \avg{\frac{1}{r}}
\]
Which is again diagonal in spin space.

The last term to consider contains the position space operator $\partial_i E_j$.  The expectation value of this in the ground state must be something proportional to $\delta_{ij}$.  Then the operator $S_i S_j$ in spin space will be proportional to $S^2$.  So this term too is diagonal in spin space.

Term by term we have shown that the potential does not connect any two ground level states of different spin.  Thus, degenerate perturbation theory is not needed.

So the result of first order perturbation theory stands as our final result:
\[ \avg{H_{S \cdot B} }
	=  - \frac{e}{m} \v{S} \cdot \v{B} \left (1 - \frac{(Z\alpha)^2}{3} \right )			 
\]

\appendix


\section{Comparison with Silenko}

We can compare to a result in Silenko (arXiv:hep-th/0401183v1).  Silenko determines a Hamiltonian which is equivalent to that used above.  Dropping terms manifestly beyond $\mathcal{O}(mv^4)$:

\begin{eqnarray*}
H &=& \rho_3 \epsilon' + e\Phi +\frac{e}{4m} 
	\left [ \left \{ 	\left (\frac{g-2}{2} + \frac{m}{\epsilon' + m} \right )\frac{1}{\epsilon'}, 
				(\v{S} \cdot \gv{\pi} \times \gv{E} - \v{S} \cdot \v{E} \times \gv{\pi})
		\right \}_+
		- \rho_3 \left \{ \left ( g-2 + \frac{2m}{\epsilon'} \right ), \v{S} \cdot \v{B} \right \}_+
	\right.	\\
	&& \left.
		+\rho_3 \left \{ \frac{g-2}{2\epsilon' (\epsilon'+m)}, 
			\{\v{S} \cdot \gv{\pi}, \gv{\pi} \cdot \v{B} \}_+ \right \}_+
	\right ]
	+ \frac{e(g-1)}{4m^2}\left \{ \v{S} \cdot \gv{\nabla}, \v{S} \cdot \v{E} \right \}_+
	- \frac{e(g-1)}{2m^2} \v{\nabla} \cdot \v{E}
\end{eqnarray*}


We'll take each term and expand to $\mathcal{O}(mv^4)$. First the operator $\epsilon'$: 
\begin{eqnarray*}
\epsilon' 	
	&=& 	\sqrt{m^2 + \pi^2}						\\
	&=&	m + \frac{\pi^2}{2m} - \frac{\pi^4}{8m^2} +\mathcal{O}(mv^6)
\end{eqnarray*}

Using this, we get
\begin{eqnarray*}
H &=& \rho_3 \epsilon' + e\Phi +\frac{e}{4m} 
	\left [ \left \{ 	\frac{g-1}{2m}, 
				(\v{S} \cdot \gv{\pi} \times \gv{E} - \v{S} \cdot \v{E} \times \gv{\pi})
		\right \}_+
		- \rho_3 \left \{ g -\frac{\pi^2}{m^2} , \v{S} \cdot \v{B} \right \}_+
	\right.	\\
	&& \left.
		+\rho_3 \left \{ \frac{g-2}{4m^2}, 
			\{\v{S} \cdot \gv{\pi}, \gv{\pi} \cdot \v{B} \}_+ \right \}_+
	\right ]
	+ \frac{e(g-1)}{4m^2}\left \{ \v{S} \cdot \gv{\nabla}, \v{S} \cdot \v{E} \right \}_+
	- \frac{e(g-1)}{2m^2} \v{\nabla} \cdot \v{E}
\end{eqnarray*}

Now we'll expand the anticommutators.



We have already shown that $\gv{\pi} \times \v{E} = - \v{E} \times \gv{\pi}$, and that such a term is $\mathcal{O}(mv^4)$
\begin{eqnarray*}
\left \{ 	\frac{g-1}{2m}, 
				(\v{S} \cdot \gv{\pi} \times \gv{E} - \v{S} \cdot \v{E} \times \gv{\pi})
		\right \}_+
	&=&			-\frac{2(g-1)}{m} 	\v{S} \cdot \v{E} \times \gv{\pi}
\end{eqnarray*}

Because we specialize to constant magnetic fields, $\gv{\pi}$ commutes with $\v{B}$, so 
\begin{eqnarray*}
\left \{ g -\frac{\pi^2}{m^2} , \v{S} \cdot \v{B} \right \}_+
	&=&	(2g - \frac{2\pi^2}{m^2}) \v{S} \cdot \v{B}
\end{eqnarray*}

and

\begin{eqnarray*}
\left \{ \frac{g-2}{4m^2}, 
			\{\v{S} \cdot \gv{\pi}, \gv{\pi} \cdot \v{B} \}_+ \right \}_+
	&=&		\frac{g-2}{m^2} (\v{S} \cdot \gv{\pi}) (\gv{\pi} \cdot \v{B})
\end{eqnarray*} 

Using the identity that $[\nabla_i, E_j]=0$, $\v{\nabla} \times \v{E} = 0$:
 \begin{eqnarray*}
 \left \{ \v{S} \cdot \gv{\nabla}, \v{S} \cdot \v{E} \right \}_+
	&=& S_i S_j \nabla_i E_j + S_j S_i E_j \nabla_i	\\
	&=& (S_i S_j + S_j S_i) \nabla_i E_j			\\
	&=& (2S_i S_j + [S_j, S_i])\nabla_i E_j			\\
	&=& (2S_i S_j +iS_k\epsilon_{ijk})\nabla_i E_j	\\
	&=& 2S_i S_j\nabla_i E_j
 \end{eqnarray*}
 
So
\begin{eqnarray*}
H 	&=& \rho_3 (m + \frac{\pi^2}{2m} - \frac{\pi^4}{8m^2})  + e\Phi - (g-1)\frac{e}{2m^2} 
		\left [ 
			 \v{S} \cdot \v{E} \times \gv{\pi}
			-S_i S_j \nabla_i E_j 
			+\v{\nabla} \cdot \v{E}	
		\right ]
	\\&&
		+ \rho_3 (g-2)\frac{2}{4m^2} (\v{S} \cdot \gv{\pi}) (\gv{\pi} \cdot \v{B})
		- \rho_3 \frac{e}{m}(\frac{g}{2} - \frac{\pi^2}{2m^2}) \v{S} \cdot \v{B}	\\
	&=& \rho_3 \left(m + \frac{\pi^2}{2m} - \frac{\pi^4}{8m^2} \right)  + e\Phi - \left(\frac{g-2}{2} + \frac{g}{2}\right)\frac{e}{2m^2} 
		\left [ 
			 \v{S} \cdot \v{E} \times \gv{\pi}
			-S_i S_j \nabla_i E_j 
			+\v{\nabla} \cdot \v{E}	
		\right ]	
	\\&&
		+ \rho_3 (g-2)\frac{2}{4m^2} (\v{S} \cdot \gv{\pi}) (\gv{\pi} \cdot \v{B})
		- \rho_3 \frac{e}{m}\left [\frac{g}{2}\left(1 - \frac{\pi^2}{2m^2}\right) + \frac{g-2}{2}\frac{\pi^2}{2m^2}) \right] \v{S} \cdot \v{B}							
\end{eqnarray*}

\subsection*{Magnetic Moment}
Now we'll keep only those terms which contribute to the magnetic moment, using $\gv{\pi} = \v{p} - e\v{A}$
\begin{eqnarray*}
H_{S\cdot B}	&=&
		 \left(\frac{g-2}{2} + 	\frac{g}{2}\right)\frac{e^2}{2m^2} \v{S} \cdot \v{E} \times \v{A}
		+  (g-2)\frac{e}{4m^2} (\v{S} \cdot \v{p}) (\v{p} \cdot \v{B})
		-  \frac{e}{m}\left [\frac{g}{2}\left(1 - \frac{p^2}{2m^2}\right) + \frac{g-2}{2}\frac{p^2}{2m^2} \right] \v{S} 		\cdot \v{B}		\\
	&=&		-\frac{e}{2m} \left\{
			g \left(1 - \frac{p^2}{2m^2}\right)  \v{S} \cdot \v{B}	
			+(g-2) \frac{p^2}{2m^2} \v{S} \cdot \v{B}	
			-(g-2) \frac{ (\v{S} \cdot \v{p}) (\v{p} \cdot \v{B})}{2m^2}
			-\frac{e}{m} \left(\frac{g-2}{2} + 	\frac{g}{2}\right) \v{S} \cdot \v{E} \times \v{A}
		\right\}
\end{eqnarray*}
This exactly matches with what was found before.



\section{Identities}

Simplify $ \v{W} \times \v{B}$:
\begin{eqnarray*}
(\v{W} \times \v{B})_i
	&=&	\epsilon_{ijk} W_j B_k	\\
	&=&	i(S_k)_{ij} W_j B_k\\
	&=&	i(\v{S} \cdot \v{B})_{ij} W_j	\\
	&=&	i([\v{S} \cdot \v{B}] \v{W})_i
\end{eqnarray*}

Simplify $\v{D} \times ( \v{D} \times \v{W} ) $
\begin{eqnarray*}
(\v{D} \times [ \v{D} \times \v{W} ] )_i
	&=&	\epsilon_{ijk} D_j (\v{D} \times \v{W})_k	\\
	&=&	\epsilon_{ijk} \epsilon_{k\ell m} D_j D_\ell W_m	\\
	&=&	-(S_j)_{ki} (S_\ell)_{mk} D_j D_\ell W_m	\\
	&=&	-(\v{S} \cdot \v{D})_{ki} (\v{S} \cdot \v{D})_{mk} W_m	\\
	&=&	-\left( [\v{S} \cdot \v{D}]^2 \right)_{im} W_m	\\
	&=&	-\left( [\v{S} \cdot \v{D}]^2 \v{W} \right)_i	\\
\end{eqnarray*}

Our representation defines the spin matrices as follows:

$${(S_k)}_{ij}=-i \epsilon_{ijk}$$

They have the commutator
$$	[S_i, S_j] = i \epsilon_{ijk} S_k $$

The product of two such spin matrices is given by:
\begin{eqnarray*}
{(S_k S_\ell)}_{ij} 
	& = & {(S_k)}_{ia} {(S_\ell)}_{aj} \\
	& = & -\epsilon_{iak} \epsilon_{aj\ell} \\
	& = & (\delta_{k\ell} \delta_{ij} - \delta_{kj} \delta_{\ell i} )
\end{eqnarray*}

This implies that:

\begin{eqnarray*}
(S_i S_j A_j B_i \v{v} )_l 
	&=&  	{(S_i)}_{lm} {(S_j)}_{mn} A_j B_i v_n \\
	&=&	{(S_iS_j)}_{ln} A_j B_i v_n \\
	&=&	 (\delta_{ij} \delta_{ln} - \delta_{in} \delta_{lj} ) A_j B_i v_n \\
	&=&	(\v{A} \cdot \v{B}) v_l  - A_l ( \v{B} \cdot \v{v}) \\
\end{eqnarray*}

Or 

$$ \v{A} (\v{B} \cdot \v{v}) =  (\v{A} \cdot \v{B}   -  S_i S_j A_j B_i) \v{v} $$

Now we can use this to establish some identities:

\begin{eqnarray*}
 E^i \v{D} \cdot \v{\eta}
	&=&	\left( \left[\v{E} \cdot \v{D}  - S_j S_k E_k D_j\right] \v{\eta} \right)^i		\\
	&=&	\left( \left[\v{E} \cdot \v{D}  - (S_k S_j  - [S_k, S_j] ) E_k D_j \right] \v{\eta} \right)^i \\
	&=&	\left( \left[\v{E} \cdot \v{D}  - (S_k S_j - i \epsilon_{kjl} S_l ) E_k D_j \right] \v{\eta} \right)^i	  \\
	&=&	\left( \left[\v{E} \cdot \v{D}  - (\v{S} \cdot \v{E})( \v{S} \cdot \v{D}) +  i  S_l (\v{E} \times \v{D})_l \right] \v{\eta} \right)^i	  \\
	&=&	\left( \left[\v{E} \cdot \v{D}  - (\v{S} \cdot \v{E})( \v{S} \cdot \v{D}) +  i  \v{S} \cdot (\v{E} \times \v{D})\right] \v{\eta} \right)^i
\end{eqnarray*}

Since E and W commute:
\begin{eqnarray*}
 E^i \v{W} \cdot \v{E}
	&=& \left( \left[\v{E}^2  - S_j S_k E_k E_j\right] \v{W} \right)^i		\\
	&=& \left( \left[\v{E}^2  - (\v{S} \cdot \v{E} )^2 \right] \v{W} \right)^i		\\
\end{eqnarray*}

And
\begin{eqnarray*}
 D^i \v{D} \cdot \v{\eta}
	&=& \left( \left[\v{D}^2  - S_j S_k D_k D_j\right] \v{\eta} \right)^i		\\
	&=& \left( \left[\v{D}^2  - (S_k S_j + [S_j, S_k]) D_k D_j\right] \v{\eta} \right)^i		\\
	&=& \left( \left[\v{D}^2  - (\v{S} \cdot \v{D})^2 + i\epsilon_{jkl} S_l D_k D_j )\right] \v{\eta} \right)^i		\\
	&=& \left( \left[\v{D}^2  - (\v{S} \cdot \v{D})^2 + i\v{S} \cdot (\v{D} \times \v{D})  \right] \v{\eta} \right)^i		\\
	&=& \left( \left[\v{D}^2  - (\v{S} \cdot \v{D})^2 + e \v{S} \cdot \v{B})  \right] \v{\eta} \right)^i		\\
\end{eqnarray*}

Similarly:
\begin{eqnarray*}
 D^i \v{E} \cdot \v{W}
	&=& \left( \left[\v{D} \cdot \v{E}  - S_j S_k D_k E_j\right] \v{\eta} \right)^i		\\
	&=& \left( \left[\v{D} \cdot \v{E}  -  (S_k S_j + [S_j, S_k]) D_k E_j \right] \v{\eta} \right)^i		\\
	&=& \left( \left[\v{D} \cdot \v{E}  -  (S_k S_j +i\epsilon_{jkl} S_l) D_k E_j \right] \v{\eta} \right)^i		\\
	&=& \left( \left[\v{D} \cdot \v{E}  - (\v{S} \cdot \v{D})(\v{S} \cdot \v{E}) + i \v{S} \cdot (\v{D} \times \v{E})\right] \v{\eta} \right)^i		\\
\end{eqnarray*}	

\subsection*{Product of $H_{12}H_{21}$}
We need to calculate $\left(  \frac {\gv{\pi}^2} {2} -  (\v{S} \cdot \gv{\pi})^2 + (g-2)\frac{2}{m} \v{S} \cdot \v{B} \right )^2 $ to first order in magnetic field strength.  As a first step of simplification
\[
\left(  \frac {\gv{\pi}^2} {2} -  (\v{S} \cdot \gv{\pi})^2 + \frac{g-2}{2}\frac{e}{m} \v{S} \cdot \v{B} \right )^2 
	=	\left(  \frac {\gv{\pi}^2} {2} -  (\v{S} \cdot \gv{\pi})^2  \right )^2 
		 +\frac{g-2}{2}\frac{e}{m} \left \{ \frac{p^2}{2} - (\v{S} \cdot \v{p})^2, \v{S} \cdot \v{B} \right \}
\]	

\subsubsection*{First term}
To simplify the first term, consider one element of this matrix operator:
\begin{eqnarray*}
\left\{ \left(  \frac {\gv{\pi}^2} {2} -  (\v{S} \cdot \gv{\pi})^2   \right )^2  \right \} _{ac}
	&=& 	\left (  \frac {\gv{\pi}^2} {2} - S_i S_j \pi_i \pi_j \right)_{ab} 
				\left (  \frac {\gv{\pi}^2} {2} - S_l S_m \pi_l \pi_m \right)_{bc}\\
	&=& 	\left (  \frac {\gv{\pi}^2} {2}\delta_{ab} - [S_i S_j]_{ab} \pi_i \pi_j \right)
				\left (  \frac {\gv{\pi}^2} {2}\delta_{bc} - [S_l S_m]_{bc} \pi_l \pi_m \right)\\
	&=& 	\left (  \frac {\gv{\pi}^2} {2}\delta_{ab} - [\delta_{ab}\delta_{ij} - \delta_{aj}\delta_{bi}] \pi_i \pi_j \right)
				\left (  \frac {\gv{\pi}^2} {2}\delta_{bc} -  [\delta_{bc}\delta_{lm} - \delta_{bm}\delta_{cl}] \pi_l \pi_m \right)\\
	&=& 	\left (  -\frac {\gv{\pi}^2} {2}\delta_{ab} + \pi_b \pi_a \right)
				\left (  -\frac {\gv{\pi}^2} {2}\delta_{bc} + \pi_c \pi_b \right) \\
	&=&  	\frac{ \gv{\pi}^4 } {4} \delta_{ac}
				- \pi_c \pi_a \frac{\gv{\pi}^2}{2}
				- \frac{\gv{\pi}^2}{2} \pi_c \pi_a
				+ \pi_b \pi_a \pi_c \pi_b \\
\end{eqnarray*}

It's very useful to have the following identity:
\begin{eqnarray*}
	e(\v{S} \cdot \v{B})_{ab}
	&=&	e(S_i)_{ab} B_i	\\
	&=&	-i e \epsilon_{iab} B_i 	\\
	&=&	-i e \epsilon_{iab} (\epsilon_{ijk}\partial_j A_k)	\\
	&=&	-i e \epsilon_{iab} \epsilon_{ijk} \frac{1}{2} (\partial_j A_k \partial_k A_j) \\
	&=&	- \epsilon_{iab} \epsilon_{ijk} \frac{1}{2} [\pi_j, \pi_k]	\\
	&=&	-\epsilon_{iab} \epsilon_{ijk} \pi_j \pi_k	\\
	&=&	-(\delta_{aj} \delta_{bk}	- \delta_{ak} \delta_{bj} ) \pi_j \pi_k \\
	&=&	\pi_b \pi_a - \pi_a \pi_b	\\
\end{eqnarray*}

Therefore,
	$$ 	e(\v{S} \cdot \v{B})_{ab} = [\pi_b, \pi_a] $$



Using this, and the fact that $\pi$ commutes with S and B:
\begin{eqnarray*}
\pi_b \pi_a \pi_c \pi_b 
	&=&	\pi_b \pi_c \pi_a \pi_b 
				- \pi_b (e \v{S} \cdot \v{B})_{ac} \pi_b \\
	&=&	\pi_b \pi_c \pi_a \pi_b 
				-\gv{\pi}^2 (e \v{S} \cdot \v{B})_{ac}	\\
\end{eqnarray*}

Also,

\begin{eqnarray*}
\pi_b \pi_c \pi_a \pi_b 
	&=&	\pi_b \pi_c \pi_b \pi_a 
				+ \pi_b \pi_c (e \v{S} \cdot \v{B})_{ba}	\\
	&=&	\pi_b \pi_b \pi_c \pi_a 
				+ \pi_b \pi_a (e \v{S} \cdot \v{B})_{bc}
				+ \pi_b \pi_c (e \v{S} \cdot \v{B})_{ba}	\\
\end{eqnarray*}

So now:
\begin{eqnarray*}
\pi_b \pi_a \pi_c \pi_b  
	- \pi_c \pi_a \frac{\gv{\pi}^2}{2} 
	- \frac{\gv{\pi}^2}{4} \pi_c \pi_a
	&=&	\gv{\pi}^2 \pi_c \pi_a 
				+ \pi_b \pi_a (e \v{S} \cdot \v{B})_{bc}
				+ \pi_b \pi_c (e \v{S} \cdot \v{B})_{ba} \\
	&&			-\gv{\pi}^2 (e \v{S} \cdot \v{B})_{ac}
				- \pi_c \pi_a \frac{\gv{\pi}^2}{2} 
				- \frac{\gv{\pi}^2}{4} \pi_c \pi_a	\\
	&=&	\frac{1}{2} [\gv{\pi}^2, \pi_c \pi_a]
				+ \pi_b \pi_a (e \v{S} \cdot \v{B})_{bc}
				+ \pi_b \pi_c (e \v{S} \cdot \v{B})_{ba}
				-\gv{\pi}^2 (e \v{S} \cdot \v{B})_{ac} \\
\end{eqnarray*}

Now evaluate the commutator:
\begin{eqnarray*}
[\pi_b, \pi_c \pi_a]
	&=&	[\pi_b, \pi_c]\pi_a - \pi_c [\pi_a, \pi_b]	\\
	&=&	(e \v{S} \cdot \v{B})_{cb} \pi_a
				- \pi_c (e \v{S} \cdot \v{B})_{ba}	\\
\end{eqnarray*}

\begin{eqnarray*}
[ \gv{\pi}^2, \pi_c \pi_a]
	&=&	[\pi_b \pi_b, \pi_c \pi_a]	\\
	&=&	\pi_b [\pi_b, \pi_c \pi_a] + [\pi_b, \pi_c \pi_a] \pi_b	\\
	&=&	(e \v{S} \cdot \v{B})_{cb} ( \pi_b \pi_a + \pi_a \pi_b)
				-  (e \v{S} \cdot \v{B})_{ba} (\pi_b \pi_c + \pi_c \pi_b)	\\
\end{eqnarray*}


This gives the result:
\begin{eqnarray*}
\pi_b \pi_a \pi_c \pi_b  
	- \pi_c \pi_a \frac{\gv{\pi}^2}{2} 
	- \frac{\gv{\pi}^2}{4} \pi_c \pi_a
	&=&	\frac{1}{2} \left [ (e \v{S} \cdot \v{B})_{cb} ( \pi_b \pi_a + \pi_a \pi_b)
				-  (e \v{S} \cdot \v{B})_{ba} (\pi_b \pi_c + \pi_c \pi_b)  \right  ] \\
	&&		+ \pi_b \pi_a (e \v{S} \cdot \v{B})_{bc} 
				+ \pi_b \pi_c (e \v{S} \cdot \v{B})_{ba}
				-\gv{\pi}^2 (e \v{S} \cdot \v{B})_{ac}	\\
	&=&	\frac{1}{2} [(e \v{S} \cdot \v{B})_{cb} ( \pi_a \pi_b - \pi_b \pi_a)
				+ (e \v{S} \cdot \v{B})_{ba} (\pi_b \pi_c - \pi_c \pi_b)]
				-\gv{\pi}^2 (e \v{S} \cdot \v{B})_{ac}	\\
	&=&	\frac{1}{2} [(e \v{S} \cdot \v{B})_{cb} (e \v{S} \cdot \v{B})_{ba}
				+ (e \v{S} \cdot \v{B})_{ba} (e \v{S} \cdot \v{B})_{cb}]
				-\gv{\pi}^2 (e \v{S} \cdot \v{B})_{ac}	\\
	&=&	(e \v{S} \cdot \v{B})_{ab}(e \v{S} \cdot \v{B})_{bc})
				-\gv{\pi}^2 (e \v{S} \cdot \v{B})_{ac}	\\
	&=&	\left[ (e \v{S} \cdot \v{B})^2 \right ]_{ac}
				-\gv{\pi}^2 (e \v{S} \cdot \v{B})_{ac}	\\
\end{eqnarray*}

Since we can throw away terms of order $B^2$, the final result tells us that, to first order in B:
\begin{eqnarray*}
 \left(  \frac {\gv{\pi}^2} {2} -  (\v{S} \cdot \gv{\pi})^2   \right )^2
	&=&	\frac{ \gv{\pi}^4 } {4}  -  \gv{\pi}^2 (e \v{S} \cdot \v{B}) \\
	&=& 	\frac{ \gv{\pi}^4 } {4}  -  e \v{p}^2  \v{S} \cdot \v{B} \\
\end{eqnarray*}

\subsubsection*{Second term}
To simplify the second term, we need
\begin{eqnarray*}
\left \{ \frac{\v{p}^2}{2} - (\v{S} \cdot \v{p})^2, \v{S} \cdot \v{B} \right \}
	&=&	 \v{p}^2 \v{S} \cdot \v{B} - [(\v{S} \cdot \v{p})^2 \v{S} \cdot \v{B} + \v{S} \cdot \v{B} (\v{S} \cdot \v{p})^2  ]	\\
	&=&	 \v{p}^2 \v{S} \cdot \v{B} - (S_i S_j S_k + S_k S_j S_i) p_i p_j B_k
\end{eqnarray*}

To simplify that triple product of spin matrices, we can use their explicit form:
\begin{eqnarray*}
	(S_i S_j S_k )_{ab}
		&=&	i\epsilon_{aci}\epsilon_{cdj}\epsilon_{dbk}	\\
		&=&	i(\delta_{id} \delta_{aj} - \delta_{ij} \delta_{ad})\epsilon_{dbk}	\\
		&=&	i(\delta_{aj} \epsilon_{ibk} - \delta_{ij} \epsilon_{abk})		\\
	(S_i S_j S_k + S_k S_j S_i)_{ab}
		&=& i(\delta_aj \epsilon_{ibk} + \delta_{aj} \epsilon_{kbi} -\delta_{ij} \epsilon_{abk} -\delta{kj}\epsilon_{abi})	\\
		&=& -i(\delta_{ij} \epsilon_{abk} + \delta_{kj} \epsilon_{abi}	\\
		&=&	\delta_{ij} {(S_k)}_ab + \delta_{kj} {(S_i)}_{ab}	
\end{eqnarray*}
Now 
\begin{eqnarray*}
 \v{p}^2 \v{S} \cdot \v{B} - (S_i S_j S_k + S_k S_j S_i) p_i p_j B_k
 	&=& \v{p}^2 \v{S} \cdot \v{B} - (\delta_{ij} S_k + \delta_{kj} S_i	) p_i p_j B_k	\\
 	&=& - (\v{S} \cdot \v{p}) (\v{B} \cdot \v{p})
\end{eqnarray*}
\subsubsection*{Result}
At last, the final result is that, to the order we care about 
\[
\left(  \frac {\gv{\pi}^2} {2} -  (\v{S} \cdot \gv{\pi})^2 + (g-2)\frac{e}{m} \v{S} \cdot \v{B} \right )^2 
	=	\frac{ \gv{\pi}^4 } {4}  -  e \v{p}^2  \v{S} \cdot \v{B} \\
		 -\frac{g-2}{2}\frac{e}{m} (\v{S} \cdot \v{p}) (\v{B} \cdot \v{p})
\]	


\end{document}
