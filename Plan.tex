
\documentclass[12pt]{article}

\input{header.tex} 
\newcommand{\sigdot}[1]{ \gv{\sigma} \hspace{-2pt} \cdot \hspace{-2pt} \v{#1} \,}
\newcommand{\sigdotg}[1]{ \gv{\sigma} \hspace{-2pt} \cdot \hspace{-2pt} \gv{#1} \,}
\newcommand{\dotprod}[2]{ \v{#1} \hspace{-2pt} \cdot \hspace{-2pt} \v{#2} \,}

\newcommand{\beqa}{\begin{eqnarray*} }
\newcommand{\eeqa}{\end{eqnarray*} }

\newcommand{\pn}[2]{ 
#1  


\hspace{1em} #2
}



\title{ Outline}




\author{}
\begin{document}
\maketitle


\[
	H = \underbrace{\frac{p^2}{2m} + e\Phi }_{H_0} + V(x) 
\]

\section{Introduction}
\subsection{ Description of the problem}
Describe the general problem we investigate: 

	Two charged particles in a loosely bound system, each particle having arbitrary spin

	Calculate corrections to $g$-factor for particles in such a system

Talk about experimental motivation, measurements in O and C molecules.

Discuss how the problem is nonrelativistic, the precision we need, and the types of approximations we can then make.

\subsection{theoretical background}
Briefly discuss various theoretical approachs:
NRQED approach

Discuss prior work.

spin 1/2 approach

BMT equation

Khriplovich general spin formalism

Faustov ?


\section{Background for Nonrelativsitic Quantum Electrodynamics}
\subsection{Effective theories}
Some worked examples
\subsection{Nonrelativistic Quantum Electrodynamics}
General approach

Example with muonium


\section{The case of spin one-half}
General discussion of properties of the spin 1/2 case, in both relativistic and nonrelativistic cases.

Calculation with Foldy-Wouthyusen method, starting from equations of motions

Calculation with NRQED approach, calculations with diagrams.

\section{The case of spin one}
General discussion of properties of the spin 1 case, in both relativistic and nonrelativistic cases.

Contrast with spin 1/2.

Calculation with Foldy-Wouthyusen method, starting from equations of motions


Calculation with NRQED approach, calculations with diagrams.


\subsection{Spin-1 through diagrams}
The plan is to start from the exact spin-1 theory, and obtain the NRQED Lagrangian.

Start with relativistic Lagrangian.

Derive the electromagnetic vertices.

Discuss wave functions and find the connection between the relativistic and non-relativistic free theories.

Find one-photon terms by calculating scattering off an external field.

Find two-photon terms by calculating Compton scattering.

Write down NRQED Lagrangian.

\subsection{Spin-1 through equations of motion}

\subsubsection{Equations of motion}
Derive the Euler-Lagrange equations from the spin-1 relativistic lagrangian

Contrast the form of the wave functions here with in the previous approach.

Eliminate non-dynamic fields, and solve for the energy.


\subsubsection{Non relativistic wave functions}
Transform so that particle-antiparticle are uncoupled.

Find NR single-particle Hamiltonian.

Show that there are no corrections from a FW transformation that enter at our level of precision.

Finally find the nonrelativistic Hamiltonian.


\section{The case of general spin}


Describe the features of both the relativistic theory:

 Definition of wave functions; spin degrees of freedom
 
 Spin operators $S$ and $\Sigma$


Describe nonrelativistic theory along the same lines.

Describe the connection between the two free theories.

\subsection{NRQED Lagrangian for general spin}
Our goal is to calculate the NRQED Lagrangian.  

Discuss what constraints we have: symmetries, hermiticity, etc.

Given assumptions about the strengths of the EM field and the momentum of the particles, we need up to $\frac{1}{m^3}$ terms.

Write down all allowed terms in the Lagrangian up to that order.

Note that only up to two-photon interactions appear, and they can be fixed by gauge invariance from the one-photon interactions.


\subsection{One-photon interaction in relativistic theory}
Now take the relativistic theory, and consider what the one-photon interaction will look like.

Constrained by Lorentz transformation properties and current conservation.

Show how only two bilinear terms are then allowed.

Their coefficients will be just charge and $g$-factor, with corrections at a higher order than we need.

Write down this general interaction.


\subsection{One-photon interaction in NRQED}
Express the current in terms of the nonrelativistic wave functions.

Thus, fix the NRQED coefficients.

Write down this general-spin NRQED Lagrangian.
















\section{Corrections to $g$-factor in nonrecoil case}

Write the general NRQED Lagrangian.

Write as a Schroedinger like nonrelativistic Hamiltonian.

Calculate corrections to the $g$-factor for S-orbitals.

Show that no higher order terms in perturbation theory enter.


\section{Recoil case}
From the NRQED Lagrangian, we can calculate the effective Breit potential.

\subsection{NRQED calculation}
Calculate that potential from the one-photon interaction diagrams in NRQED.

\subsection{relativistic calculation}
We can calculate the same process in the relativistic theory, to make sure it agrees.

\subsection{CoM transformation}
Transform coordinates to the center of mass system.

When an external magnetic field is present, we need an additional unitary transformation.

\subsection{$g$-factor calculation}

Calculate the corrections to the g factor, now taking into account recoil effects.


\section{Conclusion}








\end{document}