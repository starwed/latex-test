

\section{Effective Field Theories}

The origin of quantum field theories lies in the search for an exact theory of nature.  At first to formulate a theory of electromagnetism which would include both relativity and quantum mechanics, and later to also account for the weak and strong forces.

For a theory to be exact, it must account for behavior at all energy scales.  The dramatic success of quantum electrodynamics (QED) as a field theory suggested that pursuing such theories was a reasonable course to chart.  In the 1950s, the predictions of QED agreed with very precise measurements of quantities such as the gyromagnetic ratio.  

However, success in formulating field theories for the other interactions was not so easy to come by.  After much frustration it became a common belief that they were not the correct tool to approach the strong or weak interactions.  While alternatives were sought, field theories found a different use in condensed matter physics.  

Here, while technically the exact theory underlying the behavior was known---that is, the microscopic interactions between particles---it was not so useful in describing the macroscopic behavior of the system.  But, an {\it effective} field theory, valid for the scale of interest could be developed. 

This approach proved fruitful in particle physics as well as condensed matter.  It was possible to formulate ``renormalizable'' theories of the weak and strong force, where the interaction was cut off at some arbitrary energy scale, and the predictions of such a theory shown to be independent of the cut-off value.  Eventually nonrenormalizable theories were also accepted, the cut-off in such theories having a physical significance.  Such theories were effective theories, capturing the behavior of a system in a particular energy regime.  While not an exact theory in themselves, they can never-the-less produce results of arbitrary accuracy in the scale of interest.

\subsection{How an effective theory works}

An effective theory acknowledges some fundamental energy scale.  Above this energy new processes and physics emerge, but below it the effective theory can capture all the interactions.  Of course the high energy physics still has some effect on the low, but it occurs as corrections to the low energy processes.  In the language of Feynman diagrams, the processes allowed at low energy can contain internal lines and loops, corrections which represent the effect of virtual particles.  Regardless of the energy of the external particles, there will still enter some corrections from all allowed processes.

The key to formulating the effective theory is to discard detailed information about these processes.  Rather, all allowed low energy processes are considered in the Lagrangian, but with coefficients that conceal the energy dependent behavior.  It is exactly the same case as say, the form factors of particles in QED.  There is a lot of complicated physics going on inside a proton, but if only low energy processes are dealt with, the exact details are hidden.  Effectively the proton will behave like a fundamental particle, and all the information about the exact theory is inside its form factors.

So the idea is this: to write down a Lagrangian which takes into account all possible processes which can occur in the regime of interest.  In front of each term will be a coefficient which may depend on the energy of the particular interaction.  Such a Lagrangian will capture the behavior of the system of interest with arbitrary precision, if the coefficients can be made known.

Now, in reality not {\it all} terms will be written down.  In a renormalizable theory there is a limit to what terms may appear, and coupling constants with negative mass dimension are not allowed.  Remember that each term in the Lagrangian must have mass dimension four, and that each field (no matter what the type) has {\it some} positive mass dimension. It becomes clear that terms involving a great number of fields are forbidden in a renormalizable theory, since their coefficient would necessarily be of negative mass dimension.

But in a theory that may be nonrenormalizable, interactions between an arbitrarily large number of fields will be possible.   And so to write a complete Lagrangian would involve an infinite number of terms.  However, one part of the above still holds true: as the number of fields in a term increases, the mass dimension of the coefficient must decrease.  This indicates a suppression of such terms by the fundamental energy scale of the problem.  Since (by the very definition) all the fields included in the effective Lagrangian have energy scales small in comparison to that fundamental cut-off scale, these terms give smaller and smaller contributions to any particular calculation.

So in practice, once the precision desired for a calculation is determined, only a finite number of terms in the effective Lagrangian need be considered.

Now, once all relevant terms are written, there will still be many undetermined coefficients left in the theory, and unlike the form of the terms, the coefficients cannot be determined {\it a priori}.  If an exact theory is known and tractable, then the coefficients can be calculated from that.  Or, just as is done for fundamental constants in an exact theory, one might simply have to take them from experiment.

\section{Nonrelativistic Quantum Electrodynamics}
One particular effective theory of interest is nonrelativistic quantum electrodynamics (NRQED).  It is a perfect example of how, even when an exact theory is known, an effective theory, especially suited for a particular energy regime, may be more practical.

