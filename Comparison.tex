
\subsection{Fixing the nonrelativistic coefficients}

Having calculated the same process in both the relativistic theory and in the NRQED effective theory, the two amplitudes can be compared, thus fixing the coefficients of NRQED.

The NRQED amplitude \eqref{eq:nrqedScatter} is
\beq
\begin{split}
	iM =
		ie\phi^\dagger \Bigg( - A_0 +   \frac{ \v{A} \cdot \v{p} }{m} - \frac{  (\v{A} \cdot \v{p}) \v{p}^2   }{2m^3} 
		+ c_F  \frac{\v{S} \smalldot \v{B}} {2m}   	
		+ c_D \frac{ ( \partial_i E_i ) }{8m^2}	
		+ c_Q \frac{ Q_{ij} ( \partial_i E_j ) }{8m^2}	
	\\	+ c^{1}_S \frac{  \v{E} \times \v{p} }{4m^2}
		- (c_{W_1} -c_{W_2}) \frac{   (\v{S} \smalldot \v{B} ) \v{p}^2  }{4m^3}	
		-  c_{p'p} \frac{   (\v{S} \smalldot \v{p}) (\v{B} \smalldot \v{p})  }{4m^3} \Bigg )\phi
\end{split}
\eeq


While the relativistic amplitude was
\beq
\begin{split}
iM_{REL} = -ie \phi^\dagger \Big (
		 A_0  - \frac{\v{p}\cdot \v{A} }{m} 
		- \frac{g-1}{2m^2}\left\{ \v{S} \cdot \v{E} \times \v{p} + \frac{\Sigma^2}{12} \grad \cdot \v{E} + \frac{\lambda}{8}Q_{ij} \right\} 
		\\ - g\frac{1}{2m} \v{S} \cdot \v{B}
		+ \v{S} \cdot \v{B} \frac{\v{p}^2}{2m^3}
		+ \frac{g-2}{4m^3}(\v{S} \cdot \v{p} )( \v{B} \cdot \v{p})
	\Big ) \phi
\end{split}
\eeq

Comparing the two, the coefficients are:
\beqa
	c_F &=& g \\
	c_D &=&	\frac{(g-1)}{3} \Sigma^2	\\
	c_Q &=&	\frac{g-1}{2} \lambda	\\
	c^1_S &=& 2 (g-1)	\\
	(c_{W_1} - c_{W_2}) &=&	2	\\
	c_{p'p}	&=& (g-2)		\\
\eeqa
